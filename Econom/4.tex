% This file was converted to LaTeX by Writer2LaTeX ver. 1.4
% see http://writer2latex.sourceforge.net for more info
%\documentclass[twoside,a4paper]{article}
%\usepackage[utf8]{inputenc}
%\usepackage[T2A]{fontenc}
%\usepackage[russian,english]{babel}
%\usepackage{amsmath}
%\usepackage{amssymb,amsfonts,textcomp}
%\usepackage{color}
%\usepackage{array}
%\usepackage{hhline}
%
%\begin{document}
\section[Пример экономического учения (меркантилизм)]{Пример экономического учения (меркантилизм)}
{\raggedleft\itshape
Жил-был король однажды,
\par}
{\raggedleft\itshape
При нем блоха жила.
\par}

\subsection[Предметы роскоши в Европе]{Предметы роскоши в Европе}
Эпидемии в античной древности уносили до половины населения. На Западе не имели никакого понятия о гигиене и санитарии.
Единственным средством от моровой язвы были восточные благовония. Интенсивная \ торговля с Востоком была делом жизни и
смерти не только в античное время, но и позднее. Именно поэтому так высоко ценились золото и серебро, которые были
единственным платежным средством в торговле с Востоком: варварам Запада больше нечего было предложить цивилизованному
Востоку. Ладан и мирра использовались не только как антисептики, но и как психотропные
средства.\footnote{Парфюмерия — \foreignlanguage{english}{per fumum} =
через дым.} Церковь хранит эту санитарную традицию, давно забыв о ее происхождении и пользе. Фунт
коричного масла имел в Риме цену десятилетнего плебейского пайка, доход посредников составлял 10~000\%. Пряности —
имбирь, корица, черный перец, гвоздика — тоже имели лекарственное значение. Они же обеспечили европейцев возможностью
запасать мясо, особенно для солдат и матросов: солонина быстро портилась, а мясо с пряностями открыло океанские
просторы для авантюристов, исследователей и охотников за пряностями.

Белить потолки с ладаном вошло в обычай для борьбы с насекомыми. Не любили насекомые и шелк, отчего он и был и в моде, и
в цене: фунт шелка стоил фунт золота.

Аристократы, как и все состоятельные люди, одевались в шелк, окуривали дворцы ладаном, так как не умели мыться. Учебники
учат, что Европа ввозила с Востока предметы роскоши, не жалея на это золота и серебра. Разве жизнь и здоровье —
роскошь?

Нечистоплотная Европа очень многим обязана Востоку. «Предметы роскоши» привозили, а позаимствовать канализацию оказалось
куда более трудным делом. Мрамор вместо дерева для лестниц во дворцах применили только потому, что мрамор легче убирать
и мыть, а лестницы служили обитателям дворцов туалетом. (Туалеты \flqq пролетной системы\frqq на
черных лестницах сохранялись в Петербурге и в~\foreignlanguage{english}{XIX} веке. В конце века ватерклозеты имела
только треть, а ванны одна из трехсот двухкомнатных квартир.) Выгребные ямы появились в Париже только в
\foreignlanguage{english}{XVI} веке (и сохранились в Ленинграде до середины ХХ века).

Постоянная утечка золота и серебра породила в Европе проблемы и специалистов по их разрешению. Добыча драгоценных
металлов из рудников — дело хлопотное. Проще добывать металл из монет, соскабливая его с той стороны монеты, где нет
чекана. Когда чеканы стали ставить с обеих сторон (орел и решка), осталось соскабливать металл с ребра. Потеря четверти
и трети веса стала делом обычным, а в Англии дело доходило и до половины. Именно в Англии Исаак Ньютон провел
единственную успешную денежную реформу: старую монету принимали и обменивали на новую не по весу, а по номиналу, а
новые монеты были круглыми и имели защитный чекан по ребру (зубчики); только в 1980 году благодарные потомки выпустили
банкноту достоинством в один фунт стерлингов с портретом сэра Исаака Ньютона. Задолго до Ньютона книжку «О способе
чеканки монеты» написал Николай Коперник. Правда, Польше это не помогло.

В Европе были рекордсмены по денежным реформам. Вена сумела не в ущерб казне за 150 лет выпустить 150 видов одной и той
же монеты. Во Франции король-солнце Людовик \foreignlanguage{english}{XIV} примерно один раз в пять лет обменивал по
номиналу старые монеты на новые, которые были полегче. Из одного и того же слитка серебра в конце
\foreignlanguage{english}{XV} века чеканили 10 ливров, в конце \foreignlanguage{english}{XVI} века — 30, а перед
реформой Лоу — 100. Девальвация — дело привычное.

\textbf{Новый свет }дал европейцам вожделенные золото и серебро для торговли с Востоком как через Тихий океан (порт
Акапулько), так и через Европу. Пиратам было — хоть разорвись!

Война, торговля и пиратство — \newline
Три вида сущности одной.

Однако хлынувшие в Европу золото и серебро породили новые проблемы. Для них, как и для всякого товара, превышение
предложения над спросом привело к падению цен: деньги стали дешеветь. Удешевление денег вся Европа быстро почувствовала
в росте цен: за 100 лет покупательная способность серебра упала в пять раз. Заработная плата и земельная рента быстро
обесценивались. Откуда такая напасть? Только в конце \foreignlanguage{english}{XVI} века было замечено, что эпидемия
инфляции начинается в Испании. За вывоз денег там была введена смертная казнь — не помогло. Испания до сих пор не может
выбраться из пучины бедности, куда ее ввергла богатая колониальная добыча.

\subsection[Идеология меркантилизма]{Идеология меркантилизма}
\textbf{Экономическая мысль и практика }не дремали не только в Испании: в Англии в \foreignlanguage{english}{XV} веке
был принят закон, обязавший иностранцев истратить в Англии всю выручку за привезенные ими товары, а своих подданных —
часть экспортной выручки привозить на Родину (\textit{а мы-то куда смотрим?}). Власть в европейском государстве
оттесняла церковь, подчиняла экономику своим целям и вводила новый статус человека — \foreignlanguage{english}{homo}
\foreignlanguage{english}{oeconomicus}.

Практики экономического \foreignlanguage{english}{mainstream}{}'а за 400 лет сформировали идеологию, \textit{позже}
(Адам Смит) названную \textbf{меркантилизмом} (от \foreignlanguage{english}{mercante},
\foreignlanguage{english}{merchant} — купец). Единства мнений не было (из-за отсутствия в то время цензуры, научных
советов и конференций, Высшей аттестационной комиссии, академий, телевидения и других средств массового поражения
мысли), не было даже привычного для нас единства терминологии или хотя бы внимания к этому вопросу.

Меркантилисты сосредоточили внимание на государственной казне, на питающей ее торговле, особенно — на внешней торговле.
Основной объект — деньги, кровь экономического организма. Ведь именно денег недоставало для торгового обмена с
Востоком. Ранние меркантилисты делали акцент на денежном балансе: ограничивать импорт и стимулировать экспорт.
Впоследствии перенесли акцент на торговый баланс — сначала на частный, а затем ввели общий. Деньги создают торговлю, а
торговля создает деньги. \flqq Выгода — вот чему служит обмен товарами\frqq.
(Д.Дефо\footnote{Его знаменитый трактат читают дети.}).

Присмотритесь к основным положениям меркантилистов (в современных терминах):

\begin{itemize}
\item объектом внимания является страна (а не какая-то фирма);
\item денежный баланс должен быть активным (ранний этап);
\item торговый баланс должен быть активным — продавать больше, покупать меньше;
\item защищать отечественного производителя от импорта;
\item поощрять отечественного производителя к экспорту;
\item поощрять мануфактуры (предоставить сельское хозяйство самому себе);
\item создавать новые рынки (например, колонии и сателлитов);
\item запретить вывоз денег из отечества («материнской страны»);
\item обеспечить рост населения для поддержания дешевизны труда;
\item снижать государственные расходы;
\item добиваться превышения доходов над расходами;
\item иметь жесткий порядок выдачи привилегий и монопольных прав;
\item денег в стране должно быть много.
\end{itemize}
{\itshape
Нет ли чего знакомого?..}

Меркантилисты не видели экономику в целом, поэтому причисление их к отцам-основателям меркантилизма как экономической
\textit{теории} весьма спорно.

Деньги, как навоз, плохи лишь, когда лежат без дела. Чем полнее казна, тем сильнее армия, тем лучше работают чиновники,
в том числе судьи (по Уильяму Петти).

Но неравновесный приток в страну драгоценных металлов оборачивается бедой, как в Испании. Приток золота повышает цены
\textit{внутри} страны (и снижает платежеспособный спрос), а отток понижает цены \textit{вне} страны (и повышает
платежеспособный спрос), так что желаемое соотношение экспорта и импорта \textit{принципиально не удается} поддержать
(оно возникает в стране, \textit{из} которой вывозят золото). Тем не менее, политика «разори соседа» (относительно
торгового баланса) жива до сих пор.

Аргументы в защиту экономической (и всякой другой) безопасности страны столь же известны, сколь и живучи: экспорт важнее
импорта, оборона важнее благосостояния («пушки нужнее масла»), армия — школа жизни и т.д.

{\itshape
Ошибочны ли воззрения меркантилистов?}

Избыток денег снижает процентные ставки на кредиты (хотя охотников получить кредит становится больше, а согласных отдать
деньги – меньше). Умеренная инфляция стимулирует инвестиции и занятость, особенно в компании с низкими процентами на
кредиты. На эти соображения можно и возразить: уровень занятости определяла система землепользования, а не
инвестиционная политика, не говоря уже о добровольной безработице \textit{бездельников}. А проценты на кредиты всегда
тем выше, чем беднее страна; экономика \textit{охлаждается} с повышением процентной ставки.

Меркантилисты не были экономистами-теоретиками в нашем смысле, поэтому критика их \textit{замечаний} как
\textit{положений} обнажает \textit{наш }способ понимания \textit{их} проблем — и ничего более.

Нельзя, однако, не обратить внимания на заблуждение, столь же давнее и столь же прочное, как экономическая мысль: деньги
${\equiv}$ богатство.

Вместе с тем именно меркантилисты заложили основы количественной теории денег. В~чикагском
уравнении\footnote{\textit{M} = money = запас денег; \textit{V} = velocity =
скорость обращения денег; \textit{P} = price = уровень цен; \par 
\textit{T} = trade = объем сделок. \textit{MV} ${\equiv}$ \textit{PT}} они видели в первую очередь эффект влияния \textit{М} на \textit{Т}, а не на \textit{Р }— деньги стимулируют торговлю.

{\itshape
И \ еще одна \ трудность: сравнительная статика не есть динамика.}

\textbf{Кантильон }увидел, что увеличение \textit{М} не просто изменяет уровень цен, но \textit{динамически} изменяет их
структуру: рост доходов золотодобытчиков → рост расходов на потребительские товары → рост доходов производителей
продовольствия → падение реальной заработной платы → рост номинальной заработной платы → рост цен → и далее везде.
Схематично описанный динамический эффект теперь называют \textit{эффектом Кантильона}: увеличение денежной массы
изменяет относительные цены. Если деньги инвестированы, процентная ставка падает, а если проедены — процентная ставка
растет.

\textit{Мы выдавали лишнему населению Севера деньги на переезд в другие регионы, а деньги пошли \ на ковры и хрусталь:
«красиво жить не запретишь». \ Эффект Кантильона — предмет как экономии, так \ и психологии.}

Приток золота и серебра ${\rightarrow}$ повышение внутренних цен ${\rightarrow}$ потеря позиций на внешнем рынке
${\rightarrow}$ снижение экспортных потоков ${\rightarrow}$ рост импорта ${\rightarrow}$ отток золота и серебра: эффект
сообщающихся (глобализованных) сосудов. Беда в том, что из экономики уходят не деньги, а уходит капитал как фактор
производства.

Первыми меркантилистами в России были Алексей Михайлович и его талантливый сын\footnote{Крепостники,
конечно, были меркантилистами, но они об этом не знали.}. \textbf{Последнего} меркантилиста \ пока не видно.

\subsection[Практика]{Практика}
Меркантилизм возник как инструмент борьбы государства с церковью: сторонники церкви апеллировали к морали, ее противники
– к национальному интересу. Для облегчения сбыта отечественной рыбы в Британии были введены два «рыбных дня» в неделю,
а покойников распорядились обряжать в \textit{отечественное }шерстяное платье (\textit{а куда мы-то смотрим?..)}.

Концепция \textit{общего }торгового баланса (Д.~Дефо) привела к пониманию обоюдной выгодности торговли. Дефо не разделял
тезиса «бедное население – это хорошо»: «Если заработная плата – низкая и жалкая, то такой же будет и жизнь». Концепция
эффективного спроса стала основой \ теории Кейнса и практики Франклина Рузвельта.

Меркантилисты положили начало нормативному подходу в экономической теории: экономическое процветание нации достигается
принятием правильных законов. «Государь – капитан корабля экономики». В 1615 году вышел труд \flqq Трактат
политической экономии\frqq, и появился привычный для нас термин.

Великим меркантилистом-практиком выступил Кромвель («Навигационный акт»). Меркантилисты одобряли разорявшую Англию
практику экспорта шерсти и импорта сукна. Находчивые островитяне изобрели обработку хлопка и создали систему импорта
сырья и экспорта тканей; пришлось завоевать Индию. Как только в Англии парламент окончательно одолел короля, наступил
закат меркантилизма.

Во Франции в экономической культуре абсолютной монархии меркантилизм расцвел. Во Франции (именно во Франции) авантюрист
Джон Лоу помог королю обеспечить нацию деньгами: ввел бумажные банкноты, разрешенные к уплате налогов. Лоу учредил банк
для кредитования предпринимателей под низкие проценты (до 2!) и Миссисипскую компанию, акции которой росли в цене (от
160 ливров до 18000) благодаря эмиссии банкнот. Возникла – и, конечно, обрушилась – первая финансовая пирамида. Идеи
Лоу опередили время — теперь мы считаем его успешным теоретиком, а не прогоревшим практиком. Лоу сбежал, «но дело его
живет». Тем временем Кольбер содействовал созданию мануфактур, чтобы обеспечить товарами выгодный королю экспортный
поток, так что во Франции \textit{меркантилизм} стали именовать \textit{кольбертизмом}.

Меркантилизм четыреста лет выполнял государственную регламентацию экономики, обеспечивал протекционистскую защиту
национальных экономик, его практика вошла в теорию Кейнса.

\subsection[Другие направления мысли]{Другие направления мысли}
Социальный (и экономический) ресурс католицизма был практически исчерпан, поэтому в Европе в условиях гравитационного
коллапса под лозунгом возвращения к истокам возникло новое течение мысли и действия — протестантизм. Христианские
фундаменталисты того времени оставили нам экономическую этику, ошибочно считающуюся \textit{причиной} тогдашних сдвигов
в экономике. Успех в труде (в любом труде) был предложен ими как признак божьего благоволения, грамотность из
привилегии стала обязанностью, место десятинного налога заняла (обязательная) благотворительность, прибыль и
ростовщичество были реабилитированы.

Католицизм ассоциируется сегодня с экономической отсталостью, протестантизм — с экономическим успехом. Протестантский
фундаментализм — не причина, а один из результатов коллапса. Фундаментализм всегда есть форма, в которой выступает
второй по значимости ресурс человечества, который высвобождается в ходе коллапса, — \textit{глупость}.

Другие фундаменталисты сочиняли Утопии: Мор («Золотая книжечка о лучшем устройстве государства») и Кампанелла («Город
Солнца»).

Никаких успехов за их фанатичными последователями сегодня пока не числится.

\subsection[Наследие]{Наследие}
Национализм и культ власти – идейные спутники меркантилизма.

Меркантилизм ввел экономического человека (homo oeconomicus), концепции денежного и торгового баланса. Критика
меркантилизма породила \textit{классическую школу}.

Меркантилисты побудили экономическую мысль осознать связь между денежным рынком и реальным сектором экономики.

Кредитные деньги и кредитная экспансия – фундамент (неустойчивой!) банковской системы с динамическим ссудным процентом и
инструментами реструктуризации раковых образований. Процесс инфляции ввели в практику меркантилисты, а термин и
квалифицированные операторы\footnote{Отцы долгов и сыновья кредитов.} появились позднее.

Меркантилисты ввели вечный предмет обсуждения – \textit{нехватку денег.}

Меркантилисты ввели в научный и практический оборот \textit{политическую арифметику} – (социальную) статистику и
концепции национального дохода, скорости обращения денег и (трудом) добавленной ценности.

Меркантилисты изобрели термин \textit{\ политическая экономия} и создали экономистам статус советников государя.

{\itshape Распространение мыла смыло меркантилистов.}

\subsection[Заметки на полях]{Заметки на полях}
Коперник указал (нам) четыре причины упадка государства: раздоры, смертность, неплодородие земли, обесценение денег.

Деньги должны крутиться: деньги работают, если они \textit{капитал}, а не сокровище.

Деньги — прежде всего капитал, а уж потом средство обращения.

\ «Прибыль одного — убыток другого». \textit{Другой — враг}.

Экономист — новая ученая профессия. Ученый занят не практикой, а знанием. Практику все понятно и очевидно (не хватает
только денег), ученому все непонятно и неочевидно.

«Труд есть отец и активное начало богатства, а земля — его мать». Так было введено понятие о факторах производства.

Предприниматель — тот, кто взял на себя ответственность за снабжение рынка.

Сбор и анализ экономической информации (экономической статистики) жизненно важны.

Если не хватает денег, надо открыть Банк.

{\selectlanguage{russian}\bfseries
Современные заметки}

Практические \textit{частные} рецепты меркантилизма, не затрудняющие утомительным анализом \textit{целого}, легко
усваиваются невежественными экономистами и политиками\footnote{Золото Колымы.} и дорого обходятся
населению.

По 20 различным книгам, интенсивно продаваемым в течение года в стране, можно узнать направление мысли всего народа.
\textbf{В России }в числе этих 20 книг нет ни одной книги по экономической культуре или ее истории.

\subsection[Санкт{}-Петербург — столица меркантилизма]{Санкт-Петербург — столица меркантилизма}
Петр \foreignlanguage{english}{I} приказал устроить порт в устье Невы, чтобы \textit{на своей территории} диктовать
иностранным купцам не цены, а \textit{способ оплаты русских товаров}. Царь потребовал оплачивать товары
\textit{мелочью}. На одном из первых судов прибыло 256 мешков мелких серебряных денег. Наши «специалисты» пропускали
монетки размером с ноготь мизинца, а монетки большего размера рубили, чтобы монеток стало больше. Эти \textbf{деньги}
стали настоящей основой денежного обращения в России, ими выплачивали жалованье, ими же \ оплачивали государственные
закупки. На ярмарках стали торговать на деньги. В результате стало возможным взимать налоги и иметь настоящий
государственный бюджет. Мы освоили чеканку мелочи. Важнейшими сооружениями в Санкт-Петербурге были Порт, Крепость и
Монетный двор.

\textit{Перемещение банков в Питер сегодня можно счесть продолжением петровского меркантилизма…}
%\end{document}
