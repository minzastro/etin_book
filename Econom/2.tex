% This file was converted to LaTeX by Writer2LaTeX ver. 1.4
% see http://writer2latex.sourceforge.net for more info
%\documentclass[twoside,a4paper]{article}
%\usepackage[utf8]{inputenc}
%\usepackage[T2A]{fontenc}
%\usepackage[russian,english]{babel}
%\usepackage{amsmath}
%\usepackage{amssymb,amsfonts,textcomp}
%\usepackage{color}
%\usepackage{array}
%\usepackage{hhline}
%
%\begin{document}
\section[Эволюция денег ]{Эволюция денег }

Обмен подарками положил начало обмену сопоставимыми (эквивалентными) ценностями. Технологии изготовления изделий из
металлов выдвинули чистые (благородные) металлы на ведущее место в процессах обмена.

\subsection[Богатство Крёза]{Богатство Крёза}

В середине \foreignlanguage{english}{VII} в. до н.э. в Малой Азии, в местах, где издавна научились выплавлять металлы (и
держали это умение в секрете), появились овальные слитки из сплава золота и серебра\footnote{Металлов,
не поддающихся коррозии.} величиной примерно с большой палец руки. На эти слитки с помощью штампа было нанесено
изображение головы льва. В Лидии эти слитки получили признание на рынках: невиданные \textit{деньги} чрезвычайно
облегчили торговлю. Деньги потеснили бартер. Сразу же нашлась и реализовалась вторая функция денег –
накопительная.\footnote{Девушка, накопившая \flqq лепешки\frqq с головой льва, могла
{\textquotedbl}купить\frqq себе мужа, не ожидая сговора родителей.} В следующем веке
{\textquotedbl}царь\frqq Лидии Крёз (560 – 546 гг. до н.э.) отделил золото от серебра и наладил выпуск денег
двух видов: золотых и серебряных. Владетель Персии Кир разгромил Крёза и захватил Лидию.\footnote{Во
время грабежа Крёз заметил Киру: \flqq Твои солдаты разоряют тебя и себя\frqq.} Тем временем
лидийские деньги \flqq покорили\frqq Грецию: греки принялись за добывание руды и выплавку серебра
для изготовления \textit{денег} по образцу лидийских.


Греческие \textit{базары} превратились в \textit{рынки}, расцвела торговля. Греки (не все…) уразумели, что город
(\flqq полис\frqq) может существовать самостоятельно, опираясь на рынок и денежную систему. Именно
поэтому полисы не объединились в союз=империю. Когда над греческими полисами нависла угроза вторжения Кира (главы
тоталитарной системы натурального товарного накопления), то они не заметались в поисках царя, императора или
президента, а просто собрали деньги, чтобы на них нанять и вооружить профессиональную армию
(\flqq контрактников\frqq). Киру не повезло: греков не удалось разгромить. Собранные деньги недолго
оставались \flqq в резерве\frqq: на них построили великие архитектурные сооружения, создавшие лицо
Древней Греции.


Духовная жизнь Древней Греции, впоследствии прославившаяся в неграмотной Европе, возникла не на пустом месте: деньги
требовалось считать, денежные суммы требовалось складывать и (ничего не поделаешь) вычитать, даже умножать (не дай бог,
делить!). Счет успешно рационализировал мышление. Арифметика стала не только ведущим элементом экономической культуры,
она послужила фильтром для выделения людей, имевших способности к выполнению упомянутых выше \textit{абстрактных
действий. Математика (\foreignlanguage{english}{mathema} = наука, теория) стала вершиной формировавшейся науки. Деньги
породили науку!}


Деньги создали социальные связи – и укрепили их взамен родственных. Появились денежные налоги, приношения, выкупы.
Центром жизни стал рынок. Появилось Богатство, а вместе с ним и с его помощью появилась (европейская) Культура. К
четырем элементам (воздух, вода, огонь, земля) добавился \textit{пятый}: \-деньги
(\flqq \textit{квинт}эссенция\frqq).

{\itshape
Рыночный греческий язык естественно стал впоследствии языком Нового Завета: его на рынках проповедовали рабы и
вольноотпущенники. Так общая коммерческая (=экономическая) культура обеспечила текстовый фундамент новой религии.}

\subsection[Рим]{Рим}

Римляне быстро переняли греческое новшество – деньги. На монетах, посвященных Юноне Предупреждающей
(\foreignlanguage{english}{Monere}), чеканили слово \foreignlanguage{english}{\textit{Moneta}}. В 269 г. до н.э. начали
чеканку денария. Деньги потекли!\footnote{\ \foreignlanguage{english}{Curru → currency}. Монетарная экономика Рима
невероятно интенсифицировала торговлю и государственную деятельность. В 63 году до н.э. бюджет Рима достиг величины 340
млн. сестерциев (сестерций = ¼ денария). Рим регулярно перечеканивал деньги, уменьшая в них содержание серебра. Деньги
\flqq подешевели\frqq: во \foreignlanguage{english}{II} в. н.э. мера пшеница стоила полденария, а в
следующем веке – уже 100 денариев.\footnote{Тем самым вырос ВВП!}} Государственные расходы
\flqq требовали\frqq повышения налогового давления \ и конфискации имущества у
\flqq предателей\frqq. Власть и армия были обеспечены для новых захватов и грабежей (экстенсивная
экономика). Оставалось и на раздачу пшеницы и оливкового масла \textit{пролетариям}.


Растраты требуют \flqq реформ\frqq. В 301 году Диоклетиан заморозил все цены и заработки – и товары
исчезли. Налоги стали вносить в товарах, продуктах и в труде: \flqq \textit{денег нет}\frqq. Он же
ввел наследование профессий и запретил продажу земли. Ему же принадлежат сохранившиеся по сей день экономические
институты государственного производства вооружений и государственного транспорта (\textit{ничто не ново под Луной}). А
денег все нет! В 303 году состоялось ограбление всех христиан. Опять не помогло… Тем временем сообразительные крестьяне
\flqq предали\frqq императорскую власть: многие продали (=отдали) землю латифундистам и нанялись к
ним батрачить, чтобы не платить налогов, размер которых превзошел объем производства.


\textit{Что делать?} (вечный, не только \textit{римский} вопрос). Константин простил христиан, вернул им храмы и
земельные участки под ними, но награбленное не вернул, так как оно стало достоянием Империи (Римского Народа).
Константин приступил к ограблению бесстыдно разбогатевших языческих храмов и самих язычников. В Риме императору стало
трудно жить, и он перенес столицу в Константинополь: легче создать новую власть и новую столицу, чем выплачивать старые
долги.


В 476 году монетарная экономика рухнула, население распавшейся Римской империи (вместе с варварами) на тысячу лет
вернулось в безденежную сельскую экономику (к натуральному хозяйству и бартеру). Люди разучились считать и читать,
боялись денег. Эту эпоху мы теперь называем Средневековьем. Деньги выжили в Константинополе.


Денежные центры перемещаются «в одном направлении»: Шумер → Египет → Греция → Рим → Северная Европа → Британия → США.

\subsection[Культурное наследие старых денег]{Культурное наследие старых денег}

Похолодание в железном веке охватило период примерно с 800 года по 200 год до н.э. На этот же период приходится
оживление торговли между Западом и Востоком. Потоки золота и серебра устремились из Европы на Восток
(\flqq товары\frqq Европы не прельщали жителей развитых стран
Востока).\footnote{Позже, за один только 77 год н.э., из Рима в Индию ушло 550 млн. сестерциев.}
Именно в этом периоде жили Конфуций и Лао-цзы, Будда и Заратустра, Гомер, Парменид, Гераклит, Платон и Архимед. На этот
период приходится возникновение империй Цинь Ши-хуанди и Хань в Китае, в Индии возвысились Маурьи, Кушаны и Гупты. В
это время были созданы Упанишады, творили пророки Илия, Исайя и Иеремия.


Новый взлет (на Востоке!) наступил в III – VII веках н.э.: Магомет, даосы, поэзия в Индии и в Китае.


Какое наследие оставят \textbf{\textit{новые деньги}}?

\subsection[Банкиры и банки]{Банкиры и банки}

Крестовые походы, как и любые другие, были коммерческим предприятием. Один из монашеских орденов
(\flqq братков\frqq храмовников — тамплиеров) быстро создал международную банковскую корпорацию в
атмосфере интеллектуального вакуума европейского Средневековья. А что Власть? – Король Филипп
\foreignlanguage{english}{IV} Красивый приказал согнать около 50 тамплиеров – и их сожгли живьем. Деньги и все
имущество тамплиеров пополнили королевскую казну.


Новая банковская сеть возникла в Италии. Вместо займов банкиры продавали векселя (от 8\% до 12\% годовых). Деньги
хранить и перевозить тогда было так же опасно, как и ныне, а вексель выдавался поначалу на конкретного предъявителя, но
неграмотные (тогда) бандиты не умели читать, да и не понимали, что такое вексель. Количество новых бумажных
\flqq денег\frqq быстро росло, появились чеки, возник (заимствованный из Египта) бухгалтерский учет
с приходом, расходом и сальдо (приход и расход вели отдельно по той причине, что возникавшие при вычитании
отрицательные числа были не по уму).


Банкиров презирали, в Голландии их не допускали к причастию, ни в один \flqq порядочный дом\frqq их
не пускали с парадного хода. Монархи не раз попросту грабили и изгоняли заимодавцев, прощая им свои долги.


Банкиры во Флоренции решили \flqq войти в элиту\frqq: наняли архитекторов и построили себе замки и
виллы получше графских и герцогских (у банкиров были деньги и не было скупости), заказали себе роскошные
\flqq модные\frqq костюмы и прически, заказали людям искусства картины и скульптуры с характерным
упором на изображение людей – чтобы в окружении людей можно было и себя почувствовать человеком (это и назвали
\flqq гуманизмом\frqq). Особенно преуспели в этом деле сыновья разбогатевшего торговца Медичи –
банкиры Козимо и Лоренцо. Они и создали культуру, названную в середине \foreignlanguage{english}{XIX} века Ренессансом
(Возрождением). Вот так-то!


На эпоху банкирского Ренессанса закономерно пришлось подлинное Возрождение математики и абстрактного мышления.
\flqq Арабские\frqq цифры открыли новую арифметику тем, кто способен был туда войти, в 1487 г. Лука
Паччоли выпустил трактат по бухгалтерскому учету, а в 1585 г. кассир Стевин опубликовал таблицы процентов и ввел
основные алгебраические обозначения. Математика сделалась местом обитания абстрактно мыслящих ученых – достаточно
назвать Коперника и Галилея, Декарта и Ньютона.


На языке и в стиле рынка писал Монтень, Данте \flqq превратил\frqq рыночную варварскую латынь в
итальянский язык, у Шекспира царили любовь, война и \textit{деньги}. Сегодня банкиры и валютные спекулянты (Сорос) тоже
\flqq делают культуру\frqq.


Королям Испании нужны были деньги, поэтому в 1492 году Колумб отправился \flqq открывать
Америку\frqq, в том же году из Испании были изгнаны евреи и мавры (разумеется, без денег и без
имущества).\footnote{Вакуум заполнили умевшие считать, читать и писать итальянцы, голландцы и немцы.
Испания по сей день пожинает плоды тогдашнего \flqq избытка денег\frqq и дефицита культуры.} На тот
же год пришлось учреждение Инквизиции - в истории нет ничего случайного!


С 1500 года по 1800 год в результате разграбления Нового Света 70\% мирового запаса золота (3000~тонн) и 85\% мирового
запаса серебра (150 000 тонн) попали на Иберийский полуостров. Цены там выросли на 400\%. В Англии, торговавшей с
Португалией, цены утроились, зато заработки тоже выросли – вдвое. Испания косвенно внесла вклад в денежную систему
Соединенных Штатов: в 1821 году Мексика стала самостоятельной, и мексиканский доллар стал платежным средством \ в США.
По его образцовому весу стали чеканить американский доллар. Канада выпустила свои доллары только в 1935~году.


В Англии Английский банк был частным (акционерным), власть не могла запустить в него руку, поэтому английская валюта
была поразительно прочной (национализирован в 1946 г). Именно в Банке Англии появились первые денежные агрегаты.


Нынешние агрегаты:\newline
\ \ М\textsubscript{0} = наличные деньги;\newline
\ \ М\textsubscript{1} = М\textsubscript{0} + чековые вклады в банках;\newline
\ \ М\textsubscript{2} = М\textsubscript{1} + депозиты в банках и строительных обществах, вклады коммерческих
банков;\newline
\ \ М\textsubscript{3} = М\textsubscript{2} + срочные депозиты физических лиц и государства;\newline
\ \ М\textsubscript{4} = М\textsubscript{3} + частные вклады в деятельность строительных обществ;\newline
\ \ М\textsubscript{5} = М\textsubscript{4} + национальные сбережения + казначейские векселя.


Сегодня уже обсуждаются агрегаты М\textsubscript{13} и М\textsubscript{14} на вершине этой пирамиды.


\ \ М\textsubscript{1 }\ — транзакционные деньги (для сделок), \ \ \ М\textsubscript{2} — квазиденьги, их можно
использовать не для всех сделок, но при надобности можно быстро конвертировать в М\textsubscript{1} без потерь.


В октябре 2005 года агрегаты \ оценивались: М\textsubscript{1} в \$1.37 трлн., М\textsubscript{2} \ в \$6.63 трлн.,
М\textsubscript{3} в \$10.06 трлн. Публикация данных на этом прекращена.


В США по акту 1863 года установлена государственная валютная система и учрежден Государственный банк. Сегодня
американской (и тем самым мировой) денежной системой управляет Федеральная резервная система, устанавливая ставку на
банковские кредиты и через нее управляя курсом доллара, инфляцией, безработицей, кризисами и ожиданиями в умах (у кого
они есть).


Динамики процентных ставок и самоубийств практически совпадают на протяжении всего Нового и Новейшего времени.

\subsection[Золотой стандарт]{Золотой стандарт}

В мире образовались два стандарта в обеспечении банкнот: золотой (Англия и \flqq весь развитый
мир\frqq) и серебряный (Китай и Мексика). Стандарт гарантирует беспрепятственный обмен банкнот на золото
(серебро), поэтому эмиссия денег ограничена: М\textsubscript{0} не может превзойти золотого запаса.


В США Золотой стандарт введен в 1900 году после ожесточенных споров.\footnote{Фрэнк Баум написал об этих
спорах памфлет \flqq Волшебник из страны Оз\frqq. — Читали?}


Началась охота за золотом. Франклин Делано Рузвельт в рамках Нового курса провел закон, обязавший всех граждан сдать
золото государству (на следующий год – и серебро). Конфискационные меры распространялись и на въезжавших страну
иностранцев. Закон о серебре отменили в 1963 году, о золоте – в 1974 году. Золотой запас США 10 лет тому назад
составлял 8000 тонн. В подвалах Манхэттена другие правительства хранят золотые запасы общим весом \ в 10 000 тонн.


Великобритания отменила золотой стандарт в 1931 году, Франция – в 1936 году.


Президент Никсон – из-за войны во Вьетнаме – последовал примеру Диоклетиана: заморозил цены, заработки и ренту, ввел
курс доллара относительно других валют, отменив золотой стандарт.\footnote{Никсон вошел в историю
денег: \flqq Доллар опирается на доверие, а не на золото\frqq.} Отмена золотого стандарта сняла
ограничения на эмиссию банкнот, породив массу проблем, актуальных и сегодня.


Демонетизация золота отражена в 1978 г. в Ямайских соглашениях Международного валютного фонда: специальное право
заимствования (из Фонда) ввело \flqq коллективную валютную единицу\frqq.

\subsection[Виртуальные деньги]{Виртуальные деньги}

В 1977 году в Брюсселе появилась \foreignlanguage{english}{SWIFT} – компания Общемировых межбанковских финансовых
телекоммуникаций, обслуживающая более 100 государств. Деньги в мешках и векселя в портфелях заменены базами данных,
операции можно выполнять в любое время суток. Физические лица обзавелись карточками со встроенными чипами. Карточки
стали кредитными, американцы живут в долг: общий объем задолженности превосходит объем ВВП; у американцев нет
сбережений, есть только долги – в долг жить проще и удобнее… Наличные деньги сохранились в гетто для бедных, среди
уголовников и проституток, в наркомирке.\footnote{Наших ворюг однажды поймали при попытке получить в
банке 50 000 долларов }\textit{\textcyrillic{наличными}}. Электронные деньги – на сегодня высшая форма абстрактных
денег – продолжают традицию денег: двигают вперед науку и технику. На валютном рынке продают/покупают только деньги и
только через компьютеры и сети связи. Из каждых 5000 клерков остаются на работе только 20, но от них требуют принимать
\textit{правильные} решения \textbf{\textit{мгновенно}}.


Суточный оборот виртуальных денег в 400 раз превосходит стоимость активов, стоящих за ними. Годовой доход от виртуальных
сделок — 50 триллионов долларов (= десяток годовых бюджетов США). Объем долгов в США (по всем агрегатам) исчисляется
сотнями триллионов долларов. Но не в этом дело: на валютном рынке решают не только экономические, но и политические
задачи. Власть не в состоянии контролировать виртуальный рынок, на нем господствует новая элита.


Власть без государства возрождается в новой ипостаси.

\subsection[Риски ]{Риски }

\textbf{Валютные риски} появились вместе с бумажными деньгами. Конвертируемость бумажных денег некоторое время
поддерживалась обещаниями, но с отменой золотого стандарта превратилась в фикцию. Валютные риски обусловлены
неопределенностью процентных ставок и уровня инфляции, множественностью валют. Котировки валют подвержены изменениям.
Обменные курсы валют могут быть официальными, оффшорными и «теневыми». На рынке валют все курсы являются плавающими.
Кросс-курсы с учетом транзакций вносят дополнительные риски.


Мгновенные обмены порождают спот-курсы. Сделки со сроком реализации договора порождают форвардные курсы. Сделки с правом
их перепродажи (переуступки) порождают фьючерсные курсы. Каждый из курсов имеет свои валютные рынки. Владельцы
фьючерсного рынка могут выступать гарантами сделок.


Сделка может быть сопряжена как с обязательством, так и с правом. Контракты с правом отказа — опционы — создают еще одну
степень свободы (= риска).


\textbf{Политические риски} порождены иммунитетом государств от гражданской ответственности за экономические последствия
политических и законодательных решений. Деятельность профсоюзов, союзов предпринимателей, общественных и религиозных
организаций и оппозиционеров порождает риски. Особым источником рисков является бюрократия. Таблицы политических рисков
(индексы стабильности) регулярно публикуют ведущие бизнес-журналы.


\textbf{Кредитные риски} порождены различиями в законодательстве стран и неопределенностями в правоприменительной
практике.


\textbf{Хеджирование рисков} расширяет валютный рынок. Одним из средств хеджирования рисков ТНК выступает глобальная
диверсификация, имеющая побочным эффектом колебания курсов.

\subsection{А как же мы?}

Предприятие, получив кредит, обычно опять кладет деньги в банк, банк снова может выдать из них кредит, в результате
М\textsubscript{2} в странах с нормальной экономикой обычно бывает в 4-5 раз больше, чем М\textsubscript{0}, у нас –
всего в 2-3 раза. Это оказывает серьезнейшее влияние на инфляцию в нашей \ стране. Государство начинает печатать деньги
— потому что не может допустить роста социального напряжения, особенно в условиях предстоящих выборов. Поскольку падает
платежеспособный спрос, то летят и показатели соответствующих продаж (для отечественных производителей, которые не
могут снижать цены). Стоимость кредита для них растет, а объемы его падают. В результате уменьшается \textbf{кредитный
мультипликатор }(=М\textsubscript{2}/М\textsubscript{0}) и автоматически изменяется денежная масса. Целые отрасли не
могут получить кредит. Предприятия начинают уходить в тень, то есть компенсировать убытки невыплатами налогов. Тем
самым уменьшается сфера обращения рубля — но общее-то его количество при этом не меняется! Начинает расти чисто
\textbf{монетарная инфляция}, поскольку уменьшение сферы оборота денег равносильно их увеличению при фиксированном
обороте. ЦБ поднимает рубль (для борьбы с инфляцией!), что наносит жесточайший удар по отечественным
товаропроизводителям. У нас не наличных денег мало, а кредитных. У нас не столько мал М\textsubscript{0}, сколько мало
отношение М\textsubscript{2}/М\textsubscript{0}. \textbf{Промышленная} \textbf{инфляция \ — }не потребительская\textbf{
— }в 2006 году\textbf{ }наверняка\textbf{ }повысилась до 30\%.
%\end{document}
