% This file was converted to LaTeX by Writer2LaTeX ver. 1.4
% see http://writer2latex.sourceforge.net for more info
%\documentclass[a4paper]{article}
%\usepackage[utf8]{inputenc}
%\usepackage[T2A]{fontenc}
%\usepackage[russian,english]{babel}
%\usepackage{amsmath}
%\usepackage{amssymb,amsfonts,textcomp}
%\usepackage{color}
%\usepackage{array}
%\usepackage{supertabular}
%\usepackage{hhline}
%\setlength\tabcolsep{1mm}
%\renewcommand\arraystretch{1.3}
%\title{Marina}
%\begin{document}
\subsection[Наши достижения]{Наши достижения}

\textbf{\textit{1. Голод}} — первое из пяти наших основных достижений. Всех не прокормить, да и потребности наши мы
наращиваем с большей скоростью, нежели средства их удовлетворения. За последние 25 лет численность постоянно голодающих
возросла с 500 до 800 миллионов.


Самоограничение сохраняется на Востоке, безграничный рост царит на Западе.


\textbf{\textit{2. Бедность}} — второе основное достижение, она стала социальным статусом.


За последнюю треть ХХ века внешний долг бедных стран вырос с 50 миллиардов до 2,6~триллионов долларов; за последние 20
лет они выплатили 5,4 триллиона только за обслуживание своего долга. Богатые страны ежегодно выплачивают 300 миллиардов
в виде субсидий \textit{своим}, чтобы они не стали бедными.


Голод и бедность мы поддерживаем системой налогообложения и государственной властью, религией и моралью, утопическими
экономическими программами и теориями. Бедные продолжают эксплуатировать богатых, ленивые — трудолюбивых, неспособные —
способных, плебс — элиту.


\textbf{\textit{3. Невежество — }}третье основное достижение. Человек \textit{может} стать мыслящим только \textbf{за}
\textbf{первые четыре года жизни}. Большинство людей не являются мыслящими, имеют опасно низкий порог абстрагирования,
находятся на уровне рефлекторного поведения — и имеют право голоса (и действия!) на всех уровнях. Невежественные
«красные кхмеры» после \ середины ХХ века уничтожили \textbf{\textit{грамотную}} половину населения Камбоджи.


Государственное невежество СССР во второй половине ХХ века обошлось ему в 132~миллиарда долларов, вложенных в «мирное
завоевание» третьего мира \ и тихо списанных.


325 миллионов детей в мире не посещают школу. В России около половины получивших среднее образование, как показывает
проверка, не в состоянии одолеть обязательную минимальную планку грамотности (в США — еще хуже!). Миллиарды людей
неграмотны и функционально неграмотны.


Низкий порог абстрагирования у большинства людей оставляет их на уровне дрессированных роботов. Большинство полагает
творчество (поиски нового) непозволительным: «Когда вагоновожатый ищет новые пути, трамвай сходит с рельсов».


\textbf{\textit{4. Разорение Земли }}— четвертое, решающее достижение. Разрушительное антропогенное воздействие не
только на биосферу, но и на Землю в целом становится общеизвестным.


Скотоводы опустынивали пастбища. Земледельцы превращают пашню в солончаки, даже в~ХХ~веке. Уничтожение лесов истощает
кислородный ресурс. Отравление атмосферы в (пост)индустриальную эпоху приводит к
\flqq потеплению\frqq, становящемуся необратимым.


Ресурсы, даже возобновляемые, остались источником проблем. Эффективность энергозатрат (отношение затраты:результат, в
джоулях):


\bigskip

\begin{center}
\tablefirsthead{}
\tablehead{}
\tabletail{}
\tablelasttail{}
\begin{supertabular}{cccc}
 Собирательство  &  1:100 &  Интенсивное земледелие &  1:2 \\
 Экстенсивное земледелие &  1:70 &  Индустриальное земледелие &  1:0,001 \\
\end{supertabular}
\end{center}

\bigskip


\textbf{\textit{5. Нас — миллиарды, это самое большое наше «достижение».}}


Соотношение полов М:Ж (при биологической норме 105:100):


\bigskip

\begin{center}
\tablefirsthead{}
\tablehead{}
\tabletail{}
\tablelasttail{}
\begin{supertabular}{cccc}
 Франция, 800 г. н.э. &  136:100 &  Флоренция, 1330 г. н.э. &  118:100 \\
 Англия, 1391 г. н.э. &  170:100 & Милет, V век до н.э. &  421:100 \\
\end{supertabular}
\end{center}
{\itshape
    Именно в Милете родилась наука. В Дельфах только в 1\% семей было
\textbf{две} дочери.}


Традиция изоляции молодых мужчин (в монастырях и в армиях) прекрасно сохранилась.

\subsubsection[Решение проблем свободного времени и лишних людей]{Решение проблем свободного времени и лишних
людей}

\textbf{В Каменном веке} свободного времени не было: отдых, ритуальные церемонии, пляски, песни, рассказы прерывались
едой и «работой». Катастрофическое (примерно 10-кратное) падение нашей численности в верхнем палеолите вычистило
значительную часть лишних людей, но только на период мезолита.


\textbf{В Неолите} установившиеся вместо охоты скотоводство и земледелие увеличили несущую способность земли и
восстановили благоприятные условия для самовоспроизводства лишних людей. Хотя средняя продолжительность рабочего дня
возросла с двух часов до трех с половиной, бездельников становилось все больше. Действенным средством заполнения
времени, остававшегося свободным, стали и навсегда остались религия, войны и переселения.


\textbf{В Античности} для борьбы с растущим свободным временем придумали спорт, театр, бои гладиаторов и состязания
ораторов.


\textbf{В Средние века} церковь стала вместилищем лишних людей. Но даже чума и оспа не помогли, так что лишним пришлось
отправиться в крестовые походы. Детский крестовый поход мощно поддержал традицию инфантицида. Бесконечные церковные
праздники и откровенно языческие карнавалы так и не смогли поглотить свободное время.


\textbf{В Новое время} открытие Нового света и неслыханная экспансия европейцев поставили, как оказалось впоследствии,
\ процесс роста населения Земли в новые, еще более благоприятные условия. Невиданный подъем традиционной работорговли в
Африке не решил демографических проблем этого континента.


\textbf{В Новейшее время} революционные и другие войны «не поправили дело», хотя и радикально и многократно обновили и
улучшили технические и организационные средства уничтожения людей. Школы, даже став обязательными, церковь и
классическая христианская культура не смогли эффективно занять свободное время. Промышленная революция и принесенный ею
безудержный рост продолжительности рабочего дня не смогли предотвратить социальные взрывы в
\foreignlanguage{english}{XIX} веке.


\textbf{В ХХ веке} великие революции, две мировые войны и бесчисленные «мирные инциденты» не справились с квадратичным
ростом населения. Массовая культура и средства коммуникации поглощают значительную часть свободного времени, большую
надежду на успех подают компьютерные средства борьбы с ростом населения и со свободным временем уцелевших.


\textbf{Сегодня} лишние люди не «открывают Америку», а «закрывают» ее, повсеместно идут в атаку на тех, кто их кормит,
лечит и развлекает.

\subsubsection[Как оценивают наши достижения]{Как оценивают наши достижения}

Организация Объединенных Наций оценивает достижения национальной экономики по следующим критериям:

%\liststyleWWviiiNumiii
\begin{enumerate}
\item Состояние среды обитания (в т.ч. здравоохранение, состояние санитарии).
\item Процент детей в возрасте до 3 лет с весом ниже национальной нормы.
\item Роды без медицинской помощи (процент).
\item Процент детей вне школы (по национальной норме).
\item Процент неграмотных женщин.
\end{enumerate}
%\end{document}
