% This file was converted to LaTeX by Writer2LaTeX ver. 1.4
% see http://writer2latex.sourceforge.net for more info
%\documentclass[a4paper]{article}

%\begin{document}
\section[Эволюция экономики до Нового времени]{Эволюция
экономики до Нового времени }
\subsection[Факторы эволюции экономики]{Факторы эволюции экономики}
\textbf{История экономики} ищет в механизме экономического развития причину гибели миллионов людей от бедности и
отслеживает:

%\liststyleWWviiiNumii
\begin{itemize}
\item рост / спад ВНП (ВВП);
\item развитие / деградацию структур и форм организации;
\item прогресс (этическая категория?).
\end{itemize}

\textbf{Экономическая теория} устанавливает показатели и решает проблемы их измерения и сравнения.

Установлены определяющие факторы: население, ресурсы, технологии, социальные институты. Факторы не являются
независимыми, их взаимодействие не детерминировано.

Инновации и контринновации в этих факторах определяют ход эволюции экономики.

Одна из важнейших комбинаций факторов — человеческий капитал\footnote{Рабы — человеческий капитал?}.
Этот (вторичный) фактор оказывает на экономику как положительное, так и отрицательное воздействие.

\subsection[Экономика каменного века]{Экономика каменного века}
\subsubsection[\flqq Тяжкий труд\frqq и \flqq беспросветная нужда\frqq]{\flqq Тяжкий труд\frqq и \flqq беспросветная нужда\frqq}
Охотники и собиратели в симбиозе с природой \textit{жили и не тужили}: они были обеспечены (по нормам белого англосакса)
калориями — на 105\%, белками — на 400\%.

В языке некоторых аборигенов Австралии «играть» и «работать» — одно и то же слово.

Даже сегодня один бушмен может обеспечить жизнь 4-5 человек, работая полтора-два дня в неделю по 6~часов. Две трети
бушменов работают, затрачивая на работу треть этого времени, остальные вообще не работают (дети, старики и
\textit{бездельники}).

Палеолитический рабочий режим: 1-2 дня работы (не более 4 часов) и столько же дней отдыха. Жизнь не была монотонной,
было много праздников (вместе с соседями) и «каникул».

Запасы привязывают человека к месту их хранения и препятствуют своевременному переходу на неистощенную кормовую
территорию. Общественные запасы стимулируют размножение \textit{бездельников,} индивидуальные запасы подрывают
нравственность.

Имущество — враг благополучия, особенно недвижимое.

Производство для потребления, а не для обмена, не становится избыточным: ограничивает себя отрицательной обратной
связью. Чем больше относительные трудовые способности семьи в некоторой общине, тем меньше ее члены работают (правило
Чаянова). Поэтому община устойчива. Конфликтная перенаселенная община распадается на устойчивые фрагменты: человек
завоевывает Землю.

Первенцев и лишних детей убивали, особенно девочек (будущих матерей). На время кормления ребенка женщина табуировалась,
а ведь время кормления растягивалось до трех лет. Пленная женщина не табуируется, поэтому рожает и пестует детей без
ограничений, а в голодное время мужчина-владелец съедает детей.

Брачный возраст для мужчин повышался с ростом численности населения и дошел (уже в неолите) до 30
лет\footnote{При средней продолжительности жизни не более 20 лет.}. Для молодых мужчин труд
необязателен, у них «семеро по лавкам» не сидят, поэтому их не принимают в работающую группу. Они идут в услужение к
вождю, режут друг друга в стычках с (такими же) соседями и уменьшают демопрессинг, а при удаче — увеличивают кормовую
территорию.

Симбиоз с животными изменился. Когда охотничий ресурс под воздействием похолодания и истребления стал катастрофически
убывать, мы стали даже охранять — одомашнивать — некоторых животных, убивая их только по мере надобности, охраняя их от
других хищников. Тем самым изменился и симбиоз с растительным миром. Уничтожение растительности заставляло  перегонять стада с одного пастбища на другое. Мы так и не поняли, что гибель растительности приводит к уменьшению количества осадков, в результате растительности становится еще меньше — образуется петля положительной обратной связи с отрицательными последствиями (петля на шее)\footnote{Козы очень удобны. Правда, они острыми копытцами подсекают траву — и нас...}. Примерно 10 тысяч лет тому назад мы оказались на грани гибели. Остатки многочисленных племен, опустошивших пастбища, \ уцелели в поймах рек. Не стало пастбищ, не стало скота. Пришлось расширить старые огороды и тем самым заняться \textit{поливным }земледелием. Единственным ресурсом стали люди, особенно \textit{чужие}. Скученность и «острый дефицит ресурсов» оставили шанс только каннибалам, уничтожавшим сначала всех, а потом преимущественно соседей. Потери численности возмещали рабы-воины и рабы-добровольцы (= иммигранты). Завоевание территорий и ирригация поглощали людской ресурс.

В многонациональной массе рабов сохранялась прежняя межплеменная вражда. Для управления этой массой возникли
иерархические структуры (государственные). Чем ниже уровень взаимопонимания (и коммуникабельность), тем более сильным должно быть государство.\footnote{И наоборот!} Непременными условиями стали разобщение НИЗА и сплочение ВЕРХА.

Неолитическая революция\footnote{Никакой революции не было!} 
привела к увеличению
продолжительности рабочего дня почти до 3,5 часов.\footnote{Результаты промышленной революции и достижения социализма лучше не поминать.} 
Затраты труда росли, территориальный ресурс убывал. Укрепление
государства отражено в \textit{монотеизме}.

В непрерывных войнах государственные армии терпели поражения в столкновениях с негосударственными кочевыми племенами. Из
нецивилизованного окружения энергичные люди проникали в армию (= силовую структуру), а оттуда - в политическую элиту.

Кто построил Иерихон? – Его враги.

Разделение труда и ограничения на брачных партнеров породили условия для развития рынка, а затем появились города,
государство, письменность. Ограничения на долговое рабство ослабили, а ограничений на рабство пленных вообще не стало.
Человек стал товаром, а не только добычей. 

\subsubsection[Кто работает? А кто ест?]{Кто работает? А кто ест?}

Мужчины-охотники современного охотничьего племени проводят сухой сезон за игрой в карты: проигравший свои наконечники к
стрелам освобождает себя от охоты. Современный араб на Ближнем Востоке пасет верблюдов, пьет кофе и курит, а ставят
шатры, ухаживают за овцами и козами, приносят воду — женщины.

Только три четверти домохозяйств справляются с задачей самообеспечения. Более того, некоторые семьи вырабатывают средств
потребления существенно больше общинного минимума. Куда девать добро? — Раздавать или уничтожать.

Щедрость в раздаче избытков повышает социальный статус одаривающего и порождает дополнительные моральные обязательства
одариваемого. Щедрый — вождь, пока он щедр. Вождь в начале процесса не богаче, а щедрее. Впоследствии народ одаривает
вождя, возмещая дары по цепи положительной обратной связи и обогащая казну вождя для еще более щедрых раздач.
Укрепление политической власти стимулирует производство, политика становится предпринимательством (и наоборот).
Социальные взрывы уничтожают недостаточно щедрых вождей.

Норма: две трети общинников вырабатывают избыточную продукцию — треть всей продукции. Бездельники — пять процентов
популяции — никогда ничего не производят.

\subsubsection[Рынок]{Рынок}
\textbf{Кула }— обмен предметами, не имеющими — в наших глазах — никакой ценности. Обмен между жителями многочисленных
островов Океании шел в двух ``противоположных" направлениях. В одном направлении передавали
браслеты из раковин, в другом — ожерелья из раковин. Обмены были разновременными и безвозмездными. Ожерелья и браслеты
не задерживались у \textit{временных} владельцев и продолжали движение. Давать
\foreignlanguage{english}{{\textgreater}} владеть.

Рыночный обмен был натуральным (бартерным). Межплеменной обмен мог быть «немым» (без контакта сторон, заочным), и мирным
(оружие имели при себе, но обычно не применяли). Иной раз в обеспечение обмена брали (давали) заложников. Торг мог
перейти в сражение. Трофеи, добытые на «военном» рынке, порождали спрос и расширяли обмен на «мирном» рынке. Поначалу
«выход на рынок» возглавляли вожди, но отлучка часто была небезопасна, поэтому их заменили «торгпреды» (будущие купцы).
Излишки, не участвовавшие в обмене, уничтожали.

Некоторые племена специализировались на весьма прибыльных посреднических операциях, не информируя своих клиентов о
происхождении товаров.

Описанные обычаи дожили до ХХ века\footnote{И даже до \foreignlanguage{english}{XXI}
века!}. 

\subsubsection[Эффективность ]{Эффективность}
Территория одного и того же размера способна прокормить в 20 раз больше скотоводов, чем охотников. Несущая способность
той же территории увеличивается еще в 30 раз при переходе к земледелию (но — скрытое голодание: нехватка животных
белков).

В каждой фазе мы истощаем очередной ресурс природы — на всем протяжении нашей истории.

\subsection[Первые цивилизации]{Первые цивилизации}
Земледелие (мотыжное!) стало основой экономики. Металлы — самородная медь, затем золото и серебро — поначалу
использовали только для изготовления украшений.

\textbf{\textit{В Шумере}} государственное рабовладельческое земледелие породило письменность как средство учета.
Аппаратчики составили 1/5 населения. Одомашненные кошки стерегли зернохранилища. Города защищали население от набегов.
Развилась морская торговля.

\textbf{\textit{В Египте}} предки египтян, вытоптав Сахару и потеряв скот, сгрудились на берегах Нила,. На новом месте
ирригация давала снимать два урожая в год. Урожайность (сам-):

\begin{center}
\tablefirsthead{}
\tablehead{}
\tabletail{}
\tablelasttail{}
\begin{supertabular}{cccc}
Древний Египет & 20 & Россия (сегодня) &  7\\
 Европа (500 лет тому назад) &  5 & США (сегодня) & 15\\
\end{supertabular}
\end{center}

Бюрократия Египта придумала ежедневный бухгалтерский учет расхода и прихода, беря на заметку каждую головку чеснока и
каждую меру зерна. Строительство пирамид поглощало свободное время и давало власти ``дело".

\textbf{\textit{В Греции}} микенцы создали к \foreignlanguage{english}{XV}{}-\foreignlanguage{english}{XIII} векам до
н.э. великую цивилизацию. Фортификация, письменность (правда, заимствованная), керамика, железное оружие, ювелирное
искусство, торговые связи с Египтом и Средним Востоком — краткий перечень их достижений. В «темные века»
(\foreignlanguage{english}{XI}{}-\foreignlanguage{english}{VIII} века до н.э.) произошел ужасающий регресс. Население
ушло, оставшимся прежние достижения не были нужны. Депопуляция вернула некоторых греков к кочевому скотоводству.

\textbf{\textit{На Востоке}}: города в Индии имели отличные инфраструктуры, в них жило до 100~тысяч человек. Эта
цивилизация существовала до 1500 года до н.э. Рост населения привел к ресурсному кризису. Население стало уходить из
ранее процветавших городов, сделав их ненужными.

\textbf{Азиатский способ производства} основан на единстве власти и собственности, обусловленном системой земледелия,
для него характерны автаркия и экспансия. Тотальный контроль. Уничтожение противников, затем сторонников, учреждение
культа Великого Каннибалиссимуса. Стагнация и крах.

\subsection[Античность]{Античность}
\subsubsection[Греция ]{Греция }
Большая несущая способность земледельческих долин стабилизировала их локальную экономику, малая несущая способность
прибрежных районов обусловила быстрый рост морского разбоя и колонизации чужих побережий и последующей колонизации уже
из колоний. Расселение греков не привело Грецию к стагнации микенского типа. Развитие торговли и освоение железных
орудий обусловило развитие экономики: Греция экспортировала продукцию высоких технологий — вино, оливковое масло,
шерсть, медь, предметы роскоши и быта, импортировала рабов и продукцию экстенсивных технологий — зерно, лес. Несущая
способность материковой Греции увеличилась раз в 5. Чеканка денег привела к интенсификации экономики, выделилась новая
элита, умевшая считать деньги и товары, писать и читать. Греческие суда имели до пяти палуб и экипаж до 1000 человек.
Греческие города возникли на островах и по берегам не только Средиземного и Эгейского морей, но и Черного моря.

Рост численности населения Великой Греции привел к росту социальной напряженности. Победа тоталитарной Спарты в
гражданских войнах стала победой деревни над городом, демографический коллапс Спарты охватил Грецию. Рост наемных армий
не смог предотвратить победу Македонии. Александр Македонский успешно провел компанию по колонизации греками необъятных
территорий и привел Грецию к необратимому упадку.

Греция и греки стали легкой добычей Рима, освоившего достижения греческой экономики.

\subsubsection[Рим ]{Рим }
Царский Рим, республиканский Рим и императорский Рим до предела использовали несущую способность родного полуострова и после внутренних войн приступили к успешной экспансии. На завоеванных землях патриции разводили скот: это было
выгоднее хлебопашества. Дешевый рабский труд (около 10 миллионов!) разорил коренное население завоевателей, крестьяне
стали пролетариями\footnote{Найдите в \textit{старом} словаре слово
\foreignlanguage{english}{proletarius}.} и хлынули в города и в армию. Землю поглотили латифундии. Армия из
крестьянской стала наемной.

Долгое время воины не получали землю, армия сохраняла боеспособность. Римляне властью Сената выводили избыточное
население в колонии\footnote{Как свое, так и союзников.}. Предоставление побежденным римского
гражданства успешно служило направлению социальной напряженности \textit{вовне}, на расширение территории и на
колонизацию завоеванных земель. Цена на землю стала падать. Ветераны стали получать землю: 120~000 от Цезаря, 300~000
от Августа (в основном вне Италии). Военачальники «повысили рейтинг», империя перешла от завоеваний к обороне. Ветераны
захотели получать не землю, а деньги.


Колонизация и политика «хлеба и зрелищ» привели к депопуляции окрестностей Рима, к росту латифундий и росту численности
рабов. Сельское хозяйство из интенсивного становилось экстенсивным. Кризис усилила безудержная чеканка монет. Удержал
армию от превращения в активную внутреннюю силу только натиск варваров. Наемную армию пришлось подкрепить рекрутами
(!).


Экономика натурализовалась, налоги стали брать натурой, денежное обращение и торговля регрессировали. Латифундисты
(будущие феодалы) стали наделять крестьян землей, обращая их в колонов — экономических рабов (будущих крепостных).


Империю пришлось разделить. Западная часть окончательно деградировала, оставив нам христианство, учившее возмещать
экономическую нищету духовным (=религиозным) богатством. Восточная (грекоязычная) часть прожила еще только 1000 лет.


«Священная Римская Империя Германского Народа» дожила до 1812 года.

\subsubsection[Гибель античного мира]{Гибель античного мира}

Экономическая деградация привела к необратимому ослаблению армии. Другие следствия — деградация
нравственности\footnote{Одно из следствий — «победа» христианства.} и утрата идентичности.


По мнению Л.Н.Гумилева этнос живет не более 1500 лет. Рим существовал как раз 1229 лет. Римские пассионарии устремили
энергию на «разборки», а субпассионарии пришли к власти и создали «антисистему», направленную на безоглядное достижение
эфемерных целей «ценой» дешевых жизней своих же граждан.


Античный мир оставил нам вооружение, римское право, греческую философию и науку, античное искусство, суровые уроки
рыночной экономики и основы феодализма.

\subsection[Средние века в Европе]{Средние века в Европе}

В «темные века» (\foreignlanguage{english}{VI}{}-\foreignlanguage{english}{VIII} века) утрачены ветряные и водяные
мельницы, многие ремесла и технологии. Бандиты брали крестьян «под крышу» построенных руками крестьян замков и
кормились их трудом. В \foreignlanguage{english}{VIII} веке изобрели стремя — (тяжело вооруженным) воевать стало легче.


Крепостная зависимость стала «добровольной». Ремесленники нашли убежище в монастырях, там же укрылись врачи и просто
грамотные люди. Отдавшиеся церкви крестьяне довели церковные владения до половины всего земельного фонда. Перед 1000
годом церковь посеяла панику, предсказав наступление конца света и Страшного суда \ и обещав спасение тем, кто успеет
пожертвовать имущество церкви\footnote{Опомнившиеся простаки пытались потом отсудить свои дары.}.


Численность населения упала с 70 млн. до 30 млн. человек.


Земля могла быть независимым наследственным владением, пожизненным владением военнослужащего, наследственным владением
под присягой сюзерену (ленник обязан воевать до 40 дней в году, сюзерен обязан выкупать его из плена). Основным
экономическим отношением стала рента: барщина, оброк и денежная. Фиксированная рента устанавливалась по душам (мужским)
или по размеру земельного участка. Прогрессивная рента (доля дохода) могла быть натуральной или денежной. Условия могли
быть пожизненными или даже наследственными. Расширение рынка сбыта обусловило прикрепление крестьян к земле.


С начала \foreignlanguage{english}{XI} века начался экономический подъем, возрождение ремесел и городов (площадью от 1,5
до 3 га). Основой подъема стало укрепление сословного деления: \textit{внутри} \textit{сословия} человек был
\textit{свободен} и \textit{защищен}. Доля свободного городского
населения\footnote{\foreignlanguage{english}{\textsf{\ }}\foreignlanguage{english}{«Der Stadt macht frei»}} достигает
25\%. В Париже стало 60 тыс. жителей, во Флоренции — 100 тыс. Возобновилась чеканка монет, в Венеции состоялся выпуск
первых ценных бумаг. Местная торговля была вялой, дальняя торговля быстро
росла.\footnote{\foreignlanguage{english}{\ }\textcyrillic{См}\foreignlanguage{english}{.
«}5\foreignlanguage{english}{. }\textcyrillic{Пример экономического учения (меркантилизм)»}} Прекратились походы
викингов.


\textit{Майорат} привел к избытку дворян (рыцарей). Избыточное мужское население частично поглотили крестовые походы.


Ремесленники объединились в цеха для защиты своих прав и заодно для жесткой — в духе времени — регламентации
деятельности от технологии и рабочего времени до изделий и цен. В дальнейшем новаторы противопоставили цехам
мануфактуры.


Торговцы в тех же целях объединились в гильдии.


Сословная структура была очень жесткой. Крестьяне составляли более 90\%, средний класс — примерно 7\%, высший класс — не
более 2\%, так что сто крестьян кормили десяток ремесленников (торговцев, грамотеев и др.) и двух отпетых бездельников
из политической элиты. Браки были «эндосословными».

Монастыри формировали инфраструктуру и были центрами образования и паломничества (= туризма).

Мылись часто и тщательно только находившиеся (поэтому) под подозрением мусульмане и евреи.

Средневековое экономическое «возрождение» прервали в \foreignlanguage{english}{XIV} веке Столетняя и другие войны, имевшие экономические причины. Чума унесла около трети населения, особенно пострадали города. К середине века численность населения упала вдвое. На попытки феодалов повысить ренту крестьяне ответили массовыми восстаниями. Голод возникал еще и потому, что после неурожая съедали семенное зерно. Немалую роль сыграло временное похолодание: в Англии исчез виноград, в Норвегии — хлеб.

Изобретение артиллерии потребовало массовой наемной армии — и пришлось приступить к «освобождению» крестьян (за плату, шедшую на вооружение рыцарей).

Регресс привел к экстенсификации земледелия. Города обезлюдели, спрос и цены на продукты питания упали. В Западной Европе это привело к росту животноводства, а в Восточной Европе «возродилось» натуральное хозяйство. Крестьяне ударились в бега, феодалы закрепощали оставшихся, арендная плата достигла половины урожая.

В \foreignlanguage{english}{XV} веке численность населения возросла, возросли объемы производства и торговли. Турки,
завоевавшие Малую Азию, запретили «неверным» торговый доступ на Ближний и Средний Восток, перекрыли товарный обмен
между Европой и Востоком. Турецкая бесцельная блокада и перенаселение в наиболее развитых регионах Европы породили
эпоху Великих географических открытий.

\subsection[Заря капитализма]{Заря капитализма}

Феодалы создали государство, а оно пристроило их опасную энергию к делу: завоевывать земли не у соседей, а за океаном.
Особенный успех это начинание имело в Испании после Реконкисты. Ремесленники нашли себе дело, даже «праздные
размышления» интеллектуалов о шарообразности Земли пригодились в буре новой экспансии. Торговля стала \textit{мировой}
— с центром в Антверпене.


Золото, столь необходимое для торговли с Востоком, хлынуло в Европу из Нового Света. Первыми жертвами инфляции стали
Испания, Португалия и Англия (в ней за столетие цены повысились на 155\%, зато зарплата тоже повысилась — на 30\%).


Феодалы катастрофически проиграли: земельная рента осталась на прежнем уровне!


Инициатива в экономике перешла к предпринимателям — «буржуям». Именно они обеспечивали армии завоевателей колоний
оружием, боеприпасами, обмундированием. Падение численности населения сделало актуальными новые «человекосберегающие»
технологии. Водяные двигатели стали использовать не только для помола зерна, но и для дутья в печах для получения
ковкого металла — металл неожиданно расплавился и потек в виде чугуна. Стоило подуть на смесь чугуна и угля —
получилась сталь! Воду стали использовать для приведения в действие станков и молотов, насосов и подъемников в шахтах.


Цены теперь определял не цеховой староста, а покупательский спрос. Англия перестала вывозить шерсть и ввозить сукно, так
что в шерстяной промышленности, свободной от диктата цеховиков, работала половина англичан. Крестьян-земледельцев
сгоняли, огораживая пашни под пастбища для овец.


Европейский флот насчитывал около 20 тысяч кораблей, четыре пятых флота принадлежали Голландии (вот почему корабельному
делу учились в Голландии!), центр мировой торговли переместился в Амстердам (во Франции было 15 военных кораблей). В
Голландии из 600~000 мужчин — 150~000 моряков и 100~000 рыбаков. Протекционистский Навигационный акт привел к войне. В
четырех войнах с Англией — четыре поражения. Центр экономического развития переместился в Англию.


Англия стала супердержавой («Правь, Британия») в результате роста промышленности и военного роста на морях, став выше
Испании (\foreignlanguage{english}{XVI} в.), выше Голландии (\foreignlanguage{english}{XVII} в.), выше Франции
(\foreignlanguage{english}{XVIII} в.).

{\selectlanguage{english}\itshape
\foreignlanguage{russian}{Подписка для укрепления флота Ост-Индской компании за 3 дня собрала 2 млн. \ фунтов
стерлингов.}}


В Англии количество паровых машин возросло с 320 (1800 г.) до 15000 (1825 г.). Паровая машина сохранила рабство в
Америке. К концу \foreignlanguage{english}{XVIII} века паровая машина стала универсальным двигателем и стимулировала
спрос на металл и на новые технологии обработки металла. Паровую машину поставили не только на морское судно, но и на
рельсовые пути. Спрос на машины породил \textit{машиностроение}.


В обществе на место крестьянина стал рабочий, на место купца — фабрикант.

\subsection[Реформация ]{Реформация }

Распространение книгопечатания открыло дорогу к знаниям, особенно тем, кто сумел освоить самостоятельное чтение
книг\footnote{Гуттенберг — 1448 г. В России тогда же началась выгонка этилового спирта из зерна.}.
Немалую роль сыграло безбрачие духовенства, породившее школьную технологию горизонтальной передачи генов культуры.
Нашествие денег породило в обществе центры их концентрации и оборота. Одним из таких центров стала церковь, открыто
ставшая коммерческой структурой: за деньги продавали не только должности, но и отпущение грехов (даже будущих).
Нравственное и интеллектуальное разложение бессемейной католической церковной элиты способствовало успеху Реформации.


Ведущими факторами Реформации были:

%\liststyleWWviiiNumvi
\begin{itemize}
\item 
образование и укрепление национальных государств в противовес тенденции папства к общеевропейской и даже всемирной
власти;
\item церковь как землевладелец и как сборщик десятинного налога вызывала экономическую «неприязнь»;
\item 
унаследованное от античности стремление варваров идентифицировать себя с римлянами посредством (варварской) латыни.
\end{itemize}

Лютер перевел священные книги на немецкий язык и тем самым создал литературный немецкий язык. Священников стали
выбирать. Церковное имущество «пустили в дело».


Реформаторы провозгласили экономический успех свидетельством благоволения Бога. Верующего освободили от церкви - оплачиваемого посредника между ним и Богом.


Протестантские Швейцария, Германия (Северная) и Англия получили дополнительный стимул экономического развития, а Франция
после изгнания гугенотов (кальвинистов) его утратила. Австрия и Польша остались католическими, Скандинавия примкнула к
антикатоликам, не говоря уже о Северной Америке.


Экономический прогресс и реформация церкви коррелируют, так как имеют один и тот же источник, а не связаны
причинно-следственной зависимостью.

\subsection[Просвещение ]{Просвещение}
\footnotetext{Полтораста лет между английской и французской революциями.}

Популярность этой эпохи в среде ученых обязана тому авторитету, который получила наука в среде аристократов — в
политической элите Франции.


Ведущим фактором в эволюции экономики стал картофель: за столетие число голодных лет упало вдвое. Население Европы за
время Просвещения возросло со 100 до 175 миллионов человек (до Урала — со 130 до 225), население Северной Америки — с
14 до 25 миллионов.


Денежная реформа в Англии привела к тезаврации и утечке полноценных монет. Появились не только бумажные деньги, но и
ценные бумаги основанного в 1694 году Английского банка (частного!).


Полным ходом идет раздел мира между лидерами экономического развития, в число которых вошла и
Россия.\footnote{Англия захватила $1/4$, Россия — \textbf{1/8} часть суши.}

Во Франции дворяне-предприниматели лишались налоговых привилегий. Узкий внутренний рынок ограничил ремесленников
рынком оружия и предметов роскоши. Абсолютизм кормился растущими налогами (и откупами), разорял страну. Кольбер навел
было порядок, но аферист Джон Лоу эмиссией кредитных билетов довел Банк Франции до краха. Тюрго и Неккер не справились
с монархом и аристократами. Взрыв стал неизбежен, аристократы уплатили жизнями (и не только своими) за глупость и
упрямство (не только свои).

\subsection{Фазы эволюции}

\textbf{В архаической фазе }охота обеспечила жизнь и развитие социума. Мы научились пользоваться огнем и выжгли среду
обитания, вооружились и истребили крупную живность. Наступил экологический кризис. Медленный рост численности сменился
катастрофическим падением.


Мы перешли \textbf{в новую фазу}, позже ее назвали \textbf{традиционной}. Бездумное истребление диких животных мы
заменили унылой ``охотой" на прирученных — домашних — животных. Случайный посев злаков
превратился в интенсивное земледелие — поливное и подсечное. Мы приступили к вытаптыванию пастбищ и засолению пашен
средствами ирригации. Нас стало больше, человек приступил к заселению Земли и ``пустил в
расход" практически всю живую и неживую природу. Основой экономики стало \textbf{зерно}, появились
торговля, деньги, умение считать, писать и читать создало новую элиту.


Дрова пока остались, но мы добрались до угля. Если ранее мы ``на предельной скорости" проходили немногие десятки километров за сутки, то теперь на лошади или под парусом мы одолевали полтораста километров.
Грабеж открыл новые горизонты!


Эволюция экономики ускорилась. Устаревшие социальные конструкции рухнули\footnote{``Цинизм
принципов, мистифицируя авторитеты, парализует ресурсы социальной конструкции".}. Мартин Лютер открыл
нам \textbf{индустриальную фазу}. Машина стала править миром! Основой экономики стали энергоносители, система обмена и
валюта стали мировыми. Экономика стала кредитной. Мы ``пустили в расход" всю биосферу и весь
земной шар. Несмотря на резкое падение рождаемости в ходе каждой ``ррреволюции", нас
становится все больше.


Пора выйти в Космос и навести в нем \textbf{\textit{наш порядок}}!
%\end{document}
