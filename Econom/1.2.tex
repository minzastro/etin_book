% This file was converted to LaTeX by Writer2LaTeX ver. 1.4
% see http://writer2latex.sourceforge.net for more info
\documentclass[a4paper]{article}
\usepackage[utf8]{inputenc}
\usepackage[T2A,T1]{fontenc}
\usepackage[russian,english]{babel}
\usepackage{amsmath}
\usepackage{amssymb,amsfonts,textcomp}
\usepackage{color}
\usepackage{array}
\usepackage{supertabular}
\usepackage{hhline}
\usepackage{hyperref}
\hypersetup{pdftex, colorlinks=true, linkcolor=blue, citecolor=blue, filecolor=blue, urlcolor=blue, pdftitle=Marina}
% footnotes configuration
\makeatletter
\renewcommand\thefootnote{\arabic{footnote}}
\makeatother
% Outline numbering
\setcounter{secnumdepth}{0}
\makeatletter
\newcommand\arraybslash{\let\\\@arraycr}
\makeatother
% List styles
\newcommand\liststyleWWviiiNumvii{%
\renewcommand\labelitemi{$-$}
\renewcommand\labelitemii{o}
\renewcommand\labelitemiii{${\blacksquare}$}
\renewcommand\labelitemiv{{\textbullet}}
}
% Page layout (geometry)
\setlength\voffset{-1in}
\setlength\hoffset{-1in}
\setlength\topmargin{0.3154in}
\setlength\oddsidemargin{0.7874in}
\setlength\textheight{10.037699in}
\setlength\textwidth{6.8897996in}
\setlength\footskip{0.5122in}
\setlength\headheight{0.2756in}
\setlength\headsep{0.2366in}
% Footnote rule
\setlength{\skip\footins}{0.0469in}
\renewcommand\footnoterule{\vspace*{-0.0071in}\setlength\leftskip{0pt}\setlength\rightskip{0pt plus 1fil}\noindent\textcolor{black}{\rule{0.25\columnwidth}{0.0071in}}\vspace*{0.0398in}}
% Pages styles
\makeatletter
\newcommand\ps@Standard{
  \renewcommand\@oddhead{\bfseries \textcyrillic{Эволюция экономики и экономических теорий}}
  \renewcommand\@evenhead{\bfseries \textcyrillic{Эволюция экономики и экономических теорий}}
  \renewcommand\@oddfoot{\textstylePageNumber{\textsf{\thepage{}}}}
  \renewcommand\@evenfoot{\textstylePageNumber{\textsf{\thepage{}}}}
  \renewcommand\thepage{\arabic{page}}
}
\makeatother
\pagestyle{Standard}
\setlength\tabcolsep{1mm}
\renewcommand\arraystretch{1.3}
\title{Marina}
\begin{document}
\clearpage\setcounter{page}{1}\pagestyle{Standard}
\section[1.2 Эволюция экономики в Новое время]{\rmfamily 1.2 Эволюция экономики в Новое время}
{\selectlanguage{russian}
Данные приведены в основном по Англии как лидеру развития европейской экономики.}

\subsection[Народное хозяйство]{\rmfamily Народное хозяйство}
\subsubsection[Инфраструктура]{\rmfamily\bfseries Инфраструктура}
{\selectlanguage{russian}
Островное положение Англии и особенности ландшафта позволили заменить перевозку грузов по суше перевозкой на судах и
баржах в прибрежных водах и по возрастающей сети каналов. Парламентскими актами на дорожные
тресты\footnote{\ \textcyrillic{Трест = trust = доверие.}} была возложена ответственность за прокладку дорог с твердым
покрытием, что сокращало транспортные расходы в три раза. Протяженность дорог увеличилась с 3400 миль в 1750 г. до
22000 миль в 1836 г.}

{\selectlanguage{russian}
Грунтовые дороги создали транспортную систему, развитую впоследствии железными дорогами. Железные дороги покрыли Англию,
а затем и Европу. Пароходы втрое сократили время на перевозку грузов.}

{\selectlanguage{russian}
Акционерные компании поначалу были запрещены, были разрешены частные предприятия с единоличной ответственностью
владельца по обязательствам фирмы и товарищества — с солидарной ответственностью всех членов по обязательствам фирмы.}

{\selectlanguage{russian}
Городская инфраструктура долгое время оставалась «деревенской»: в городах почти не было мощеных проездов, не было
водопровода и канализации, все отходы выбрасывали на «улицу». Антисанитария влекла высокую смертность, особенно
детскую.}

{\selectlanguage{russian}
Существенный вклад в развитие городов внес Джеймс Уатт изобретением парового (водяного) отопления.}

\subsubsection[Технологии ]{\rmfamily\bfseries Технологии }
{\selectlanguage{russian}
Новации нащупывали методом академика Тыка — методом проб и ошибок, корректно названным экспериментальным методом.
Новатор в (будущей) индустрии назывался \textit{инженер}\footnote{\ \foreignlanguage{english}{Ingenium}
(\textcyrillic{лат.) = природный ум, изобретательность, остроумная выдумка (а не должность!).}}. }

{\selectlanguage{russian}
Новые технологии встречали сопротивление цеховиков, боявшихся конкуренции, ремесленников, терявших работу, и властей,
опасавшихся роста безработицы. }

{\selectlanguage{russian}
Текстильное производство стало лидером инноваций. Изобретение машинных средств прядения и ткачества вынудило
сконцентрировать производство в фабричных помещениях: только в просторных зданиях удавалось разместить громоздкие
механизмы. Импорт хлопка для машинной обработки вырос за \foreignlanguage{english}{XVIII} век с 500 до 25000 тонн, к
1850 г.— до 500~000 тонн. Юг США справился с поставкой хлопка в результате сохранения рабства и изобретения машин для
очистки хлопка.}

{\selectlanguage{russian}
Второе место занимали строители производственных помещений. }

{\selectlanguage{russian}
Металлургию питали военные заказы, новации породило почти полное сведение лесов — добыча угля выросла в 15 раз за 150
лет. Горное дело особенно продвинулось в Германии, так что в Англии монопольные права давались тем, кто приглашал
немецких горных инженеров. Нашелся ученик Тыка, обработавший каменный уголь тем же способом, что и лес при изготовлении
древесного угля, — получился \textbf{кокс}. Изменилась выплавка металла, появилась техника проката. Отходы в виде газа
пошли на освещение улиц. Возрос спрос на уголь и на руду. Женщины и дети уже не справлялись с перетаскиванием добытого
из забоев на санях, их заменили пони, тащившие тележки по железным пластинам. После замены пластин железными брусьями
работу «поставили на рельсы». }

{\selectlanguage{russian}
Джеймс Уатт, чинивший одну из паровых машин Ньюкомена (для шахтного насоса), создал новую паровую машину, гораздо
меньшего размера, более производительную — вообще более совершенную. Сверлильный станок его соседа Уилкинсона,
делавшего пушечные стволы, пригодился для изготовления цилиндров. Дело пошло, хотя не удавалось получить кредит в 5000
фунтов («понятные» кредиты купцам легко давались миллионами фунтов). Уилкинсон применил машину Уатта для подачи воздуха
в доменную печь. Уатт получил патент.\footnote{\ \textcyrillic{И воспротивился установке паровой машины на локомотив,
так что паровоз появился только по окончании срока действия этого патента. Паровоз сразу пошел по сотням миль железных
дорог.}} КПД машины Уатта составлял 5\%; зубчатые передачи съедали 80\%.}

{\selectlanguage{russian}
Железные дороги породили спрос на металл и машины и увеличили спрос на транспортные услуги: добыча перевозимого угля
выросла за \foreignlanguage{english}{XIX} век почти в 15 раз.}

{\selectlanguage{russian}
Железнодорожная сеть быстро росла (в тысячах километров):}


\bigskip

\begin{center}
\tablefirsthead{}
\tablehead{}
\tabletail{}
\tablelasttail{}
\begin{supertabular}{|m{1.1538599in}|m{0.37685984in}|m{0.37685984in}|m{0.38375986in}|}
\hline
{\selectlanguage{russian}\bfseries Регион} &
{\selectlanguage{russian}\bfseries 1850} &
{\selectlanguage{russian}\bfseries 1875} &
{\selectlanguage{russian}\bfseries 1900}\\\hline
{\selectlanguage{russian} Великобритания} &
{\selectlanguage{russian} 2,5} &
{\selectlanguage{russian} 23} &
{\selectlanguage{russian} 30}\\\hline
{\selectlanguage{russian} США} &
{\selectlanguage{russian} 5} &
{\selectlanguage{russian} 85} &
{\selectlanguage{russian} 400}\\\hline
{\selectlanguage{russian} Россия} &
{\selectlanguage{russian} 0,03} &
{\selectlanguage{russian} 0,11} &
{\selectlanguage{russian} 62}\\\hline
{\selectlanguage{russian}\bfseries Весь мир} &
{\selectlanguage{russian}\bfseries 9} &
{\selectlanguage{russian}\bfseries 180} &
{\selectlanguage{russian}\bfseries 680}\\\hline
\end{supertabular}
\end{center}
{\selectlanguage{russian}
Импорт нового сырья из колоний породил новые отрасли: обработку сахарного тростника для получения рома и сахара,
обработку табака, выпуск фарфоровых изделий. }

{\selectlanguage{russian}
Книгопечатание: уже в начале Нового времени каталог книжной ярмарки содержал около 40 тысяч наименований. }

{\selectlanguage{russian}
В торговле деловая техника привела к развитию безналичных расчетов и породила систему кредитования. Деловая переписка
быстро стала системой передачи экономической и политической информации, обслуживавшей не только купцов, но и властные
структуры. Так появились первые службы связи и информационные агентства.}

{\selectlanguage{russian}
Разделение труда и кооперация не только повысили производительность труда, но и структурировали технологию, облегчив
академику Тыку поиск новаций.}

{\selectlanguage{russian}
В сельском хозяйстве внедрялись новые кормовые культуры. Огораживание земель позволило выполнять селекцию домашнего
скота и переходить к более совершенным системам севооборота.}

{\selectlanguage{russian}
Изобретения химиков использованы в обработке тканей: сначала удалось использовать для отбеливания серную кислоту, затем
— хлор и его соединения.}

{\selectlanguage{russian}
Научные теории открывали новые отрасли. Примеры: электроэнергетика и применения электроэнергии, оптика, органическая
химия.}

{\selectlanguage{russian}
Виды технологических новшеств:}

\begin{flushleft}
\tablefirsthead{}
\tablehead{}
\tabletail{}
\tablelasttail{}
\begin{supertabular}{|m{2.2677598in}|m{2.2677598in}|m{2.2747598in}|}
\hline
\centering{\selectlanguage{russian}\bfseries Изобретение} &
\centering{\selectlanguage{russian}\bfseries Инновация} &
\centering\arraybslash{\selectlanguage{russian}\bfseries Диффузия}\\\hline
\centering{\selectlanguage{russian} Патентуемое новшество} &
\centering{\selectlanguage{russian} Внедряемое новшество} &
\centering\arraybslash{\selectlanguage{russian} Распространение инноваций по отраслям и регионам}\\\hline
\end{supertabular}
\end{flushleft}
{\selectlanguage{russian}
Эпохальные инновации: навигация и кораблестроение (уступали по значению книгопечатанию).}

{\selectlanguage{russian}
Наиболее фундаментальные изменения остались незамеченными — изменения в интеллектуальной сфере (и здесь не обошлось без
Джеймса Уатта). }

{\selectlanguage{russian}
Наука училась у техники — первое начало термодинамики\footnote{\ \textcyrillic{Закон сохранения энергии (для
термодинамических систем).}} «выработано» паровой машиной.}

{\selectlanguage{russian}
Другие новации только упомянем: фотография, кинематограф, телеграф, радио, телефон, электрическая лампочка,
электрогенератор, электродвигатель, двигатель внутреннего сгорания, автомобиль, фонограф, пишущая машинка, красители,
взрывчатые вещества, пулемет, новые металлы, холодильник, консерванты.}

{\selectlanguage{russian}
Успехи лидеров обернулись поражением: Англия остановилась на угле, ее конкуренты обратились к электроэнергии.
Национальный доход Англии все возрастал, а перспективы роста исчезли. Центр экономического развития надолго
(навсегда?..) переместился за океан.}

\subsubsection[Государственный сектор]{\rmfamily\bfseries Государственный сектор}
{\selectlanguage{russian}
В Англии расходы государства в ВНП составляли до 10\%, в Германии и в США — не более 5\%. В бедных государствах доля
расходов государства в ВНП — до 30\% и выше. }

{\selectlanguage{russian}
Государственная поддержка образования в Англии была образцом отставания — в начале ХХ века едва ли каждый восьмой
ребенок школьного возраста посещал школу.}

\subsection[Природные ресурсы ]{\textrm{Природные ресурсы }}
{\selectlanguage{russian}
К рудам, соли и сере — основным ресурсам предшествующей эпохи — добавились уголь, гидроэнергия и
нефть\footnote{\ \textcyrillic{Бурение для добычи нефти изобрел Д.И.Менделеев.}}, минералы и новые руды. Нефть была в
руках России (благодаря предприимчивости Рокфеллеров и Нобилей) и США. Арабская нефть была в руках Англии. }

{\selectlanguage{russian}
Борьба за ресурсы становилась военной борьбой, торговаться стало некогда. Мировая война (между европейцами) стала
неотвратимой. США оставались в стороне: американцы воспретили европейцам использовать свои «канонерки» в Западном
полушарии. }

{\selectlanguage{russian}
Избыток ресурсов имелся только в России из-за хронического бедствия — нехватки денег.}

\subsection[Население ]{\textrm{Население }}
{\selectlanguage{russian}
Экологическая ниша домашнего труда резко сократилась. Женщины не бросили детей и не остались дома: на фабриках и даже в
шахтах работали женщины и дети. Мужчины поддерживали работу машин и надзирали за упомянутыми работниками, привычно
подчиняя их себе.}

{\selectlanguage{russian}
Производство мыла сказалось на здоровье и тем самым на численности населения.}

{\centering\selectlanguage{russian}
Численность населения (миллионов человек)
\par}

\begin{center}
\tablefirsthead{}
\tablehead{}
\tabletail{}
\tablelasttail{}
\begin{supertabular}{|m{1.1538599in}|m{0.37685984in}|m{0.37685984in}|m{0.38375986in}|}
\hline
{\selectlanguage{russian}\bfseries Регион} &
{\selectlanguage{russian}\bfseries 1800} &
{\selectlanguage{russian}\bfseries 1850} &
{\selectlanguage{russian}\bfseries 1900}\\\hline
{\selectlanguage{russian} Европа} &
{\selectlanguage{russian} 187} &
{\selectlanguage{russian} 266} &
{\selectlanguage{russian} 401}\\\hline
{\selectlanguage{russian} Великобритания} &
{\selectlanguage{russian} 11} &
{\selectlanguage{russian} 21} &
{\selectlanguage{russian} 37}\\\hline
{\selectlanguage{russian} Германия} &
{\selectlanguage{russian} 25} &
{\selectlanguage{russian} 36} &
{\selectlanguage{russian} 56}\\\hline
{\selectlanguage{russian} Россия} &
{\selectlanguage{russian} 37} &
{\selectlanguage{russian} 60} &
{\selectlanguage{russian} 111}\\\hline
{\selectlanguage{russian}\bfseries Весь мир} &
{\selectlanguage{russian}\bfseries 906} &
{\selectlanguage{russian}\bfseries 1171} &
{\selectlanguage{russian}\bfseries 1608}\\\hline
\end{supertabular}
\end{center}
{\selectlanguage{russian}
Еще через 50 лет в Европе жило 559 млн., в России — 193 млн., во всем мире — 2400 млн. человек. В дальнейшем численность
населения удваивается каждые 25 лет. За \foreignlanguage{english}{XIX} век население Европы выросло примерно в 2 раза,
а население Северной Америки — в 14 раз.}

{\selectlanguage{russian}
Миграция изменяет не только численность, но и качество населения. Инициативные и талантливые жители мигрировали: Россия
→ континентальная Европа → Великобритания → Северная Америка. Высылка неплатежеспособных должников и преступников из
Великобритании в колонии имела разные последствия для каждой из сторон… \ }

{\selectlanguage{russian}
Концентрация рабочей силы в городах довела численность городского населения к началу ХХ века до ¾ всего населения. }

{\selectlanguage{russian}
Новации закрывали одни экологические ниши и открывали другие. Только поспевай!..}

\subsection[Торговля]{\rmfamily Торговля}
{\selectlanguage{russian}
Хлебные законы Англии разрешали импорт хлеба только после повышения внутренней цены сверх установленного предела, тем
самым удорожали рабочую силу и вызывали «встречные» меры против экспорта английских товаров. Отмена протекционистских
препятствий (хлебных законов и Навигационного акта) положила начало политике свободной
торговли.\footnote{\ \textcyrillic{Рузвельт в ХХ веке обусловил американскую помощь Англии подписанием Атлантической
хартии о свободной торговле. Черчилль проиграл Америке в обмен на победу над Германией.}} Свобода торговли означала
свободу конкуренции на рынке, ставшем сначала международным, а потом и общемировым. Свободная конкуренция не была
объявлена, она возникла из следующих обстоятельств:}

\liststyleWWviiiNumvii
\begin{itemize}
\item {\selectlanguage{russian}
производители товаров не использовали торговых марок;}
\item {\selectlanguage{russian}
производители «равноправны», так как каждый производит очень мало товара;}
\item {\selectlanguage{russian}
предложение имеет высокую эластичность относительно спроса.}
\end{itemize}
{\selectlanguage{russian}
Свободу торговли зачастую обеспечивала свобода действий «канонерок».}

{\selectlanguage{russian}
К концу XIX века свободную конкуренцию вытеснила монополистическая конкуренция. Возникли картели (сговор независимых),
синдикаты (независимые производители с единой системой на входе и на выходе), тресты (юридически независимые фирмы с
объединенными финансами, производством и сбытом). Горизонтальные и вертикальные концерны (холдинги) — одно из самых
эффективных изобретений монополистов.}

{\selectlanguage{russian}\itshape
Сегодня Всемирная Торговая Организация обеспечивает свободу торговли и свободу конкуренции на общем рынке входящих в нее
экономических регионов, ограничивает протекционизм и демпинговый разбой, является ресурсом глобализации экономики.}

{\selectlanguage{russian}
Темпы роста торговли — в основном внутренней — в несколько раз превышали темпы роста населения.}

{\selectlanguage{russian}
Ориентация на экспорт тормозила развитие собственных обрабатывающих отраслей и поражала бедностью население, оставшееся
в стороне от доходного экспорта. Использование доходов от экспорта на импорт не сырья, а товаров усугубляло бедность и
экономическую зависимость.}

{\selectlanguage{russian}
Денежное обращение было совершенно недостаточным, поэтому образовались местные «банки», вводившие в местное обращение
расписки и жетоны как суррогаты денег.\footnote{\ \textcyrillic{В наше время эту функцию выполняют кредитные карты и
другие суррогаты.}}}

{\selectlanguage{russian}
«Неожиданное» превышение предложения над спросом впервые породило мировой кризис в 1825 году, затем в 1836, в 1847, в
1857, в 1866 годах. Следующий – обвальный – кризис начался в 1873 году и затянулся на несколько лет. Промышленность
восстановилась (темпы роста экономик-лидеров не падали ниже 3,5\% годовых в течение 30 лет), а свободная торговля
мелких независимых предпринимателей отжила свой век — наступило время монопольной конкуренции.
\foreignlanguage{english}{Standard} \foreignlanguage{english}{Oil} начал с 10\%, скупив ослабевших в ходе кризиса
«коллег», а через несколько лет овладел 90\% рынка нефтепродуктов. Государства предприняли попытки возврата к
протекционизму (иной раз утонченному) в ответ на протесты промышленников против губительной конкуренции.}

\subsection[Социальные институты ]{\rmfamily Социальные институты }
{\selectlanguage{russian}
Сформировалась новая экономическая элита — высший класс капиталистов (2,5\%). Интеллектуалы (лица свободных профессий)
имели невысокий статус и составляли около 5\%. Лавочники и служащие («средний класс») составляли около 10\%. Крестьяне
и ремесленники составляли около 20\%. Лица наемного труда (с низкой квалификацией — пролетарии) составляли примерно
60\%.\footnote{\ \textcyrillic{В России основная масса населения имела общинную структуру.}} }

{\selectlanguage{russian}
Описанная структура унаследовала результаты уничтожения среднего класса после гибели античного мира. Социальное
напряжение в социуме настоятельно требовало расширения стабилизирующего среднего класса. Соединенные Штаты и
европейские промышленные (и социальные!) лидеры успешно решили эту задачу и избегли гражданских войн и революций.
Отставшие страны закономерно попали в петлю деспотических режимов (Германия, Италия, Россия, Япония и др.). }

{\selectlanguage{russian}
Юридическая система Англии обеспечила защиту частных интересов от монархического и государственного аппетита. Во Франции
понадобился гениальный Кодекс Наполеона\footnote{\ \textcyrillic{В него входили Гражданский и Коммерческий кодексы.}},
чтобы обезоружить палеоконсерваторов и проложить дорогу в будущее для всей Европы. }

{\selectlanguage{russian}
Возникли новые структуры — организационные, возникли наука и практика научной организации труда и управления. В
промышленности появился конвейер (трудами всегда увольняемого Тейлора). Диффузия распространила конвейер на воспитание
детей, на среднюю и высшую школу, на общество в целом… Тем не менее, человеческий капитал стал возрастать, особенно в
регионах, где тоталитарные и конвейерные системы образования оказались «медленными» и не подчинили себе высшие
технические учебные заведения.}

{\selectlanguage{russian}
К концу Нового времени на образование акционерных обществ все еще требовалось специальное разрешение только в двух
империях — в Российской и в Османской.}

{\selectlanguage{russian}
Образовательные структуры отразили необходимость накопления человеческого капитала. Уровень грамотности населения
отражает уровень понимания государством роли неграмотности в экономике и военном потенциале.}

{\centering\selectlanguage{russian}
Уровень грамотности населения в 1900 году
\par}

\begin{center}
\tablefirsthead{}
\tablehead{}
\tabletail{}
\tablelasttail{}
\begin{supertabular}{|m{0.9698598in}|m{0.6705598in}|m{0.6330598in}|m{0.6775598in}|}
\hline
\centering{\selectlanguage{russian}\bfseries Страна } &
\centering{\selectlanguage{russian}\bfseries Процент } &
\centering{\selectlanguage{russian}\bfseries Страна} &
\centering\arraybslash{\selectlanguage{russian}\bfseries Процент}\\\hline
\centering{\selectlanguage{russian} Швеция } &
\centering{\selectlanguage{russian} 99} &
\centering{\selectlanguage{russian} Бельгия} &
\centering\arraybslash{\selectlanguage{russian} 81}\\\hline
\centering{\selectlanguage{russian} США (белые)} &
\centering{\selectlanguage{russian} 94} &
\centering{\selectlanguage{russian} Испания} &
\centering\arraybslash{\selectlanguage{russian} 44}\\\hline
\centering{\selectlanguage{russian} Пруссия } &
\centering{\selectlanguage{russian} 88} &
\centering{\selectlanguage{russian} Россия} &
\centering\arraybslash{\selectlanguage{russian} 28}\\\hline
\end{supertabular}
\end{center}

\bigskip

{\selectlanguage{russian}
Университеты долго оставались цитаделью чиновничества. }


\bigskip
\end{document}
