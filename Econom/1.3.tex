% This file was converted to LaTeX by Writer2LaTeX ver. 1.4
% see http://writer2latex.sourceforge.net for more info
%\documentclass[twoside,a4paper]{article}
%\usepackage[utf8]{inputenc}
%\usepackage[T2A]{fontenc}
%\usepackage[russian,english]{babel}
%\usepackage{amsmath}
%\usepackage{amssymb,amsfonts,textcomp}
%\usepackage{color}
%\usepackage{array}
%\usepackage{hhline}
%
%\begin{document}
\subsection[Пример эволюции сектора экономики]{Пример эволюции сектора экономики}
\subsubsection[Человек и земля]{Человек и земля}

Жизнь земледельца и скотовода на кормовой территории зависит от ее размера и качества и от трудозатрат. Судьба популяции
зависит от системы землепользования (а не от властителей и системы управления, не от религиозных обычаев и даже не от
образования).


Подсечно-огневая система очень хороша, пока есть что жечь и есть куда идти дальше; судьба этой системы прозрачна, как
сгоревший лес.


Земля может быть использована под пашню или под пастбище – для Крупного Рогатого Скота (КРС). Скот удобряет пастбище, а
против снижения урожайности постоянно используемой пашни мы ввели разные системы: многолетнюю залежь (перелог),
укороченную залежь (ротацию), короткую залежь (пар), сохраняя и ежегодную обработку со сменой культур. Главное же –
научились собирать навоз не только как топливо, но и как удобрение.


Увеличение пашни (за счет пастбища) снижает поступление навоза на пашню и ухудшает виды на \ урожай. Увеличение пастбища
повышает поступление навоза на пашню и улучшает виды на урожайность, но может привести к падению валового сбора
продукции с пашни.

Перед 1914 годом в Петербургской губернии было 104 головы КРС на 100 голов населения, и \ почти все эти головы кое-что
понимали.

В Простоквашино проницательным экономическим мыслителем и практиком был кот Матроскин.


Пашня и пастбище интегрируются в зависимости от климата, почв и качества населения:


в безморозной (Западной и Центральной) Европе интеграция была полной;


в Восточной Европе (с морозами) и в (благодатных) субтропиках интеграция была более низкой;


в (блаженных) тропиках интеграция попросту отсутствовала.


Динамика давления населения на почву и тем самым политические и экономические судьбы популяций (и экономических теорий)
соответственно различны.


\textbf{В тропиках} органические удобрения «были без надобности», поэтому система земледелия была ротационной с
удлинением циклов по мере падения урожайности. Давление населения на почву нечем было компенсировать. Тропические люди
не запасали для скотины корм на сухой период: полагались на природу и на шаманов (впоследствии на идеологов, на
правительство и на ООН).


\textbf{В субтропиках юга Европы} (Древний Рим) двупольная система кормила до 30 чел/км\textsuperscript{2}, но переход к
трехполью не состоялся из-за отсутствия достаточной площади пастбищ. Пришлось прибегнуть к частичному травополью
(подсев бобовых и кормовых). Тем временем поголовье КРС снижалось (субтропические люди тоже запасали корм для себя, а
не для скотины), \ снижалась урожайность и возрастало давление популяций на землю и друг на друга.


\textbf{В субтропиках Азии} (Месопотамия, Индия, Китай) наряду с переходом к грядковой культуре развивались
ирригационные системы (с засолением почв) на площади до ¼ пашни. Интенсивная система имела несущую способность до
150-200 чел/км\textsuperscript{2}. Эта система в Китае, Корее и Японии урезала пастбище до 10\% и даже до 1\%
культурной площади. Урожайность поддерживали трудозатраты, в том числе на компосты.


Выходит, что \textbf{в субтропиках} люди (пашня) вытесняют КРС (пастбище) → снижается количество навоза → растут
трудозатраты → снижается производительность труда → растут затраты на ирригацию → сокращается площадь на одного
работника → повышается трудовое давление на него. Аграрное перенаселение не находит выхода в урбанизации из-за
недостаточной производительности сельского хозяйства: она не достигает некоторой пороговой величины. В условиях, когда
вне с/х могло жить не более 15\% населения, рост населения не \ приводил к «индустриализации» и к увеличению
продовольственного потока. Не стало палеолитической проблемы избытков, остались одни нехватки (кроме населения).


\textbf{В умеренной Европе} двуполье с 6-летней ротацией и подсекой имело несущую способность не более 10-15
чел/км\textsuperscript{2}. Рудименты этой системы и культуры сохранились в бедных кельтских глубинках (Ирландия,
Шотландия, Уэльс, Бретань).\footnote{Не связаны ли с этим тамошние агрессивные / сепаратистские
настроения? Архаичность культуры?} Мягкая зима позволяет не заготавливать на зиму корм для КРС. С ростом
поголовья скота происходит переход: трехполье → травополье → плодосмена. Результат: рост с/х площади на 1~чел., рост
урожайности, рост населения с передачей избытка аграрного населения в город (пример с голландской армией после войны с
Испанией) и избытка продовольствия туда же с последующей компенсацией потоком индустриальной продукции, используемой
для повышения урожайности. Чем выше производительность труда (и чем ниже пороговый — для перекачки — темп роста
населения) в деревне, тем ниже пороговый темп роста промышленной продукции, обеспечивающий эффективную перекачку
населения. \textit{Вот где собака зарыта!}


\textbf{В Центральной Европе} в условиях недостатка КРС складывается система наступления пашни на пастбище с
положительной обратной связью, и дело становится совсем дрянь: наступают снижение урожайности и демографический коллапс
(Германия за XVII век потеряла 40\% населения). Чума в Европе улучшила соотношение поголовий людей скота, тем самым
высвободила скот. Выживший народ стал сеять овес и для лошадей, стал пахать на них: лошадь, которую ранее разводили для
военщины, нашла новое полезное применение. \ Дело пошло: хозяйства укрупнились, соотношение пашня:пастбище достигло в
Англии оптимальной (для перекачки населения) величины 3:1. Соотношение поголовий людей и КРС в Англии в условиях
травопольной системы к концу XVIII века достигло 1:1.


Урожайность (\foreignlanguage{english}{XVIII} век):


\ \ \ \ Англия \ \ 20 ц/га\ \ \ \ 50 ц/чел


\ \ \ \ Япония\ \ \ \ 16 ц/га\ \ \ \  \ 4 ц/чел


\textbf{Ситуация в Северной Европе} облегчена уходом избыточного мужского населения «на войну», так что дело не дошло до
демографического коллапса, произошел только демографический переход.


\textbf{В европейском }\textbf{\textit{аутфилде}} в условиях роста населения долгая ротация, несмотря на большое
количество скота, сохранялась вплоть до ХХ века.


\textbf{В Восточной Европе} развитие продолжилось не далее трехполья: не хватило навоза.


\textbf{А как же мы?} Русская зима ограничивает поголовье КРС и тем самым поголовье и качество людей. Миграция на
неосвоенные территории только частично решает проблему.

\subsubsection[Что же делать бедным людям на богатой земле?]{Что же делать бедным людям на богатой земле?}

Навести порядок, т.е. установить настоящую власть, построить деспотию.


Власть:


введет налоги на прокормление города (и себя);


установит цены \textit{на все}, установит рабочий день, определит перечень продукции;


национализирует землю и другой капитал (ликвидирует олигархов);


стабилизирует потребление (населением).


Налоги и цены станут расти, начнутся крестьянские и рабочие волнения. Если государство велико, оно распадется на части,
а процесс продолжится в каждой из частей: свертывание товарно-денежных отношений и развитие «справедливого» бартера —
\textit{гравитационный} коллапс экономики. Так в Риме произошел переход от республики к империи, в Китае — Циньская
консолидация, в России — Великая Октябрьская социалистическая революция.


Усиливается социальная мобильность, возникает жесткая вертикаль власти («обрубать ветки, укреплять ствол»), вводятся
ограничения на посредническую деятельность (торговую) и послабления для земледельцев (считается, что снижение продукции
сельского хозяйства порождено не аграрным перенаселением, а недостатком льгот). Наступает некоторая стабилизация: рост
населения замедляется вплоть до депопуляции, но снижение потребления продолжается. Снижение продовольственного потока
приводит к сокращению и ослаблению силовых структур, к возникновению неподавляемых междоусобиц и волнений, к разрушению
властной и экономической инфраструктур. Гравитационный коллапс перерастает в \textit{демографический} коллапс. Чем
продолжительнее период стабильности в рамках гравитационного коллапса, тем глубже наступающий демографический коллапс.



\textit{Спад высвобождает ресурсы}. Народ выдерживает и строит новую жизнь, радостно вступает в период
экономического роста (подобного предыдущему), неизбежно приводящего к новому демографическому коллапсу и к новому витку
цикла.

На эпоху кризиса приходится подъем в сферах искусства, религии и даже науки: туда направляются интеллектуальные ресурсы,
не нашедшие применения в производящих секторах.

\subsubsection[Судьбы народов]{Судьбы народов}

В тропиках \textit{жизнь проще}, гравитационный коллапс легче, зато демографический коллапс обретает катастрофический
характер (Микены, Месопотамия, Египет, майя, африканские государства).


В Европе демографический коллапс наступал, не дожидаясь гравитационного. Город не успевал выполнить \textit{историческую
функцию} \textit{индустриализации} \textit{сельского хозяйства}. Брошенные на произвол судьбы крестьяне сминают
пастбище, ища спасения в расширении пашни, лишают пашню навоза, а наступившее снижение урожайности не компенсируется
расширением пашни. Сил уже нет на гравитационный коллапс, так как отсутствует период стабилизации, и сразу наступает
демографический коллапс. Гравитационная деспотия так и не успевает появиться, некому \textit{навести порядок}.
Беспрепятственно развиваются промышленность и товарно-денежные отношения, на село поступает техника, способная
компенсировать нехватку трудовых ресурсов на прокормление всей популяции, происходит сбалансированная перекачка избытка
аграрного населения в город. Избыток населения в городе — это уже другой вопрос.


Устанавливается «эволюционная петля»:


Гравитационный коллапс → Гравитационная деспотия → Демографический коллапс → Распад и восстановление деспотии →
Демографический коллапс → и далее в светлое будущее.


Чтобы выбраться из этой петли, требуется обеспечить экономически и технологически согласованный одновременный рост
продукции сельского хозяйства и промышленности. Основой этого баланса является навозная куча — 10 тонн на каждом
гектаре пашни. Таких райских уголков оказалось совсем немного, почти только Англия и Голландия, в них горожане
составляли уже в середине \foreignlanguage{english}{XVII} века 40\% и 60\% соответственно. Скотине было вольготно, хотя
в результате отбора ее рост поубавился, а вес мозга уменьшился примерно на 20\%.


Скотине неплохо жилось и во влажной атлантической зоне (Ирландия, Шотландия, Уэльс, Бретань), но поставляемый ею навоз
не принес цивилизационных результатов: злаки были недостаточно урожайны, не поставляли необходимые эмбрионам
аминокислоты и витамины — и населяют эти места сепаратисты и забияки.

Японцы заимствовали западную технику и успешно сбалансировали структуру популяции: лишние люди
с земли переходили в город на готовые технологии, город снабжал деревню техническими средствами, не допуская снижения
выхода с/х продукции. Голландцы догадались импортировать английскую технику в том же XIX 
веке, что и японцы, и тоже преуспели.


Крупный Рогатый Скот уверенно привел удачливую часть Европы к капитализму другим путем:


Гравитационный коллапс → Абсолютная монархия\footnote{«Превращение всех налогов в денежные налоги — для
абсолютной монархии вопрос жизни» (Маркс)}. → Экспансия → Демографический коллапс → Республика (парламентская
монархия) → Президентская республика $\leftrightarrow $ Парламентская республика и далее в светлое будущее.

Проще жить в альтернативных цивилизациях Ближнего и Среднего Востока, сохраняющих устойчивые паттерны палеолитического
благоденствия. Хотите?..

\subsubsection[А как же мы? ]{А как же мы? }

У нас гравитационный коллапс успешно проведен в \foreignlanguage{english}{XVI} веке Иваном \foreignlanguage{english}{IV}
во время формального перехода к трехполью. КРС к Смутному времени\footnote{В 1601 году Волга стала в
августе. Этот санный путь и привел в Смутное время: мы съели семенное зерно.} народилось мало, поэтому
демографический коллапс состоялся «по полной программе». Поляки осилили трехполье, но травополье им оказалось не под
силу, поэтому они стали после серии демографических коллапсов легкой добычей добрых соседей.


Мы, конечно, попытались импортировать с Запада столь необходимую нам технику и \textit{все то}, что могло бы возместить
нехватку навоза. Результатом вестернизации не стала абсолютная монархия, как \textit{у них,} для этого у нас недостало
плотности населения. Бегство крестьян на свободные земли снижало демодавление и поддерживало производство продукции на
душу населения на сносном уровне. Трехполье в результате высвобождения скота (в ходе демографического коллапса) мы
все-таки ввели, плотность населения повысили, давление аграрного перенаселения облегчили, уже пора было перекачивать
лишних людей в город, но навоза было все-таки мало, а про химические удобрения мы еще не слыхали. Климат у нас такой,
что поехали мы к восстановлению гравитационной деспотии. Не спасло нас и поступление с Запада капиталов и технологий.


В большой экономической эпохе гравитационной деспотии от первых Романовых \ до воцарения Екатерины Великой мы
пережили подъем, затем пик, закончившийся \ Крымской войной, и настал спад. КРС было в 5 раз меньше, чем требуется для
поддержания производства продукции на селе (1,2 головы вместо 6 на 1 десятину пара), урожайность была ниже в 3~раза,
чем в Англии или в Германии. 300\nobreakdash-летний цикл закончился демографическим коллапсом Великого Октября (потеря
10\% населения и почти полное истребление элиты). Аграрно-перенаселенный Центр (будущий Красный пояс) решительно
поддержал нарождавшуюся деспотию, ее мероприятия в духе Цинь Ши-хуанди, а богатые районы Юга, Сибири и Дальнего
Востока, которые не были поражены аграрной люмпенизацией, поддерживали белых.

Закономерно возникла новая деспотия. Скот после перехода в собственность крестьян стал питаться зерном (не отдавать же
его большевикам), поэтому товарного зерна стало почти вдвое меньше, да и нагрузка на пашню увеличилась: народ двинулся
обратно на село. Российская экономика стала уникальной: экспорт зерна к 1927 году упал относительно 1913 года в 6 раз,
с 1928 года в городах пришлось ввести карточную систему. Межотраслевой перелив капитала практически прекратился, а об
иностранных инвестициях и импорте техники и думать было нечего. Изъятие продукции у крестьян немедленно привело к
сокращению посевных площадей. Деспотия приступила к коллективизации, т.е. к уничтожению аграрного перенаселения и
интенсификации крестьянского труда с целью перекачки рабочей силы в город. Село приступило к распашке пастбища,
снижение урожайности было частично перекрыто повышением производительности труда за счет простой кооперации и
разделения труда. Крестьянин на несколько поколений был переведен на подножный корм. Затем деспотия перешла к другим
формам \textit{массового} \textit{террора }как средства развития демографического коллапса. Экономика стала экономикой
плановых, т.е. придуманных цен, с плановыми \textit{натуральными} (а не экономическими) показателями. Госплан определял
(на арифмометрах даже в эпоху ЭВМ) места, количества и сроки производства \textbf{\textit{двух \ миллионов}}
изделий.\footnote{Какое еще нужно доказательство мощи советской экономической культуры!}


Аграрное перенаселение было в значительной мере перекачано в структурно дезорганизованную промышленность; бюджетные
ресурсы, питавшие эту политику, истощились, и мы оказались в экономическом тупике.\footnote{Тупик — это
место, из которого нет выхода даже назад.}

\subsubsection[Замечания о деспотии]{Замечания о деспотии}

Деспотия ограничивает свободу человека, чтобы сделать его более предсказуемым, а деспотию – стабильной.

Канатоходец переходит из одного нестабильного положения в следующее нестабильное за счет свободы в балансировании;
сужение свободы балансирования расширяет свободу падения.


Вложение средств в образование имеет следствием повышение способности человека к предвидению и тем самым делает
некоторых людей менее предсказуемыми, расширяет их возможности самостоятельного выхода из коллапса. Деспотия поэтому
вкладывает ресурсы в культуру только для достижения собственных целей.

В 1933 году рейхсканцлер Гитлер обратился к группе маститых
экономистов\footnote{Экономисты имеют масть.} с вопросом: «Сможет ли
Германия прокормить немецкий народ?» – Отчет заканчивался ответом «Нет». В том же году, ознакомившись с этим сразу же
засекреченным отчетом, Гитлер и его ближайшие приспешники приняли решение о военном натиске на Восток
(Drang nach Osten) для завоевания жизненного пространства (Lebensraum).

Составители отчета \textbf{не знали} о
химических удобрениях и сельскохозяйственных технологиях, уже изобретенных в Германии и впоследствии прокормивших не
только «германскую нацию». Невежество «ученых» и агрессивность «политиков» обошлись неисчислимыми потерями не только
«расе господ», но и всему миру — как всегда.
%\end{document}
