% This file was converted to LaTeX by Writer2LaTeX ver. 1.4
% see http://writer2latex.sourceforge.net for more info
%\documentclass[a4paper]{article}
%\usepackage[utf8]{inputenc}
%\usepackage[T2A]{fontenc}
%\usepackage[russian,english]{babel}
%\usepackage{amsmath}
%\usepackage{amssymb,amsfonts,textcomp}
%\usepackage{color}
%\usepackage{array}
%\usepackage{hhline}
%
%\begin{document}
\section[Циклы в экономике]{Циклы в экономике}

\bigskip

Теоретики исследуют циклические процессы в экономике в надежде получить средство для построения прогнозов
(предсказаний). Чижевский, ссылаясь на 11-летние циклы, удачно предсказал Великую депрессию на 1927-1929 годы. Открытие
Саймоном Кузнецом\footnote{За это время сменяется поколение людей и происходит обновление основных
средств.} циклов (20-25 лет) чередования быстрого и медленного развития научно-технического прогресса, численности
населения и национального дохода в США отмечено Нобелевской премией.

Из долгосрочных циклов наиболее известны циклы Кондратьева\footnote{В 1930 г. арестован, в 1932 г.
осужден на 8 лет (приговор отменен в 1987 г.), в 1938 г. расстрелян (приговор отменен в 1963 г.). Уцелевшие основные
труды опубликованы в 1991 г.} (Вологда, 1922 г.), определяемые сроком жизни элементов экономической инфраструктуры.
Момент, когда инвестирование капитала в новые элементы становится привлекательным, становится началом
\textit{повышательной волны} цикла. Активизация инвесторов приводит к исчерпанию финансовых ресурсов. Момент повышения
ссудного процента становится началом \textit{понижательной волны} цикла.

Одна из трактовок циклов Кондратьева:


\bigskip

1753\ \  \ \ \ \ \ \ 1814 \ \ \ \ \ \ \ \ \ \ \ \ 1873 \ \ \ \ \ \ \ \ \ \ \ \ \ 1921
\ \ \ \ \ \ \ \ \ \ \ \ \ \ \ \ 1970 \ \ \ \ \ \ \ \ \ \ \ \ \ \ \ 2005 \ \ \ \ \ \ \ \ \ \ \ \ \ 2041

[Warning: Draw object ignored][Warning: Draw object ignored][Warning: Draw object ignored][Warning: Draw object
ignored][Warning: Draw object ignored][Warning: Draw object ignored][Warning: Draw object ignored][Warning: Draw object
ignored][Warning: Draw object ignored][Warning: Draw object ignored][Warning: Draw object ignored][Warning: Draw object
ignored]

\ \ \ \ \ \ \ \ 1787 \ \ \ \ \ \ \ \ \ 1843 \ \ \ \ \ \ \ \ \ \ \ \ 1895 \ \ \ \ \ \ \ \ \ \ \ 1945
\ \ \ \ \ \ \ \ \ \ \ \ \ \ 1985 \ \ \ \ \ \ \ \ \ \ 2017

\ \ Кризис \ \ \ \ \ \ \ \ Потрясения \ \ \ \ \ \ \ Кризис \ \ \ \ \ \ \ \ Потрясения \ \ \ \ \ \ \ \ \ \ \ \ Кризис
\ \ \ \ \ \ \ \ \ Потрясения \ \ \ \ \ \ \ \ \ \ \ \ \ \

структуры \ \ \ \ \ \ \ \ \ \ \ \ \ \ \ \ \ \ \ \ \ \ \ \ \ \ \ \ \ \ \ структуры
\ \ \ \ \ \ \ \ \ \ \ \ \ \ \ \ \ \ \ \ \ \ \ \ \ \ \ \ \ \ \ \ \ \ структуры


\bigskip

\ \ \ \ \ \ \ \ \ \ \ \ Подъем \ \ \ \ \ \ \ \ Революция \ \ \ \ \ Подъем \ \ \ \ \ \ \ \ \ \ \ \ Революция
\ \ \ \ \ \ Подъем \ \ \ \ \ \ \ \ \ \ \ \ \ \ Революция

\ \ \ \ \ \ \ \ \ \ \ технологии \ \ \ \ \ \ \ \ рынка \ \ \ \ \ \ \ \ технологии \ \ \ \ \ \ \ \ \ \ \ \ рынка
\ \ \ \ \ \ \ \ \ \ технологии \ \ \ \ \ \ \ \ \ \ \ \ \ рынка


\bigskip


Длительность цикла \ \ \ \ \ 56 \ \ \ \ \ \ \ \ \ \ \ \ \ \ \ \ \ \ \ 52 \ \ \ \ \ \ \ \ \ \ \ \ \ \ \ \ \ \ \ 50
\ \ \ \ \ \ \ \ \ \ \ \ \ \ \ \ \ \ \ \ \ \ \ 40 \ \ \ \ \ \ \ \ \ \ \ \ \ \ \ \ \ \ \ \ \ \ 32

\bigskip

Прогресс и деградация одновременны в каждой точке эволюционного процесса. Под воздействием суточных, сезонных, годовых,
солнечных и др. флуктуаций в нелинейной экономике генерируются возмущения, ритмизирующие процесс ее развития: кризисы
структуры экономики и потрясения в фазе понижения, подъемы технологии и революции рынка в фазе понижения.

Изменение длительности циклов отражает ускорение хода эволюции экономики (и социума).

Сегодня наиболее актуальна проблема перехода от разрешения кризиса к (упреждающему) управлению кризисом. Общепринятой
концепции циклов пока нет.

\textbf{Социальные институты имеют циклы реформ — контрреформ}, дополняющие картину экономической и социальной
тектоники. Примеры для России:

Александр I \ → Николай I; \ \ \ \ Александр II \ → \ Александр III; \ \ \ Николай II → \ Ленин и Сталин;

Хрущев \ → \ Брежнев и др.; \ \ \ \ Горбачев, Ельцин \ \ → \ \ и далее везде.


\bigskip

Реформаторы представляют либерализм, контрреформаторы — патернализм.

Либеральные реформаторы игнорируют социальные проблемы населения.

Патерналистские контрреформаторы игнорируют права и свободы населения.

Тоталитарные контрреформаторы опираются на коллективистское сознание, объединение власти и собственности,
государственническую экономику и экстенсивное «развитие» в форме внешней экспансии.

Нехватка ресурсов для осуществления радикальных военно-индустриальных контрреформ порождает кризис и открывает дорогу в
новый цикл реформирования.

Население поддерживает тех и других: элита и общество не успевают созреть для понимания и решения созданных ими
проблем.

Кризисы во внешней среде препятствуют экстенсивному решению внутренних проблем.


\bigskip

\textbf{Традиции и новации }разделяют общество. Основные контроверзы:

Новое допускается, если оно не противоречит традиции \ $\leftrightarrow $ \ Новое всегда имеет преимущество

Мифологическая и религиозная организация общества \ \ \ $\leftrightarrow $ \ Светская организация общества

Общество подчинено природным циклам\ \ \ \  \ \ \ \ \ \ \ \ \ \ $\leftrightarrow $ \ Общество подчинено собственным
циклам

Проявление индивидуальности - антиобщественно \ \ \ \ \ \ \ \ \ \ $\leftrightarrow $ \ Интересы индивидуума превыше
всего

Вера {\textgreater} разум\ \ \ \ \ \ \ \ \ \  \ \ \ \ \ \ \ \ \ \ \ $\leftrightarrow $ \ Разум {\textgreater} вера

Авторитаризм\ \ \ \ \ \ \ \ \ \  \ \ \ \ \ \ \ \ \ \ \ $\leftrightarrow $ \ Демократия

Авторитет идеологии выше авторитета науки \ \ \ \ \ \ \ \ \ \ \ \ \ \ \ \ $\leftrightarrow $ \ \ Авторитет науки выше
авторитета идеологии

Основные связи в обществе — личные\ \ \ \  \ \ \ \ \ \ \ \ $\leftrightarrow $ \ \ Основные связи в обществе — безличные


\bigskip
%\end{document}
