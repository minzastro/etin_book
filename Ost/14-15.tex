% This file was converted to LaTeX by Writer2LaTeX ver. 1.4
% see http://writer2latex.sourceforge.net for more info
\documentclass[a4paper]{article}
\usepackage[utf8]{inputenc}
\usepackage[T2A,T1]{fontenc}
\usepackage[russian,english]{babel}
\usepackage{amsmath}
\usepackage{amssymb,amsfonts,textcomp}
\usepackage{color}
\usepackage{array}
\usepackage{hhline}
\usepackage{hyperref}
\hypersetup{pdftex, colorlinks=true, linkcolor=blue, citecolor=blue, filecolor=blue, urlcolor=blue, pdftitle=EQUIPMENT}
% footnotes configuration
\makeatletter
\renewcommand\thefootnote{\arabic{footnote}}
\makeatother
% Text styles
\newcommand\textstyleiii[1]{\foreignlanguage{russian}{\textbf{#1}}}
\newcommand\textstylePageNumber[1]{#1}
% Outline numbering
\setcounter{secnumdepth}{0}
% Page layout (geometry)
\setlength\voffset{-1in}
\setlength\hoffset{-1in}
\setlength\topmargin{0.3937in}
\setlength\oddsidemargin{0.7874in}
\setlength\textheight{10.1945in}
\setlength\textwidth{6.8897996in}
\setlength\footskip{0.35549998in}
\setlength\headheight{0.1972in}
\setlength\headsep{0.1583in}
% Footnote rule
\setlength{\skip\footins}{0.0469in}
\renewcommand\footnoterule{\vspace*{-0.0071in}\setlength\leftskip{0pt}\setlength\rightskip{0pt plus 1fil}\noindent\textcolor{black}{\rule{0.25\columnwidth}{0.0071in}}\vspace*{0.0398in}}
% Pages styles
\makeatletter
\newcommand\ps@Standard{
  \renewcommand\@oddhead{\textstylePageNumber{\foreignlanguage{russian}{Восток }\foreignlanguage{russian}{4}\hfill }}
  \renewcommand\@evenhead{}
  \renewcommand\@oddfoot{[Warning: Draw object ignored]}
  \renewcommand\@evenfoot{[Warning: Draw object ignored]}
  \renewcommand\thepage{\arabic{page}}
}
\makeatother
\pagestyle{Standard}
\title{EQUIPMENT}
\begin{document}
\clearpage\setcounter{page}{1}\pagestyle{Standard}
\section[14.
\textcyrillic{В}\textcyrillic{о}\textcyrillic{с}\textcyrillic{т}\textcyrillic{о}\textcyrillic{ч}\textcyrillic{н}\textcyrillic{е}\textcyrillic{е}
\textcyrillic{В}\textcyrillic{о}\textcyrillic{с}\textcyrillic{т}\textcyrillic{о}\textcyrillic{к}\textcyrillic{а}]{\foreignlanguage{russian}{14.
Восточнее Востока}}
\subsection[{\textquotedbl}Черная дыра{\textquotedbl}]{\selectlanguage{russian} {\textquotedbl}Черная
дыра{\textquotedbl}}
{\selectlanguage{russian}
Японская цивилизация — дочерняя по отношению к китайской. Японцы ввезли из Срединного царства рис, иероглифику, металлы,
ремесла, даосизм, конфуцианство, буддизм — и остались японцами. }

{\selectlanguage{russian}
Япония по отношению к внешнему миру любознательна, но закрыта. Япония поглощает информацию из внешнего мира, но не
распространяет информацию во внешнем мире. Периодам активного усвоения информации извне приходят на смену периоды
закрытости, изоляционизма. Структура японского культурного пространства обеспечивает быстрое прохождение информации и
товаров, культурную гомогенность. Четыре тысячи островов японцам оказалось легче интегрировать, чем другим этносам
сплотиться на территории протяженностью десять тысяч километров. \ }

{\selectlanguage{russian}
Японская экономика имеет судьбоносные особенности. Поливное рисосеяние обходится без удобрений, такое земледелие не
зависит от скотоводства, хозяйство не порождает агрессию, направленную на завоевание пастбищ. А как же белковая
недостаточность? – Близость любого поселения к невероятно богатому рыбой и прочим добром побережью (3492 съедобных
вида, а в Средиземном море – всего 1322) оставляет эту проблему на континенте. Побережье – 280 тысяч километров –
приносит удобрения, снабжает микроэлементами (мы не в степи!). Японская экономика интенсивна (трудоголики), а западная
— экстенсивна (завоеватели). }

{\selectlanguage{russian}
Япония расположена {\textquotedbl}восточнее Востока{\textquotedbl} и была естественным прибежищем активных и поэтому
теснимых обитателей континента. }

{\selectlanguage{russian}
Суходольное земледелие и собирательство в глубине островов, равно как и поливное рисосеяние и морской промысел на
побережье не были самодостаточны, так что товарообмен стал неотъемлемой частью хозяйственного механизма.}

{\selectlanguage{russian}\itshape
Флот монголов, пытавшихся – в лучших континентальных традициях – в \foreignlanguage{english}{XIII} веке вторгнуться в
Японию, был рассеян {\textquotedbl}божественным ветром{\textquotedbl} (камикадзе).}

{\selectlanguage{russian}
Языком устного общения был японский, с множеством диалектов, языком письменного общения стал иероглифический
(китайский). Как и в Китае, иероглифическая письменность послужила средством преодоления диалектной раздробленности.
Японцы писали не только государственные правовые и налоговые документы, но и летописи. В Японии несколько
письменностей: японская культура, как и экономика, интенсивна. Школа (кокутай) = язык + мораль + география + история;
изучаемые тексты не адаптированы. В школе перечисляют преданных императору людей. Школа дает видение нации. }

{\selectlanguage{russian}
В доброе старое время чиновник имел в году 320 рабочих дней и 185 рабочих вечеров. Рабочий день начинался в 2 часа
(ночи!) и заканчивался в полдень: такой порядок обеспечивал своевременную обработку пришедших \textit{за день}
документов.\footnote{\foreignlanguage{russian}{\ Мы — не в Японии!}} Для учета рабочего времени чиновник приходил на
работу с биркой (рабочим номером). }

{\selectlanguage{russian}
Нам не понять: \ как японская фирма за свой счет справляет свадьбу каждой сотрудницы, как по случаю рождения ребенка
маме предоставляют отпуск с сохранением заработка, как прием на работу считается пожизненным — и многого другого… }

\subsection[Духовная жизнь]{\selectlanguage{russian} Духовная жизнь}
{\selectlanguage{russian}
Стержень — этико-политическая конфуцианская система, возникшая после синтеза традиционного \textit{синтоизма} и буддизма
в формате \textit{махáяны}. Высшее благо — благо государства. Распространение
дзэн-буддизма\footnote{\foreignlanguage{russian}{\ {\textquotedbl}Каждый должен найти своего Учителя{\textquotedbl}.}}
породило представление о ценности личности. }

{\selectlanguage{russian}
Даосская магия (школа медитации) прижилась в Японии, но ее сферой остались гадания и (китайская) медицина. }

{\selectlanguage{russian}
Консерватизм и культ господина в неоконфуцианстве пришлись очень кстати в политической структуре Японии. Великий
всеобщий порядок \textit{ли} оказался идентичен восьми миллионам синтоистских божеств. Синтоизм и неоконфуцианство
{\textquotedbl}слились в экстазе{\textquotedbl}. }

{\selectlanguage{russian}
Религиозные секты вели обособленное (от общественных процессов) существование, поэтому в элите не было характерных для
Запада напряжений и не возникали утопии. }

{\selectlanguage{russian}
Плюрализм и национальная идентичность в равной мере характерны для японской духовности. }

{\selectlanguage{russian}\itshape
Во время Второй мировой войны в Японии были запрещены Библия, Тойнби, Маркс и Бертран Рассел.}

\subsubsection{Синтоизм}
{\selectlanguage{russian}
Синто = путь духов (\textit{ками}). Синтоистский шаманизм опирался на ритуалы почитания предков, еды, шествий, бесед,
жертвоприношений в дни хозяйственных праздников. Пантеон богов пополнялся бесконтрольно жрецами, строившими не реже
одного раза в 20 лет новые храмы. Посещение храма — почти как у нас: человек заходит, останавливается перед
{\textquotedbl}алтарем{\textquotedbl}, бросает монетку в ящик перед ним, кланяется и после хлопка в ладоши уходит. }

{\selectlanguage{russian}
\textit{Ками }невидимы, кроме одного — \textit{тэнно} (так японские даосы называли \textit{небесного владыку}).
Европейцы на свой лад называют этого \textit{ками }\textbf{императором}. \textbf{\ }\ }

{\selectlanguage{russian}
Укрепившийся после слияния с неоконфуцианством синтоизм в \foreignlanguage{english}{XX} веке стал государственной
идеологией. После поражения во Второй мировой войне синтоизм сильно изменился, но сохранился в характерном для Японии
духе корпоративности.}

\subsubsection{Буддизм}
{\selectlanguage{russian}
Буддизм проник в Японию в середине \foreignlanguage{english}{VI} века и стал инструментом политической борьбы.
Буддийское монашество пришлось очень кстати: император, по обычаю отрекавшийся от престола в пользу наследника, теперь
становился монахом и \textit{продолжал управлять} страной в качестве регента. Примеру императора следовали и чиновники.
В конце \foreignlanguage{english}{VII} века буддизм стал официальной государственной
религией.\footnote{\foreignlanguage{russian}{\ Фигуру Будды высотой 16 метров покрыли золотом, собранным по всей
Японии.}}}

{\selectlanguage{russian}
Для народа буддийские заклинания свели к фразе {\textquotedbl}Наму Амиду буцу{\textquotedbl} (или к одному
{\textquotedbl}Амида{\textquotedbl}). Повторение (медитация!) этих коротких формул (до 70 тысяч раз в сутки)
гарантировало блаженство.}

{\selectlanguage{russian}
Удачное решение нашли в Японии для сближения синтоизма и буддизма: бесчисленные боги просто были объявлены буддами в
очередных перерождениях. Совместные праздники скрепили этот (непостижимый для Запада) союз.}

{\selectlanguage{russian}
Буддизм вызвал к жизни архитектуру, скульптуру, изобразительное искусство. Их развитие шло темпами, неслыханными на
Западе и на Востоке. }

{\selectlanguage{russian}\itshape
Пагода имеет пять крыш, воплощающих пять элементов мира: дерево, огонь, землю, железо и воду.}

{\selectlanguage{russian}
Консерватизм и культ господина в неоконфуцианстве пришлись очень кстати в политической структуре Японии. Великий
всеобщий порядок \textit{ли} пришелся кстати восьми миллионам синтоистских божеств. Синтоизм и неоконфуцианство
{\textquotedbl}слились в экстазе{\textquotedbl}. }

\subsubsection{Дзэн}
{\selectlanguage{russian}
\textbf{Дзэн} (от санскритского \textit{дхьяна} - медитация, сосредоточение, созерцание; китайское \textit{чань}) как
одна из школ буддизма возник в Китае в \foreignlanguage{english}{V}{}-\foreignlanguage{english}{VI} веках, примерно с
1200 года получил распространение в Японии, где существует до настоящего времени. Для \textit{Дзэн} характерны
критическое отношение к тексту как средству передачи истинного знания, вера в возможность обретения
{\textquotedbl}просветления{\textquotedbl} (\textit{сартори}) посредством размышления над бессмысленным с позиции
логики диалогом или вопросом (\textit{коан}), элементы интуитивизма. С середины \foreignlanguage{english}{XX} века
приобрел популярность среди западной интеллигенции (в рамках контркультуры).}

{\selectlanguage{russian}
\ \ Суровость, даже жестокость дзэн-воспитания пришлись по вкусу самураям. Мальчиков учили делать харакири, девочек –
закалываться кинжалом. Никакого вечного блаженства, только достойная смерть и место в памяти живых! }

{\selectlanguage{russian}
\textbf{Дзэн} представляет собой уникальное смешение философских систем, принадлежащих различным культурам, включает в
себя даосскую любовь к простоте, естественности и спонтанности и всеохватывающий прагматизм конфуцианства. }

{\selectlanguage{russian}
Дзэн — это упражнение в просветлении, причем \ просветление не имеет ни объяснения, ни \ \ истолкования. \textit{Дзэн}
не располагает специальным учением или философией, формальными символами веры или догмами и утверждает, что именно
свобода ото всех установленных убеждений делает его духовное содержание подлинным. Сильнее, чем какая-либо другая
школа, \textit{Дзэн} убежден в том, что слова не могут выразить высшую истину:}

{\selectlanguage{russian}
{\textquotedbl}Особое учение вне писаний, \newline
Не основанное на словах и буквах, \newline
Взывающее непосредственно к душе человека, \newline
Прозревающее природу каждого \newline
И позволяющее достичь Буддовости.{\textquotedbl}}

{\selectlanguage{russian}
Техника \textit{Дзэн} типична для \textit{японского} образа мышления — скорее интуитивного, чем интеллектуального,
предпочитающего излагать факты без пространных пояснений. Наставники дзэн, отрицая понятийное мышление, разработали
методы непосредственного указания истины при помощи внезапных спонтанных реплик или действий, останавливающих
мыслительный процесс и подводящих ученика к мистическому и одновременно естественному и спонтанному восприятию
действительности. }

{\selectlanguage{russian}
Просветление в \textit{Дзэн} означает не удаление от мира, а, наоборот, активное участие в повседневных делах, означает
мгновенное восприятие Буддовости всего сущего, и в первую очередь — вещей, дел и людей, принимающих участие в
повседневной жизни. \ }

{\selectlanguage{russian}\itshape
Когда дзэнского наставника спросили, как он представляет себе поиски природы Будды, он ответил: {\textquotedbl}Это
похоже на то, как если бы кто-то ездил на быке в поисках этого быка{\textquotedbl}.}

{\selectlanguage{russian}
Сегодня в \textit{Дзэн} существуют две основные школы. }

{\selectlanguage{russian}
\textbf{Школа Риндзай} (внезапная) \ уделяет основное внимание периодическим беседам ученика с учителем, проходящим в
формальной обстановке. Ученик описывает достигнутое им восприятие \textit{коана}. Для решения \textit{коана} необходимы
длительные периоды усиленной концентрации, которые в итоге приводят к внезапному прозрению. Опытный наставник может
распознать то состояние ученика, при котором он находится на грани внезапного просветления, и
{\textquotedbl}втолкнуть{\textquotedbl} его в \textit{сатори} при помощи неожиданного поступка — удара палкой или
крика. }

{\selectlanguage{russian}\itshape
Монах, пришедший просить о наставничестве, сказал Бодхидхарме: {\textquotedbl}Мое сознание неспокойно. Пожалуйста,
успокойте мое сознание. — Принеси мне сюда свое сознание, — ответил Бодхидхарма, — и я его успокою! — Но когда я ищу
свое сознание, — сказал монах, — я не могу найти его. — Вот! — хлопнул в ладоши Бодхидхарма. — Я успокоил твое
сознание! {\textquotedbl}}

{\selectlanguage{russian}
\textbf{Школа Сото} (постепенная) избегает шоковых методов Риндзай и готовит постепенное созревание \textit{Дзэн},
{\textquotedbl}подобного весеннему ветерку, ласкающему цветок, помогая ему распуститься{\textquotedbl}. Применяются две
основные формы медитации: {\textquotedbl}тихая сидячая{\textquotedbl} и повседневные занятия и работа. }

{\selectlanguage{russian}
\textit{Дзэн} оказал огромное влияние на все стороны образа жизни японцев. Среди них не только различные виды искусства
и ремесла, но также разнообразные церемонии (чаепития и составления букета), воинские искусства (стрельба из лука,
фехтование и дзюдо). Каждый из этих видов деятельности называется \textit{До}, или Путь к просветлению ( =
\ \textit{Дао}). }

{\selectlanguage{russian}
Медленные, установленные ритуалом движения участников чаепития, спонтанный росчерк пера или кисти в живописи или
каллиграфии, \ духовный кодекс БУСИДО ({\textquotedbl}Путь воина{\textquotedbl}) — все это воплощает спонтанность,
простоту и абсолютное присутствие духа \textit{дзэнского} образа жизни. Хотя все они требуют совершенства техники,
истинное мастерство достигается только лишь тогда, когда возможности техники исчерпаны, когда искусство становится
{\textquotedbl}безыскусным искусством{\textquotedbl}, прямым продолжением подсознания.}

{\selectlanguage{russian}
Эстетика \textit{Дзэн} господствует в Японии.}

\subsection[\textcyrillic{С}\textcyrillic{о}\textcyrillic{в}\textcyrillic{р}\textcyrillic{е}\textcyrillic{м}\textcyrillic{е}\textcyrillic{н}\textcyrillic{н}\textcyrillic{ы}\textcyrillic{е}
\textcyrillic{с}\textcyrillic{е}\textcyrillic{к}\textcyrillic{т}\textcyrillic{ы}]{\textstyleiii{Современные}\foreignlanguage{russian}{
}\textstyleiii{секты}}
{\selectlanguage{russian}
Секты \ практичны, мистика уступила место трезвой взаимопомощи, взаимному доверию, самоидентификации среди
{\textquotedbl}своих{\textquotedbl}. Секта – прежде всего организация. Численность некоторых сект исчисляется
миллионами.}

{\selectlanguage{russian}\itshape
Есть и террористические секты. Химическое, биологическое оружие и обыкновенную взрывчатку они получают на Западе.}

\subsection[Япония и Запад]{\selectlanguage{russian} Япония и Запад}
{\selectlanguage{russian}
В прошлом веке Япония неоднократно пыталась усвоить западный экспансионизм – каждый раз мало сказать неудачно, а с
катастрофическими последствиями. Однако из последнего поражения Япония вышла носителем {\textquotedbl}японского
экономического чуда{\textquotedbl}. Неудивительно: нам всякий результат длительного сосредоточенного труда
представляется {\textquotedbl}экономическим чудом{\textquotedbl}. }

{\selectlanguage{russian}
Подлинным чудом (непонятным {\textquotedbl}простому человеку{\textquotedbl}) не впервые оказывается интенсивный труд в
атмосфере охраняемой традициями самоорганизующейся стабильности.}


\bigskip

\section[15. Общие замечания{}-2 ]{\selectlanguage{russian} 15. Общие замечания-2 }
\subsection[Точка бифуркации]{\selectlanguage{russian} Точка бифуркации}
{\selectlanguage{russian}
Скотоводы и земледельцы — две ветви развития из точки демографической катастрофы (вымерло 90\%), носящей название
{\textquotedbl}неолитическая революция{\textquotedbl}. }

{\selectlanguage{russian}
Скотоводы — уцелевшие охотники — сохранили ресурс экспансии и приложили его к борьбе за пастбища. Этот же ресурс они
затрачивали на охоту на земледельцев. В популяциях скотоводов продолжался интенсивный отбор левополушарных — будущих
ораторов, политиков, философов, ученых, создателей Запада. Усвоив навыки возделывания земли, они создали себе проблему
распределения земли на пастбище и пашню. И до сих пор борются с ней (не видя ее). В кипящем пространстве Евразии
беспрестанно переселявшиеся народы утрачивали традиции, заимствовали {\textquotedbl}чужой ум{\textquotedbl}, истребляя
его носителей. }

{\selectlanguage{russian}
Правополушарные земледельцы воздвигали обнесенные стенами города и хранили традиции. }

{\selectlanguage{russian}
\textbf{Когнитивная эволюция и нейромедиаторы.} Процессы в правом полушарии сопровождаются выработкой нейромедиаторов,
сходных по составу (и по действию!) с опиатными нейромедиаторами, выделяемыми вегетативной нервной системой при
выполнении {\textquotedbl}бытовых{\textquotedbl} функций. Акт творчества - вознагражден! Мистический опыт — источник
опиатного допинга. }

{\selectlanguage{russian}
Разной была цена человеческой жизни в этих ветвях. }

{\selectlanguage{russian}
\textbf{Осевое время}. Бронзовый век породил профессиональные армии, вооруженные тяжелыми мечами. Появление дешевого
железа породило массовую армию и массовую гибель людей. Обеспокоенные
{\textquotedbl}умники{\textquotedbl}\footnote{\foreignlanguage{russian}{\ Конфуций, Будда, Заратустра, Пифагор, Сократ
и многие другие.}} выступили перед элитой с проповедью: {\textquotedbl}Знание есть добродетель. Жестокость – от
недостаточной мудрости. Быть мудрым значит быть добрым{\textquotedbl}. Культура активно выступила как регулятор
агрессии. }

\subsection[Восточная ветвь ]{\selectlanguage{russian} Восточная ветвь }
{\selectlanguage{russian}
Нет единой Восточной цивилизации, есть восточный тип общества, отличный от
западного\footnote{\foreignlanguage{russian}{\ Цивилизации Востока радикально отличны друг от друга.}}. }

{\selectlanguage{russian}
Запад: {\textquotedbl}Свобода человека — превыше всего{\textquotedbl}. Власть регулирует отношения между людьми.}

{\selectlanguage{russian}
Восток: {\textquotedbl}Нация — это семья{\textquotedbl}. Власть осуществляет закон в семье.}

{\selectlanguage{russian}
Мировоззрение в восточных традициях основано на прямом, не опосредованном рассудком восприятии действительности. Оно
имеет характерные черты, не зависящие от того, на каком географическом, историческом и культурном фоне разворачивается
данная традиция. Индуист и даос могут выделять разные аспекты этого мировосприятия, японский и индийский буддисты могут
по-разному описывать свои ощущения, но основные элементы мировоззрения во всех этих традициях совпадают.}

{\selectlanguage{russian}
Самая важная характерная черта восточного мировоззрения, можно сказать, его сущность, — осознание единства и
взаимосвязанности всех вещей и явлений, восприятие всех явлений природы как проявлений лежащего в основе единственной
сущности. В индуизме она называется \textit{Брáхман}, в буддизме — \textit{Дхармакайя}, в даосизме — \textit{Дао}. }

{\selectlanguage{russian}
В обычной жизни мы не осознаем этого единства, разделяя мир на самостоятельные предметы и события. Эта абстракция,
порожденная нашим разграничивающим и категоризирующим интеллектом — не более чем иллюзия. Индуисты, например, считают,
что эта иллюзия порождена \textit{Авидьей}, то есть неведением ума, околдованного \textit{Майей}. Традиция
{\textquotedbl}исправляет{\textquotedbl} сознание при помощи медитации, делает его уравновешенным и спокойным.
\textit{Самадхи} (медитация) буквально переводится как {\textquotedbl}душевное равновесие{\textquotedbl}. }

\subsection[Власть, собственность и культура]{\selectlanguage{russian} Власть, собственность и культура}
{\selectlanguage{russian}
Частный собственник (= личность = индивидуум) должен быть управляем — государством (Китай), обществом (Индия), властью и
обществом (Япония). Культура во всех случаях – инструмент управления. Сохранение структуры имеет безусловный приоритет
перед экономической активностью. Восточная структура не консервативна, как поверхностно оценивают ее европейцы, а
устойчива и имеет огромный эволюционный ресурс. }

\subsection[Устойчивость культуры]{\selectlanguage{russian} Устойчивость культуры}
{\selectlanguage{russian}
\textbf{Примеры} устойчивых культур — прямо перед нами.}

{\selectlanguage{russian}
\textbf{Цыгане}\footnote{\foreignlanguage{russian}{\ От греческого }\foreignlanguage{russian}{\textit{аттинганос =
атсинганос = }}\foreignlanguage{russian}{предсказатели. В России –}\foreignlanguage{russian}{\textit{
ромалэ}}\foreignlanguage{russian}{ (самоназвание от
}\foreignlanguage{russian}{\textit{ром}}\foreignlanguage{russian}{).}} в России — православные, в Западной Европе –
католики и протестанты, в Средней Азии — мусульмане. Религия не входит в собственную культуру цыган. }

{\selectlanguage{russian}
Цыгане везде — эндогамны. Цыганки все-таки могут выходить замуж {\textquotedbl}на сторону{\textquotedbl}, но прием в
цыганскую общину {\textquotedbl}чужого{\textquotedbl} мужчины — редчайшее исключение. Добрачное поведение цыганок —
безупречно. Если девушка нарушала Правило, ее родители совершали в хомуте круг позора по табору.}

{\selectlanguage{russian}
Культура цыган существует полторы тысячи лет в диаспоре и в Индии. Осколок периферийной касты низшей варны
шудр\footnote{\foreignlanguage{russian}{\ Уличные скоморохи.}}, рассеянный по всему миру, притесняемый и истребляемый,
— носитель устойчивого культурного гена. Основой этнокультурной устойчивости цыган не является религия, ее основой
является культура — в полном смысле слова. }

{\selectlanguage{russian}\itshape
Представьте себе устойчивость культуры Индии с миллиардом носителей генов культуры.}

{\selectlanguage{russian}
\textbf{Русская литература} — действительно Великая литература. Основанная не шибко образованными дворянами, говорившими
дома где по-французски\footnote{\foreignlanguage{russian}{\ Пушкин в Лицее имел уважительную кличку
{\textquotedbl}француз{\textquotedbl}.}}, где по-немецки, появившаяся в поголовно неграмотной стране и продолженная
худородными разночинцами, расстрелянная и растоптанная едва научившимися читать гегемонами, она жива и по сей день —
\textbf{всемирное чудо самодостаточности и устойчивости. }}
\end{document}
