% This file was converted to LaTeX by Writer2LaTeX ver. 1.4
% see http://writer2latex.sourceforge.net for more info
\documentclass[a4paper]{article}
\usepackage[utf8]{inputenc}
\usepackage[T2A,T1]{fontenc}
\usepackage[russian,english]{babel}
\usepackage{amsmath}
\usepackage{amssymb,amsfonts,textcomp}
\usepackage{color}
\usepackage{array}
\usepackage{hhline}
\usepackage{hyperref}
\hypersetup{pdftex, colorlinks=true, linkcolor=blue, citecolor=blue, filecolor=blue, urlcolor=blue, pdftitle=\textcyrillic{Осевое время}}
% footnotes configuration
\makeatletter
\renewcommand\thefootnote{\arabic{footnote}}
\makeatother
% Outline numbering
\setcounter{secnumdepth}{0}
% List styles
\newcommand\liststyleWWviiiNumi{%
\renewcommand\labelitemi{{\textbullet}}
\renewcommand\labelitemii{o}
\renewcommand\labelitemiii{${\blacksquare}$}
\renewcommand\labelitemiv{{\textbullet}}
}
% Page layout (geometry)
\setlength\voffset{-1in}
\setlength\hoffset{-1in}
\setlength\topmargin{0.5909in}
\setlength\oddsidemargin{0.5909in}
\setlength\textheight{10.511099in}
\setlength\textwidth{7.087in}
\setlength\footskip{0.0cm}
\setlength\headheight{0cm}
\setlength\headsep{0cm}
% Footnote rule
\setlength{\skip\footins}{0.0469in}
\renewcommand\footnoterule{\vspace*{-0.0071in}\setlength\leftskip{0pt}\setlength\rightskip{0pt plus 1fil}\noindent\textcolor{black}{\rule{0.25\columnwidth}{0.0071in}}\vspace*{0.0398in}}
% Pages styles
\makeatletter
\newcommand\ps@Standard{
  \renewcommand\@oddhead{}
  \renewcommand\@evenhead{}
  \renewcommand\@oddfoot{}
  \renewcommand\@evenfoot{}
  \renewcommand\thepage{\arabic{page}}
}
\makeatother
\pagestyle{Standard}
\title{\textcyrillic{Осевое время}}
\begin{document}
\clearpage\setcounter{page}{1}\pagestyle{Standard}
\subsection{Осевое время}
{\selectlanguage{russian}
Осевое время\footnote{\ \textcyrillic{Концепцию осевого времени ввел Карл Ясперс (1883-1963) в сочинении
{\textquotedbl}Смысл и назначение истории{\textquotedbl} (рус. 1991).}} — в широком смысле с
\foreignlanguage{english}{V} в. до н.э. по \foreignlanguage{english}{V} в. н.э. — рассматривается как ось мировой
истории. Период 800-200 гг. до н.э. был временем духовного напряжения, временем духовного прорыва:}

\liststyleWWviiiNumi
\begin{itemize}
\item {\selectlanguage{russian}
Китай: Лао-цзы и Конфуций;}
\item {\selectlanguage{russian}
Индия\footnote{\ \textcyrillic{В Индии осевые процессы восходят ко времени создания Вед.}}: Упанишады, Будда;}
\item {\selectlanguage{russian}
Иран: Заратустра;}
\item {\selectlanguage{russian}
Греция: Гомер, Платон, Архимед, Фалес, Пифагор;}
\item {\selectlanguage{russian}
Палестина: Ветхий Завет, Новый Завет.}
\end{itemize}
{\selectlanguage{russian}
Осевые народы: китайцы, индийцы, иранцы, иудеи, греки. Македонцы и римляне не смогли принять опыт Осевого времени.}

{\selectlanguage{russian}
Осевое время ознаменовало конец великих культур древности. Наступил конец мифологической эпохи, началась эпоха
философии. Осевое время = время великого \textbf{прорыва}. }

{\selectlanguage{russian}
В Осевое время произошло осознание бытия, мышление стало объектом мышления, были созданы основы мировых религий и
основные категории во всех сферах мысли. }

{\selectlanguage{russian}
В Осевое время в орбитах великих культур древности (Нил, Хуанхэ, Инд, Тигр и Евфрат) формируется духовная основа
человечества. Произошло тотальное отрицание наличного земного бытия человека в пользу новых высших духовных ценностей,
нового мировоззрения, творческих форм активности, единства общества. Жизнь получила универсальный смысл. Возникла
духовная элита, отделенная Текстами от других элит.}

{\selectlanguage{russian}
Впоследствии наступившая свобода стала анархией, завершившейся созданием империй в Китае, в Индии, в Европе и — гибелью
этих империй. }


\bigskip

{\selectlanguage{russian}
Ренессанс — эпоха возрождения возможностей Осевого времени. Мы уже употребили эти возможности \textbf{для
научно-технического} \textbf{прорыва}. }


\bigskip

{\selectlanguage{russian}\bfseries
А что сделаете вы?}


\bigskip

{\selectlanguage{russian}
Добавлю от себя.}


\bigskip

{\selectlanguage{russian}
Возникновение и параллельное (независимое) развитие нового мироощущения в столь несхожих популяциях есть проявление
общего процесса \textit{эволюции}. В ту же эпоху освоившие лошадь кочевники приступили к экспансии — территориальной, а
не духовной. Может быть, это разные проявления \textit{пассионарности} — по Л.Н. Гумилеву?.. }


\bigskip
\end{document}
