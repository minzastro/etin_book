% This file was converted to LaTeX by Writer2LaTeX ver. 1.4
% see http://writer2latex.sourceforge.net for more info
\documentclass[a4paper]{article}
\usepackage[utf8]{inputenc}
\usepackage[T2A,T1]{fontenc}
\usepackage[russian,english]{babel}
\usepackage{amsmath}
\usepackage{amssymb,amsfonts,textcomp}
\usepackage{color}
\usepackage{array}
\usepackage{hhline}
\usepackage{hyperref}
\hypersetup{pdftex, colorlinks=true, linkcolor=blue, citecolor=blue, filecolor=blue, urlcolor=blue, pdftitle=EQUIPMENT}
% footnotes configuration
\makeatletter
\renewcommand\thefootnote{\arabic{footnote}}
\makeatother
% Outline numbering
\setcounter{secnumdepth}{0}
% List styles
\newcommand\liststyleWWviiiNumvi{%
\renewcommand\theenumi{\arabic{enumi}}
\renewcommand\theenumii{\alph{enumii}}
\renewcommand\theenumiii{\roman{enumiii}}
\renewcommand\theenumiv{\arabic{enumiv}}
\renewcommand\labelenumi{\theenumi.}
\renewcommand\labelenumii{\theenumii.}
\renewcommand\labelenumiii{\theenumiii.}
\renewcommand\labelenumiv{\theenumiv.}
}
\newcommand\liststyleWWviiiNumx{%
\renewcommand\theenumi{\arabic{enumi}}
\renewcommand\theenumii{\alph{enumii}}
\renewcommand\theenumiii{\roman{enumiii}}
\renewcommand\theenumiv{\arabic{enumiv}}
\renewcommand\labelenumi{\theenumi.}
\renewcommand\labelenumii{\theenumii.}
\renewcommand\labelenumiii{\theenumiii.}
\renewcommand\labelenumiv{\theenumiv.}
}
\newcommand\liststyleWWviiiNumvii{%
\renewcommand\theenumi{\arabic{enumi}}
\renewcommand\theenumii{\alph{enumii}}
\renewcommand\theenumiii{\roman{enumiii}}
\renewcommand\theenumiv{\arabic{enumiv}}
\renewcommand\labelenumi{\theenumi.}
\renewcommand\labelenumii{\theenumii.}
\renewcommand\labelenumiii{\theenumiii.}
\renewcommand\labelenumiv{\theenumiv.}
}
% Page layout (geometry)
\setlength\voffset{-1in}
\setlength\hoffset{-1in}
\setlength\topmargin{0.3937in}
\setlength\oddsidemargin{0.7874in}
\setlength\textheight{9.9582in}
\setlength\textwidth{6.8897996in}
\setlength\footskip{0.5918in}
\setlength\headheight{0.1972in}
\setlength\headsep{0.1583in}
% Footnote rule
\setlength{\skip\footins}{0.0469in}
\renewcommand\footnoterule{\vspace*{-0.0071in}\setlength\leftskip{0pt}\setlength\rightskip{0pt plus 1fil}\noindent\textcolor{black}{\rule{0.25\columnwidth}{0.0071in}}\vspace*{0.0398in}}
% Pages styles
\makeatletter
\newcommand\ps@Standard{
  \renewcommand\@oddhead{\textstylePageNumber{\foreignlanguage{russian}{Восток 2}}}
  \renewcommand\@evenhead{\textstylePageNumber{\foreignlanguage{russian}{Восток 2}}}
  \renewcommand\@oddfoot{[Warning: Draw object ignored]}
  \renewcommand\@evenfoot{[Warning: Draw object ignored]}
  \renewcommand\thepage{\arabic{page}}
}
\makeatother
\pagestyle{Standard}
\title{EQUIPMENT}
\begin{document}
\clearpage\setcounter{page}{1}\pagestyle{Standard}
{\selectlanguage{russian}
После ознакомления с даршанами \textit{астики} перейдем к даршанам \textit{настики}. Эти оппозиционные доктрины сделали
своим фундаментом этику.}

\section[5. Джайнизм]{\selectlanguage{russian} 5. Джайнизм}
\subsection[Немного истории]{\selectlanguage{russian} Немного истории}
{\selectlanguage{russian}
Джайнизм\footnote{\foreignlanguage{russian}{\ Оформился одновременно с буддизмом. Есть мнение, что оба учения имеют
общий «корень».}} – неортодоксальное учение кшатрийской аристократии. Кшатрии конкурировали с брахманами и полагали
убыточными для себя обильные жертвоприношения. Основал это учение Джина = духовный победитель (599 - 527). }

{\selectlanguage{russian}
Джайнские тексты \ записывались с \foreignlanguage{english}{IV} в. до н.э. по \foreignlanguage{english}{V} в. н.э. на
разговорных языках: учение становилось доступным, языки обогащались текстами.}

{\selectlanguage{russian}
Джайнов ныне 2-3 миллиона, очень живучая секта.}

{\selectlanguage{russian}\itshape
Джайны имели обычай обрезания, это помогло им пережить нашествие мусульман. }

\subsection[Теория познания]{\selectlanguage{russian} Теория познания}
{\selectlanguage{russian}
\textit{Познание} подобно свету: оно обнаруживает себя и познаваемое. Различают пять правильных \textit{видов познания}
(первые два – опосредствованные, остальные – непосредственные).}

{\selectlanguage{russian}
1. Обычное познание: восприятие, мысль, память, узнавание, индукция, дедукция.}

{\selectlanguage{russian}
2. Свидетельство (=авторитет): коллективное познание с помощью слов, символов и знаков. }

{\selectlanguage{russian}
3. Ясновидение – непосредственное познание вещей.}

{\selectlanguage{russian}
4. Телепатия – непосредственное безошибочное познание мыслей других людей.}

{\selectlanguage{russian}
5. Всеведение – знание абсолюта. \ Его можно ощутить, но нельзя описать и передать.}

{\selectlanguage{russian}
Три неправильных способа познания: сомнение, ошибка, небрежное / ложное знание.}

{\selectlanguage{russian}
Существуют две \textit{формы познания}. \textit{Прамана} есть познание вещей как таковых, с помощью абстракций выделения
и отождествления. \textit{Ная} есть познание вещей в их отношениях, рассмотренных с выбранной нами – на основе наших
устремлений – точки зрения. Точка зрения не дает полного представления, но что поделаешь!}

{\selectlanguage{russian}
Всякое познание только вероятно, только гипотетично\footnote{\foreignlanguage{russian}{\ В логике джайнов нет места
нашему скепсису.}}. Различия охватывает и интегрирует сама реальность, которая многообразна и изменчива. Объединение
всех частных точек зрения не приводит к истине, к ней приводит всеведение, которым наделены \textit{освобожденные}. В
познании мы поднимаемся выше \textit{я}, исключающего другие \textit{я}.}

{\selectlanguage{russian}
Имеются три формы сознания: \textit{познание}, \textit{ощущение}, \textit{желания}. Душа имеется во всех живых телах,
поэтому сострадать следует всему живому. Души обладают разной степенью сознательности. Душа имеет размеры и соразмерна
телу. К концу жизни душа сжимается до размеров, требуемых следующим рождением. }

{\selectlanguage{russian}
Дух и материя \ независимы и самостоятельны. Вместе с тем активная душа определяет свойства вмещающего ее пассивного
тела.}

{\selectlanguage{russian}
Джайны преисполнены оптимизма.}

\subsection[Этика]{\selectlanguage{russian} Этика}
{\selectlanguage{russian}
Основа джайнской этики – отказ от владения имуществом. \textit{Дигамбары} (одетые небом) ходят нагишом, не признают
ничьих текстов и не имеют своих, \textit{шветамбары} (одетые в белое) носят белую одежду и более умеренны в
самоистязании.}

{\selectlanguage{russian}
\textit{Путь к свободе} пролегает через веру в Джину, познание учения и безупречное поведение. Пять обетов: не наносить
вреда живому (\textit{ахимса}), говорить правду, не красть, не прелюбодействовать, не стяжать. Монахи давали еще и обет
безбрачия. Джайны не противопоставляют своего учения другим, а стремятся просветить непросвещенных. Именно джайнам
индийцы обязаны исчезновением кровавых жертвоприношений. }

{\selectlanguage{russian}
Правильная вера, правильное знание и правильное поведение приводят к природному совершенству: безграничной вере,
безграничной силе и безграничному блаженству. Высшая добродетель — терпение, удовольствие — источник греха. }

{\selectlanguage{russian}\itshape
\ «О, человек! Ты – свой собственный друг; почему желаешь ты друзей вне себя \ самого?»}

{\selectlanguage{russian}
Когда достоинство исчерпывается, жизнь уходит.}

\subsection[Картина мира]{\selectlanguage{russian} Картина мира}
{\selectlanguage{russian}
Концепция сотворения мира не объясняет страдание. Концепция предопределенности делает невозможной моральную
ответственность.}

{\selectlanguage{russian}
Субъект = \textit{джива} = наслаждающийся (=душа). Объект = \textit{аджива} = доставляющий наслаждение. }

{\selectlanguage{russian}
\textbf{Дживы} состоят из души и тела. Дживы делятся на \textit{зависимые} (в колесе рождений), \textit{освобожденные}
(не становятся вечно совершенными), \textit{вечно совершенные}.}

{\selectlanguage{russian}
Карма обволакивает дживу оболочками: знанием, интуицией, эмоциями, самооценкой, сроком жизни, обстоятельствами
существования, ограничением желаний.}

{\selectlanguage{russian}
Одни а\textbf{дживы }не имеют формы (пространство, время, дхарма, адхарма), другие имеют форму (материя). }

{\selectlanguage{russian}
\textit{Пространство} разделено на часть, занятую миром вещей, и часть, которая абсолютно пуста (бездонное ничто).
\textit{Время} – процесс перехода от прошлого к настоящему. Различают вечное (неизмеряемое) и относительное
(измеряемое) время. Время — разрушитель.}

{\selectlanguage{russian}
\textit{Дхарма} – посредник, но не причина движения. Дхарма бестелесна, лишена чувственных качеств, неделима, вечна.
Дхарма сосуществует с миром вещей. \textit{Адхарма} — посредник, но не причина покоя. Адхарма бестелесна, лишена
чувственных качеств, неделима, вечна и сосуществует с бездонным \textit{ничто}.}

{\selectlanguage{russian}
В \textit{точке пространства} могут существовать один элемент дхармы, один элемент адхармы, частица времени и атомы
материи.}

{\selectlanguage{russian}
\textit{Материя} – носитель энергии. Материя может быть грубой или тонкой. Материя состоит из неделимых, бесконечно
малых, вечных, не имеющих формы (но имеющих вес) атомов. Атом имеет вкус, цвет, запах и вязкость/сухость. Атомы
соединяются в агрегаты, образуя \textit{элементы}: землю, воду, огонь, воздух. \textit{Мир} — самый большой агрегат.
Движение может быть простым, а может быть эволюцией.}

{\selectlanguage{russian}
\textit{Мокша – }абсолютное освобождение дживы от адживы\textit{.} }

\section{6. \textcyrillic{Буддизм}}
\subsection[Немного истории]{\selectlanguage{russian} Немного истории}
{\selectlanguage{russian}
Еще одним открытым — в отличие от брахманизма — неортодоксальным учением стал буддизм, возникший, как и джайнизм, в
среде кшатриев (VI век до н.э.). Оба учения послужили самоутверждению варны кшатриев (затем и вайшьев). Оба учения
имели основателей. Основателем буддизма стал Сиддхарта Гаутама Шакьямуни, ставший под деревом Бодхи (познания)
Просветленным (=Буддой). Легенда повествует о встречах и диспутах \ Гаутамы и Джайны. Из многих школ наиболее известны
Махаяна (Большой путь) и Хинаяна (Малый путь). Буддизм был почти полностью мирно вытеснен из Индии, но не угас, а стал
одной из {\textquotedbl}мировых религий{\textquotedbl}. Об одной из ее ветвей — дзен — побеседуем позднее.}

\subsection[Теория познания]{\selectlanguage{russian} Теория познания}
{\selectlanguage{russian}
Приняв ведическую концепцию \textit{риты} — независимой от богов носительницы морального и физического порядка, —
буддисты заменили Колесо Брахмана Колесом Дхармы (= \textit{дхаммы}). }

{\selectlanguage{russian}\bfseries
Нападки на брахманизм – запрещены.}

{\selectlanguage{russian}
Отвергнуты пять бесполезных вопросов:}

\liststyleWWviiiNumvi
\begin{enumerate}
\item {\selectlanguage{russian}
Мир вечен или нет?}
\item {\selectlanguage{russian}
Мир конечен или нет?}
\item {\selectlanguage{russian}
Душа и тело – совпадают или нет?}
\item {\selectlanguage{russian}
Познавший истину бессмертен или нет?}
\item {\selectlanguage{russian}
Познавший истину одновременно смертен и бессмертен?}
\end{enumerate}
{\selectlanguage{russian}\itshape
«О чем невозможно говорить, о том следует молчать».}

{\selectlanguage{russian}
Все события ума и тела — пустой процесс, идущий сам собой, за ним никого и ничего нет. }

{\selectlanguage{russian}
Все вещи изменяются, они подвержены разложению и исчезновению, они могут существовать только одно мгновение, оставаясь
самими собой. Жизнь есть поток сознания, так что никакой души нет. Человек состоит из пяти групп элементов: формы,
чувств, восприятия, прошлого опыта, сознания самого себя. В буддийской философии, исследовавшей концепцию пустоты
(\textit{шунья}), возникла идея нуля. }

\subsection[Учение]{\selectlanguage{russian} Учение}
{\selectlanguage{russian}
Мудрость не есть знание, не есть сила, мудрость — это гармония. Начальная практика медитации приводит постепенно к тому,
что вся жизнь становится медитацией.}

{\selectlanguage{russian}
Будда признал аскезу несостоятельной и отверг ее, посвятив себя \ размышлению: «Я не сдвинусь с места, пока не достигну
высшего и абсолютного знания». Будде познал \textit{четыре благородные истины}:}

\liststyleWWviiiNumx
\begin{enumerate}
\item {\selectlanguage{russian}
Жизнь полна страданий (и кратких и ненадежных радостей).}
\item {\selectlanguage{russian}
Рождение — источник страданий, рождение имеет источником стремление к становлению.}
\item {\selectlanguage{russian}
Освобождение от страданий — в деятельности без привязанности, без ненависти, без ослепления.}
\item {\selectlanguage{russian}
Путь к освобождению – восьмеричный благородный путь.}
\end{enumerate}
{\selectlanguage{russian}
\textit{Восьмеричный благородный путь} к освобождению от страданий:}

\liststyleWWviiiNumvii
\begin{enumerate}
\item {\selectlanguage{russian}
Правильные взгляды = правильное понимание четырех благородных истин.}
\item {\selectlanguage{russian}
Правильная решимость отказаться от привязанности к миру, от дурных намерений, от вражды.}
\item {\selectlanguage{russian}
Правильная речь — без лжи, без клеветы, без жестоких слов, без фривольности.}
\item {\selectlanguage{russian}
Правильные действия — не причинять вред, не воровать.}
\item {\selectlanguage{russian}
Правильный образ жизни — честно работать.}
\item {\selectlanguage{russian}
Правильное усилие — отвергать старые и новые вредные идеи, набирать и закреплять хорошие идеи.}
\item {\selectlanguage{russian}
Правильное направление мысли — постоянно помнить о бренности тела и вещей.}
\item {\selectlanguage{russian}
Правильное сосредоточение чистого ума на осмыслении и исследовании истин, на правильном осознании радости и покоя (в
невозможности размышления), на правильном безразличии в отрешении от роста радости; на достижении абсолютной
невозмутимости, абсолютного самообладания, абсолютного безразличия, на достижении состояния без страданий и без
освобождения.}
\end{enumerate}
{\selectlanguage{russian}
Колесо бытия состоит из 12 звеньев: 1) страдание, старость и смерть; 2) повторное рождение; 3)~стремление к бытию; 4)
привязанность; 5) жажда наслаждений; 6) чувственный опыт; 7)~чувственные
стремления;~8)~ощущения~органов~чувств~и~ума;~9)~тело~и~разум; 10) сознание; 11) впечатления прошлой жизни;
12)~неведение истины.}

{\selectlanguage{russian}
В одном шаге от \textbf{нирваны} \textit{просветленный} останавливается и становится \textit{бодхисаттвой} – посвященным
помощи другим людям (только в ветви махаяна). }

{\selectlanguage{russian}
\textit{Понимание учения} в буддизме сродни \textit{пониманию музыки}.}

{\selectlanguage{russian}
В современном буддизме женщина может быть не только проповедником, но и главой общины.}

\subsection[Этика]{\selectlanguage{russian} Этика}
{\selectlanguage{russian}
Из благородного молчания Будды возникла основная нравственная концепция буддизма — сострадание, в целостности своей —
любовь. На смену индуистскому общему культу чувств в буддизме приходит культ любви и альтруизма.}

{\selectlanguage{russian}
Этика буддизма разделяет монахов (бхикшу) и мирян, предоставляя монахам возможность бегства из мира, в котором они
оказались лишними, а мирян оставляет в мире: они подают монахам милостыню. Буддийские монастыри, мужские и женские,
построенные мирянами вокруг захоронений просветленных бхикшу, стали центрами (нерелигиозного) образования и искусства.}

\subsection[Картина мира]{\selectlanguage{russian} Картина мира}
{\selectlanguage{russian}
Миров много, как песчинок в Ганге. Миры сгруппированы в мировые системы.\footnote{\foreignlanguage{russian}{\ Широта
взгляда, на днях достигнутая на Западе.}} Каждый Мир – это плоский диск. В центре Мира находится гора, окруженная
кольцами горных хребтов. Наш Мир разделен на четыре континента, главный из них – Индостан, только там появляются
\textit{будды}.}

{\selectlanguage{russian}
Кармический мир размещен на небе, на земле, под землей. На небе — сферы демонов и богов, на земле — сферы животных и
людей, под землей — сферы обитателей ада и прожорливых духов. В сфере богов обитают бодхисатвы. После каждого периода
господства дхармы мир погружается во тьму \textit{авидьи} — до появления следующего будды.}

{\selectlanguage{russian}
Будда может создать мир усилием ума, может летать, стать невидимым. }

\subsection[Тексты]{\selectlanguage{russian} Тексты}
{\selectlanguage{russian}
Канонизированные тексты — \textit{трипитака} — находятся в трех корзинах (на пальмовых листьях): в \textit{питаке
рассказов}, \textit{питаке правил поведения} и \textit{питаке учения} (=философии).}

{\selectlanguage{russian}
Одна из частей питаки рассказов — \textbf{Дхаммапада} — предназначена нетерпеливым как краткое изложение учения.}

{\selectlanguage{russian}
Несмотря на поучение Будды — «сущность не в словах, а в переживаниях» — буддисты создали и продолжают создавать тексты,
воздействующие на миллионы людей. В рамках \textit{махаяны} созданы тексты на санскрите, в рамках \ \textit{хинаяны} –
на просторечном пали.}

\subsection[Эдикты Ашоки]{\selectlanguage{russian} Эдикты Ашоки}
{\selectlanguage{russian}
Ашока (середина III в. до н.э.), один из властителей династии Маурья, завоевал почти весь Индостана, погубил сотни тысяч
людей. После победы обратился в буддизм и оказал ему государственную поддержку. Единственный раскаявшийся воитель,
Ашока приказал высечь на камне \textit{дхаммапили} — надписи о дхамме. Их публично читали нараспев каждые четыре месяца
\ в году, в 12-летнем цикле. Они содержат наставления о почитании отца и матери, об ахимсе, о добром отношении к родным
и близким. Чтение было ритуалом, который обеспечивал ход времени. Все значение ритуала проясняется в свете того факта,
что пространство и время составляли одну сущность.}

\section[7.
\textcyrillic{Л}\textcyrillic{о}\textcyrillic{к}\textcyrillic{а}\textcyrillic{я}\textcyrillic{т}\textcyrillic{а}
]{\foreignlanguage{russian}{7. Локаята }}
{\selectlanguage{russian}
В упанишадах упомянут Брихаспати — легендарный основатель \textit{локаяты}, возвестивший это учение \textit{асурам}
(враждебным аборигенам Индии), чтобы погубить их. Тексты локаятиков не сохранились, локаята реконструируется из цитат
ее оппонентов. Локаята = воззрение народа (\textit{лока}). }

{\selectlanguage{russian}
Тело — основа сущего, душа — иллюзия. Никакого существования после смерти нет. Нет дживы, нет бессмертия души. Смысл
жизни — в наслаждении жизнью. Авторитет, откровения, каноны \ бесполезны. Источником реального знания о мире служат
чувственные восприятия и внутреннее осознание. Исполнение правил варн, ашрамов и всего, о чем говорят брахманы, —
бесплодно. Нет кармы. Знатоки вед — мошенники. }

{\selectlanguage{russian}
Умеренность есть добродетель, но — долой аскезу! }

{\selectlanguage{russian}
На локаяту часто наклеивают ярлык с надписью {\textquotedbl}материализм{\textquotedbl}.}

\section[\ 8. \textcyrillic{И}\textcyrillic{н}\textcyrillic{д}\textcyrillic{и}\textcyrillic{я} \textcyrillic{и}
\textcyrillic{З}\textcyrillic{а}\textcyrillic{п}\textcyrillic{а}\textcyrillic{д} ]{\foreignlanguage{russian}{\ 8. Индия
и Запад }}
{\selectlanguage{russian}
Западное лекало \textit{религии} практически неприложимо в Индии.\footnote{\foreignlanguage{russian}{\ На Западе
невозможен общепринятый в Индии эвфемизм: дорога богов = задний проход.}} Западный шаблон \textit{философии} приходится
применять за неимением другого инструмента. На Западе интеллектуалы отвернулись от религии, создавая новое мышление, на
Востоке новое мышление всегда создано внутри традиционной культуры. Литература Запада есть результат диффузии (и
прямого копирования) литературы Индии.}

{\selectlanguage{russian}\itshape
Основные ценности индуизма: кама (чувственное наслаждение), артха (благополучие), дхарма (нравственный закон), мокша
(внутренняя свобода). Какие из них можно найти на Западе?}

{\selectlanguage{russian}\itshape
В Индии немыслимо распятие Просветителя, потерпевшего неудачу в проповеди истины, рядом с вором; немыслимо прославление
Неудачника, пошедшего на смерть со страховым полисом на воскрешение и вечную славу; немыслимо празднование годовщины
изгнания Пророка.}

{\selectlanguage{russian}
В Индии ведущей в культуре является группа (коллектив, семья, род, клан, каста, варна, племя, нация, класс, государство
и так далее), на Западе – индивидуум (личность, творческая индивидуальность, \textit{я}, мыслитель, вождь, лидер,
властитель, пророк, сын Бога и сам Бог). На Востоке человек испытывает \textbf{\textit{стыд}} за проступки другого
человека из группы, группы или ее части, на Западе чувство вины локализовано в личной
\textbf{\textit{совести}}\textit{.} \textit{Грех} – категория западной (религиозной) этики, равно как и (неизбывное)
чувство вины перед Богом – самым большим начальником. Прощение греха – {\textquotedbl}моральная основа{\textquotedbl}
западной безнравственности. }

{\selectlanguage{russian}
Самостоятельная личность – основа и результат западной ветви развития, символ ее успехов и деградации. Человек на
Востоке – не свободный западный критик, требующий защиты от критикуемого правопорядка, а (невыделенный) элемент
общества. Общество в Индии сохраняет единство, имея (признанных) 12 религий.}

{\selectlanguage{russian}
Сравнивающим школы и учителей отвечали – «ищите своего», чужие взгляды не служили источником \ существования, не было
профессиональных \ борцов с ересями, а на Западе до сих пор ищут самого большого карлика — Гулливера в стране
лилипутов. Восточные учителя — \textit{гуру} — далеко не столь социально опасны, как западные основоположники и фюреры,
так как гуру действуют в рамках совершенно иной культуры, не находящейся на грани взрыва под давлением заключенных в
нее атомизированных индивидуумов – атомов, молекул и \textbf{\textit{радикалов}}.}

{\selectlanguage{russian}
Человек на Востоке решал те же две проблемы бытия, что и его собрат на Западе: куда девать лишнее время и куда девать
лишних людей. Восток осознал и успешно решил куда более важную проблему усмирения {\textquotedbl}скачущей
обезьяны{\textquotedbl} — творческой потенции человека: пагубную творческую активность сознания и вредную
любознательность заняли \textit{медитацией}.\footnote{\foreignlanguage{russian}{\ Медитация – высшее достижение
человека в борьбе с творческой мыслью, с разумом.}} }

{\selectlanguage{russian}\itshape
Сладко спит отрешенный, без надежд, без средств, без цели. Поэтому отвращенья к миру достигает ищущий счастья.}

{\selectlanguage{russian}
Лишних людей дисциплинировали групповой моралью, изолировали в аскезе, в медитации, в отшельничестве и в монастырях –
эффективнее, чем на Западе. Войн в Индии хватало, но патологической европейской и ближневосточной экспансии не было
(зато был ритуал \textit{ашвамедха!)}. Не было и национальной идеи. Недостаток социальной мобильности был компенсирован
духовной мобильностью в безграничном необитаемом пространстве медитирования. }
\end{document}
