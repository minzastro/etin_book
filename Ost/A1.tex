% This file was converted to LaTeX by Writer2LaTeX ver. 1.4
% see http://writer2latex.sourceforge.net for more info
\documentclass[a4paper]{article}
\usepackage[utf8]{inputenc}
\usepackage[T2A,T1]{fontenc}
\usepackage[russian,english]{babel}
\usepackage{amsmath}
\usepackage{amssymb,amsfonts,textcomp}
\usepackage{color}
\usepackage{array}
\usepackage{hhline}
\usepackage{hyperref}
\hypersetup{pdftex, colorlinks=true, linkcolor=blue, citecolor=blue, filecolor=blue, urlcolor=blue, pdftitle=\textcyrillic{Лао Цзы}}
% Page layout (geometry)
\setlength\voffset{-1in}
\setlength\hoffset{-1in}
\setlength\topmargin{0.3937in}
\setlength\oddsidemargin{0.7874in}
\setlength\textheight{10.1945in}
\setlength\textwidth{6.8897996in}
\setlength\footskip{0.35549998in}
\setlength\headheight{0.1972in}
\setlength\headsep{0.1583in}
% Footnote rule
\setlength{\skip\footins}{0.0469in}
\renewcommand\footnoterule{\vspace*{-0.0071in}\setlength\leftskip{0pt}\setlength\rightskip{0pt plus 1fil}\noindent\textcolor{black}{\rule{0.25\columnwidth}{0.0071in}}\vspace*{0.0398in}}
% Pages styles
\makeatletter
\newcommand\ps@Standard{
  \renewcommand\@oddhead{\textstylePageNumber{\textcyrillic{Восток }\foreignlanguage{russian}{6}}}
  \renewcommand\@evenhead{\@oddhead}
  \renewcommand\@oddfoot{[Warning: Draw object ignored]}
  \renewcommand\@evenfoot{\@oddfoot}
  \renewcommand\thepage{\arabic{page}}
}
\makeatother
\pagestyle{Standard}
\title{\textcyrillic{Лао Цзы}}
\begin{document}
\clearpage\setcounter{page}{1}\pagestyle{Standard}

\bigskip

{\centering\selectlanguage{russian}\bfseries
Лао-цзы 
\par}


\bigskip

{\centering\selectlanguage{russian}\bfseries
Дао-дэ цзин
\par}


\bigskip

{\centering\selectlanguage{english}\ttfamily
\foreignlanguage{russian}{\textrm{\textbf{* Дао цзин *}}}
\par}


\bigskip

{\selectlanguage{english}\ttfamily
\foreignlanguage{russian}{\textrm{\textbf{1}}}\foreignlanguage{russian}{\textrm{ Дао, которое может быть выражено
словами, не есть истинное дао.}}}

{\selectlanguage{russian}
Имя, которое может быть названо, не есть истинное имя.}

{\selectlanguage{russian}
Безымянное есть начало неба и земли, обладающее именем мать всех вещей.}

{\selectlanguage{russian}
Поэтому, кто свободен от страстей, видит его [дао] чудесную [тайну], а кто имеет страсти, видит его только в конечной
форме.}

{\selectlanguage{russian}
Оба они одного и того же происхождения, но с разными названиями. Вместе они называются глубочайшими. От одного
глубочайшего к другому дверь ко всему чудесному.}

{\selectlanguage{english}\ttfamily
\foreignlanguage{russian}{\textrm{\textbf{2}}}\foreignlanguage{russian}{\textrm{ Когда люди узнают, что красивое
является красивым, появляется и безобразное.}}}

{\selectlanguage{russian}
Когда [все] узнают, что добро является добром, возникает и зло.}

{\selectlanguage{russian}
Поэтому бытие и небытие порождают друг друга, трудное и легкое создают друг друга, длинное и короткое взаимно
оформляются, высокое и низкое друг к другу склоняются, звуки, сливаясь, приходят в гармонию, предыдущее и последующее
следуют друг за другом.}

{\selectlanguage{russian}
Поэтому мудрый человек предпочитает недеяние и осуществляет учение безмолвно. Тогда все вещи приходят в движение, и они
не останавливают [своего движения]. Он создает и не обладает [тем, что создано], делает и не пользуется [тем, что
сделано]; поскольку он не гордится, [все существа его] не покидают.}

{\selectlanguage{english}\ttfamily
\foreignlanguage{russian}{\textrm{\textbf{3}}}\foreignlanguage{russian}{\textrm{ Если не уважать мудрецов, то в народе
не будет ссор. Если не ценить драгоценных предметов, то не будет воров среди народа. Если не видеть желаемого предмета,
то не будут волноваться сердца народа.}}}

{\selectlanguage{russian}
Поэтому управление мудрого человека делает их сердца пустыми, а желудки полными. Оно ослабляет их волю и укрепляет их
кости. Оно постоянно стремится к тому, чтобы у народа не было знаний и страстей, а имеющие знания не смели бы
действовать.}

{\selectlanguage{russian}
Осуществление недеяния [мудрецом] всегда приносит спокойствие.}

{\selectlanguage{english}\ttfamily
\foreignlanguage{russian}{\textrm{\textbf{4}}}\foreignlanguage{russian}{\textrm{ Дао пусто, но, действуя, оно кажется
неисчерпаемым.}}}

{\selectlanguage{russian}
О, глубочайшее! Оно кажется праотцом всех вещей.}

{\selectlanguage{russian}
Если притупить его проницательность, освободить его от хаотичности, умерить его блеск, уподобить его пылинке, то оно
будет казаться ясно существующим.}

{\selectlanguage{russian}
Я не знаю, чье оно порождение. Оно предшествует предку явлений.}

{\selectlanguage{english}\ttfamily
\foreignlanguage{russian}{\textrm{\textbf{5}}}\foreignlanguage{russian}{\textrm{ Небо и земля не обладают гуманностью, и
они относятся ко всем существам, как к траве и животным.}}}

{\selectlanguage{russian}
Мудрый человек не обладает гуманностью и не нарушает естественную жизнь народа.}

{\selectlanguage{russian}
Пространство между небом и землей подобно кузнечному меху и флейте: [то и другое] изнутри пусто и прямо. Чем [в нем]
сильнее движение, тем больше результатов.}

{\selectlanguage{russian}
Тот, кто много говорит, часто терпит неудачу. Поэтому лучше соблюдать середину [меру].}

{\selectlanguage{english}\ttfamily
\foreignlanguage{russian}{\textrm{\textbf{6}}}\foreignlanguage{russian}{\textrm{ Пустота [дао] бессмертна, и [я] называю
ее глубочайшим началом. Вход в глубочайшее начало зову корнем неба и земли. [Оно] бесконечно, как существование, и
действует без усилия.}}}

{\selectlanguage{english}\ttfamily
\foreignlanguage{russian}{\textrm{\textbf{7}}}\foreignlanguage{russian}{\textrm{ Небо и земля долговечны. Небо и земля
долговечны потому, что они существуют не для себя. Вот почему они могут быть долговечными.}}}

{\selectlanguage{russian}
Поэтому мудрый человек ставит себя позади других, благодаря чему он оказывается впереди людей. Он пренебрегает своей
жизнью, и тем самым его жизнь сохраняется. Это происходит оттого, что мудрец пренебрегает личными [интересами], и тем
самым его личные [интересы] осуществляются.}

{\selectlanguage{english}\ttfamily
\foreignlanguage{russian}{\textrm{\textbf{8}}}\foreignlanguage{russian}{\textrm{ Высшая добродетель подобна воде. Вода
приносит пользу всем существам и не борется. Она находится там, где люди не желали ли бы быть. Поэтому она похожа на
дао.}}}

{\selectlanguage{russian}
Жизнь должна следовать [законам] земли; сердце должно следовать [законам] внутренних побуждений; благотворительность
должна соответствовать гуманности; слово должно соответствовать истине; управление [страной] должно соответствовать
спокойствию; дело должно соответствовать возможностям; действие должно соответствовать времени.}

{\selectlanguage{russian}
И если не бороться, то не будет ропота [среди народа].}

{\selectlanguage{english}\ttfamily
\foreignlanguage{russian}{\textrm{\textbf{9}}}\foreignlanguage{russian}{\textrm{ Кто наполняет [сосуд] через край и
оттачивает [лезвие] слишком остро, тот не может их долго сохранить.}}}

{\selectlanguage{russian}
Если зал наполнен золотом и яшмой, то никто не в силах его охранять.}

{\selectlanguage{russian}
Если богатые и знатные горды, то этим они сами на себя навлекают несчастья.}

{\selectlanguage{russian}
Если дело завершено, человек [должен] устраниться. В этом закон естественного дао.}

{\selectlanguage{english}\ttfamily
\foreignlanguage{russian}{\textrm{\textbf{10}}}\foreignlanguage{russian}{\textrm{ Для сохранения спокойствия души
[человек] должен соблюдать единство. Тогда [в нем] не будут пробуждаться желания. Если сделать дух мягким, человек
станет подобен новорожденному. Если его созерцание становится чистым, то не будет заблуждений.}}}

{\selectlanguage{russian}
Любовь к народу и управление страной осуществляются без умствования.}

{\selectlanguage{russian}
Врата мира открываются и закрываются при соблюдении спокойствия. Знание этой истины делает возможным недеяние.}

{\selectlanguage{russian}
Рождать [существа] и воспитывать [их], создавать и не обладать [тем, что создано], творить и не воспользоваться [тем,
что сделано], будучи старшим среди других, не считать себя властелином – все это называется глубочайшим дэ.}

{\selectlanguage{english}\ttfamily
\foreignlanguage{russian}{\textrm{\textbf{11}}}\foreignlanguage{russian}{\textrm{ Тридцать спиц соединяются в одной
ступице [образуя колесо], а употребление колеса зависит от пустоты между спицами. Из глины делают сосуды, а
употребление сосудов зависит от пустоты в них. Пробивают двери и окна, чтобы сделать дом, а пользование домом зависит
от пустоты в нем. Вот что значит полезность бытия и пригодность небытия.}}}

{\selectlanguage{english}\ttfamily
\foreignlanguage{russian}{\textrm{\textbf{12}}}\foreignlanguage{russian}{\textrm{ Пять цветов притупляют зрение. Пять
звуков притупляют слух. Пять вкусовых ощущений притупляют вкус. Быстрая езда и охота волнуют сердце.}}}

{\selectlanguage{russian}
Драгоценные вещи заставляют человека совершать преступления. Поэтому усилия мудрого человека направлены к тому, чтобы
сделать жизнь сытой, а не к тому, чтобы иметь красивые вещи. Он отказывается от последних и ограничивается первым.}

{\selectlanguage{english}\ttfamily
\foreignlanguage{russian}{\textrm{\textbf{13}}}\foreignlanguage{russian}{\textrm{ Слава и позор подобны страху.
Знатность подобна великому несчастию в жизни. Что значит слава и позор подобны страху? Это значит, что нижестоящие люди
приобретают славу со страхом и теряют ее тоже со страхом. Что значит знатность подобна великому несчастию в жизни? Это
значит, что я имею великое несчастье, потому что я дорожу самим собой. Когда я не буду дорожить самим собой, тогда у
меня не будет и несчастья. Поэтому знатный, самоотверженно служа людям, может жить среди них. Гуманный, самоотверженно
служа людям, может находиться среди них.}}}

{\selectlanguage{english}\ttfamily
\foreignlanguage{russian}{\textrm{\textbf{14}}}\foreignlanguage{russian}{\textrm{ Смотрю на него и не вижу, а поэтому
называю его невидимым. Слушаю его и не слышу, поэтому называю его неслышимым. Пытаюсь схватить его и не достигаю,
поэтому называю его мельчайшим. Эти три [качества дао] необъяснимы. Поэтому они сливаются воедино. Его верх не освещен,
его низ не затемнен. Оно бесконечно и не может быть названо. Оно снова возвращается к небытию. И вот называю его формой
без форм, образом без существа. Поэтому называю его неясным и туманным. Встречаюсь с ним и не вижу лица его, следую за
ним и не вижу спины его. Держась древнего дао и овладевая теперешним бытием, можно познать древнее начало. Это
называется нитью дао.}}}

{\selectlanguage{english}\ttfamily
\foreignlanguage{russian}{\textrm{\textbf{15}}}\foreignlanguage{russian}{\textrm{ В древности тот, кто был способен к
просвещению, знал мельчайшие [вещи] и глубокую [тайну]. [Но они были] скрытными, поэтому их нельзя было узнать.
Поскольку нельзя было их узнать, произвольно даю [им] образ: они были робкими, как будто переходили зимой поток; они
были нерешительными, как будто боялись своих соседей; они были важными, как гости; они были осторожными, как будто
переходили по тающему льду; они были простыми, подобно необделанному дереву; они были необъятными, подобно долине; они
были непроницаемыми, подобно мутной воде. Это были те, которые, соблюдая спокойствие, умели грязное сделать чистым. Это
были те, которые своим умением сделать долговечное движение спокойным содействовали жизни. Они соблюдали дао и не
желали многого. Не желая многого, они ограничивались тем, что существует, и не создали нового.}}}

{\selectlanguage{english}\ttfamily
\foreignlanguage{russian}{\textrm{\textbf{16}}}\foreignlanguage{russian}{\textrm{ Доведу пустоту [своего сердца] до
конца, сохраню полный покой, и тогда все вещи сами будут расти, а я буду ждать их возвращения. Все вещи расцветают и
возвращаются к своему началу. Возвращение к началу называется покоем, а покой называется возвращением к жизни.
Возвращение к жизни называется постоянством. Знание постоянства называется просвещением, а незнание постоянства
совершает зло. Знающий постоянство становится мудрым, мудрый становится справедливым, а кто справедлив, становится
государем.}}}

{\selectlanguage{russian}
Государь следует небу, небо следует дао, а дао вечно. До конца жизни [у такого государя] не будет опасности.}

{\selectlanguage{english}\ttfamily
\foreignlanguage{russian}{\textrm{\textbf{17}}}\foreignlanguage{russian}{\textrm{ Простые люди знали, что они имели
великих людей. Они их любили и возвышали. Затем они их боялись и презирали. Поэтому, кто не заслуживает доверия, тот не
пользуется доверием [у людей]. Кто вдумчив и сдержан в словах, тот приобретает заслуги и совершает дела, и народ
говорит, что он следует естественности.}}}

{\selectlanguage{english}\ttfamily
\foreignlanguage{russian}{\textrm{\textbf{18}}}\foreignlanguage{russian}{\textrm{ Когда устранили великое дао, появилась
гуманность и справедливость.}}}

{\selectlanguage{russian}
Когда появилось мудрствование, возникло и великое лицемерие. Когда шесть родственников в раздоре, тогда появляется
сыновняя почтительность и отцовская любовь. Когда в государстве царит беспорядок, тогда появляются верные слуги.}

{\selectlanguage{english}\ttfamily
\foreignlanguage{russian}{\textrm{\textbf{19}}}\foreignlanguage{russian}{\textrm{ Когда будут устранены мудрствование и
ученость, тогда народ будет счастливее во сто крат; }}\foreignlanguage{russian}{\textrm{когда будут устранены
гуманность и справедливость, тогда народ возвратится к сыновней почтительности и отцовской любви; когда будут
уничтожены хитрость и нажива, тогда исчезнут воры и разбойники. Все эти три вещи [происходят] от недостатка знаний.
Поэтому нужно указывать людям, что они должны быть простыми и скромными, уменьшать личные [желания] и освободиться от
страстей.}}}

{\selectlanguage{english}\ttfamily
\foreignlanguage{russian}{\textrm{\textbf{20}}}\foreignlanguage{russian}{\textrm{ Когда будет уничтожена ученость, тогда
не будет и печали. Как ничтожна разница между обещанием и лестью и как велика разница между добром и злом!}}}

{\selectlanguage{russian}
Надо избегать того, чего люди боятся.}

{\selectlanguage{english}\ttfamily
\foreignlanguage{russian}{\textrm{О! Как хаотичен [мир], где все еще не установлен порядок. Все люди радостны, как будто
присутствуют на торжественном угощении или празднуют наступление весны. Только я один спокоен и не выставляю себя на
свет. Я подобен ребенку, который не явился в мир. О! Я несусь! Кажется, нет места, где мог бы остановиться. Все люди
полны желаний, только я один подобен тому, кто отказался от всего. Я сердце глупого человека. О, как все пусто. Все
люди полны света. Только я один подобен тому, кто погружен во мрак. Все люди пытливы, только я один равнодушен. Я
подобен тому, кто несется в морском просторе и не знает, где ему остановиться. Все люди проявляют свою способность, и
только я один похож на глупого и низкого. Только я один отличаюсь от других тем, что ценю источник питания.}}}

{\selectlanguage{english}\ttfamily
\foreignlanguage{russian}{\textrm{\textbf{21}}}\foreignlanguage{russian}{\textrm{ Образы великого дэ подчиняются только
дао. Дао вещь неясная и туманная.}}}

{\selectlanguage{russian}
О, туманное! О, неясное! В нем заключены образы. О, неясное! О, туманное! В нем заключены семена. Его семена совершенно
достоверны, и в нем заключена истина. С древних времен до наших дней его имя не исчезает. Оно существует для
обозначения начала всех вещей. Почему я знаю начало всех вещей? Только благодаря ему.}

{\selectlanguage{english}\ttfamily
\foreignlanguage{russian}{\textrm{\textbf{22}}}\foreignlanguage{russian}{\textrm{ Неполное становится полным; кривое
становится прямым; пустое становится наполненным; ветхое сменяется новым; то, что мало, становится многим.}}}

{\selectlanguage{russian}
Многое вызывает заблуждение. Поэтому мудрый человек сохраняет единство и становится примером для всех. Он не выставляет
себя на свет, поэтому блестит; он не говорит о себе, поэтому он славен; он не прославляет себя, поэтому он заслужен; он
не возвышает себя, поэтому он является старшим среди других. Он не борется, поэтому он непобедим в мире.}

{\selectlanguage{russian}
В древности говорили, что несовершенное становится совершенным. Неужели это пустые слова?}

{\selectlanguage{russian}
Истинное, совершенное подчиняет себе все.}

{\selectlanguage{english}\ttfamily
\foreignlanguage{russian}{\textrm{\textbf{23}}}\foreignlanguage{russian}{\textrm{ Нужно меньше говорить, следовать
естественности. Быстрый ветер не продолжается все утро, сильный дождь не продержится весь день. Кто делает все это?
Небо и земля. Даже небо и земля не могут сделать что-либо долговечным, тем более человек. Поэтому он служит дао.}}}

{\selectlanguage{english}\ttfamily
\foreignlanguage{russian}{\textrm{Человек с дао тождествен дао. Человек с дэ тождествен дэ. Тот, кто потеряет,
тождествен потере. Тот, кто тождествен дао, приобретает дао.}}}

{\selectlanguage{english}\ttfamily
\foreignlanguage{russian}{\textrm{Тот, кто тождествен дэ, приобретает дэ. Тот, кто тождествен потере, приобретает
потерянное. Только сомнения порождают неверие.}}}

{\selectlanguage{english}\ttfamily
\foreignlanguage{russian}{\textrm{\textbf{24}}}\foreignlanguage{russian}{\textrm{ Кто поднялся на цыпочки, не может
[долго] стоять. Кто делает большие шаги, не может [долго] идти. Кто сам себя выставляет на свет, тот не блестит. Кто
сам себя восхваляет, тот не добудет славы. Кто нападает, не достигает успеха. Кто сам себя возвышает, не может стать
старшим среди других. Исходя из дао, все это называется лишним желанием и бесполезным поведением. Таких ненавидят все
существа. Поэтому человек, обладающий дао, не делает этого.}}}

{\selectlanguage{english}\ttfamily
\foreignlanguage{russian}{\textrm{\textbf{25}}}\foreignlanguage{russian}{\textrm{ Вот вещь в хаосе возникающая, прежде
неба и земли родившаяся! О, спокойная! О, пустотная! Одиноко стоит она и не изменяется. Повсюду действует и не
подвергается опасности [уничтожения]. Ее можно считать матерью поднебесной. Я не знаю ее имени. Обозначая знаком,
назову ее дао; произвольно давая ей имя, назову великой. Великая, назову ее преходящей. Преходящая, назову ее далекой.
Далекая, назову ее возвращающейся. Вот почему велико дао, велико небо, велика земля, велик также и государь. Во
вселенной имеются четыре великих, и среди них находится государь.}}}

{\selectlanguage{russian}
Человек следует земле. Земля следует небу. Небо следует дао, а дао следует естественности.}

{\selectlanguage{english}\ttfamily
\foreignlanguage{russian}{\textrm{\textbf{26}}}\foreignlanguage{russian}{\textrm{ Тяжелое является основой легкого.
Покой есть главное в движении. Поэтому мудрый человек действует весь день, не оставляя тяжелого дела. Хотя он питает
блестящую надежду, но находится в совершенно спокойном состоянии.}}}

{\selectlanguage{russian}
Напрасно властитель десяти тысяч колесниц, занятый собой, так легкомысленно смотрит на мир. Легкомыслие разрушает его
основу, а его торопливость приводит к потере опоры.}


\bigskip

{\selectlanguage{english}\ttfamily
\foreignlanguage{russian}{\textrm{\textbf{27}}}\foreignlanguage{russian}{\textrm{ Умеющий шагать не оставляет следов.
Умеющий говорить не допускает ошибок. Кто умеет считать, тот не пользуется счетом. Кто умеет закрывать двери, тот не
употребляет запор и закрывает их так крепко, что открыть их невозможно. Кто умеет завязывать узлы, тот не употребляет
веревку, и завязывает так прочно, что развязать невозможно. Поэтому мудрый человек постоянно умело спасает людей и их
не покидает. Он всегда умеет спасать существа, поэтому он их не покидает. Это называется глубоким просвещением.}}}

{\selectlanguage{russian}
Таким образом, добродетель является учителем недобрых, а недобрые его опорой. Если [недобрые] не ценят своего учителя и
добродетель не любит свою опору, то они хотя [считают себя] разумными, [на деле] погружены в слепоту.}

{\selectlanguage{russian}
Вот что наиболее важно и глубоко.}

{\selectlanguage{english}\ttfamily
\foreignlanguage{russian}{\textrm{\textbf{28}}}\foreignlanguage{russian}{\textrm{ Кто, зная свою храбрость, сохраняет
скромность, тот [подобно] горному ручью, становится [главным] в стране. Кто стал главным в стране, тот не покидает
постоянное дэ и возвращается к состоянию младенца. Кто, зная праздничное, сохраняет для себя будничное, тот становится
примером для всех.}}}

{\selectlanguage{russian}
Кто стал примером для всех, тот не отличается от постоянного дэ и возвращается к безначальному. Кто, зная свою славу,
сохраняет для себя безвестность, тот становится главным в стране. Кто стал главным в стране, тот достигает совершенства
в постоянном дэ и возвращается к естественности.}

{\selectlanguage{russian}
Когда естественность распадается, она превращается в средство, при помощи которого мудрый становится вождем, и его
великий порядок не разрушается.}

{\selectlanguage{english}\ttfamily
\foreignlanguage{russian}{\textrm{\textbf{29}}}\foreignlanguage{russian}{\textrm{ Если кто-нибудь силой пытается
овладеть страной, то, я вижу, он не достигает своей цели. Страна подобна таинственному сосуду, к которому нельзя
прикоснуться. Если кто-нибудь тронет [его], потерпит неудачу. Если кто-нибудь схватит [его], то его потеряет.}}}

{\selectlanguage{russian}
Поэтому одни существа идут, другие следуют за ними; одни расцветают, другие высыхают; одни укрепляются, другие слабеют;
одни создаются, другие разрушаются. Поэтому мудрый человек отказывается от излишеств, устраняет роскошь и
расточительность.}

{\selectlanguage{english}\ttfamily
\foreignlanguage{russian}{\textrm{\textbf{30}}}\foreignlanguage{russian}{\textrm{ Кто служит главе народа посредством
дао, не покоряет другие страны при помощи войск, ибо это может обратиться против него. Где побывали войска, там растут
терновник и колючки. После больших войн наступают голодные годы.}}}

{\selectlanguage{russian}
Искусный [полководец] побеждает и на этом останавливается, и он не осмеливается осуществлять насилие. Он побеждает и
себя не прославляет. Он побеждает и не нападает. Он побеждает и не гордится. Он побеждает потому, что к этому его
вынуждают. Он побеждает, но он не воинственен.}

{\selectlanguage{russian}
Когда существо, полное сил, становится старым, то это называется отсутствием дао. Кто не соблюдает дао, погибнет раньше
времени.}

{\selectlanguage{english}\ttfamily
\foreignlanguage{russian}{\textrm{\textbf{31}}}\foreignlanguage{russian}{\textrm{ Хорошее войско – средство, порождающее
несчастье, его ненавидят все существа. Поэтому человек, следующий дао, его не употребляет.}}}

{\selectlanguage{russian}
Благородный, во время мира, предпочитает уважение, а на войне применяет насилие. Войско – орудие несчастья, оно не
является орудием благородного. Он употребляет его только тогда, когда к этому его вынуждают. Главное состоит в том,
чтобы соблюдать спокойствие, а в случае победы себя не прославлять.}

{\selectlanguage{russian}
Прославлять себя победой это значит радоваться убийству людей. Тот кто радуется убийству людей, не может завоевать
сочувствия в стране.}

{\selectlanguage{russian}
Благополучие создается уважением, а несчастье происходит от насилия.}

{\selectlanguage{russian}
Слева строятся военачальники флангов, справа стоит полководец. Говорят, их нужно встретить похоронной церемонией. Если
убивают многих людей, то об этом нужно горько плакать. Победу следует отмечать похоронной церемонией.}

{\selectlanguage{english}\ttfamily
\foreignlanguage{russian}{\textrm{\textbf{32}}}\foreignlanguage{russian}{\textrm{ Дао вечно и не имеет имени. Хотя оно
существо маленькое, но в мире никто не может его себе подчинить. Если знать и государи могут его соблюдать, то все
существа сами становятся спокойными. Тогда небо и земля сольются в гармонии, наступят счастье и благополучие, а народ
без приказания успокоится.}}}

{\selectlanguage{russian}
При установлении порядка появляются имена. Поскольку возникли имена, нужно знать предел. Знание предела дает возможность
избавиться от опасности.}

{\selectlanguage{russian}
Дао, находясь в мире, похоже на горные ручьи, которые текут к рекам и морям.}

{\selectlanguage{english}\ttfamily
\foreignlanguage{russian}{\textrm{\textbf{33}}}\foreignlanguage{russian}{\textrm{ Тот, кто знает людей, благоразумен.
Знающий себя просвещен. Побеждающий людей силен. Побеждающий самого себя могущественен. Знающий достаток богат.}}}

{\selectlanguage{russian}
Кто действует с упорством, обладает волей. Кто не теряет свою природу, долговечен. Кто умер, но не забыт, тот
бессмертен.}

{\selectlanguage{english}\ttfamily
\foreignlanguage{russian}{\textrm{\textbf{34}}}\foreignlanguage{russian}{\textrm{ Великое дао растекается повсюду. Оно
может быть направо и налево.}}}

{\selectlanguage{english}\ttfamily
\foreignlanguage{russian}{\textrm{Благодаря нему рождаются все существа, и они не останавливаются [в своем росте]. Оно
совершает подвиги, но хвалы себе не желает. С любовью воспитывая все существа, оно не становится их господином. Оно
никогда не имеет своего желания, поэтому его можно назвать маленьким [скромным]. Все существа возвращаются к нему, и
оно не рассматривает себя как господина. Его можно назвать великим. Оно становится великим благодаря тому, что никогда
не считает себя таковым.}}}

{\selectlanguage{english}\ttfamily
\foreignlanguage{russian}{\textrm{\textbf{35}}}\foreignlanguage{russian}{\textrm{ К тому, кто представляет собой великий
образ [дао], приходит весь народ. Люди приходят, и он им не причиняет вреда. Он приносит им мир, спокойствие, музыку и
пищу. Даже путешественник у него останавливается.}}}

{\selectlanguage{russian}
Когда дао выходит изо рта, оно пресное, безвкусное. Оно незримо и его нельзя услышать. В действии оно неисчерпаемо.}


\bigskip

{\selectlanguage{english}\ttfamily
\foreignlanguage{russian}{\textrm{\textbf{36}}}\foreignlanguage{russian}{\textrm{ То, что сжимают, расширяется. То, что
ослабляют, укрепляется. То, что уничтожают — расцветает. Кто хочет отнять что-нибудь у другого, непременно потеряет
свое. Все это называется трудно постижимым. Мягкое преодолевает твердое, слабые побеждают сильных. Рыба не может
покинуть глубину. Острое оружие в государстве нельзя показывать людям.}}}

{\selectlanguage{english}\ttfamily
\foreignlanguage{russian}{\textrm{\textbf{37}}}\foreignlanguage{russian}{\textrm{ Дао постоянно осуществляет недеяние,
тем самым нет ничего такого, что бы оно ни делало. Если знать и государи будут его соблюдать, то все существа будут
изменяться сами собой. Если те, которые изменяются, захотят действовать, то я буду подавлять их при помощи простого
бытия, не обладающего именем. Не обладающее именем простое бытие для себя ничего не желает. Отсутствие желания приносит
покой, и тогда порядок в стране сам собой установится.}}}


\bigskip

{\centering\selectlanguage{english}\ttfamily
\foreignlanguage{russian}{\textrm{\textbf{* Дэ цзин *}}}
\par}


\bigskip

{\selectlanguage{english}\ttfamily
\foreignlanguage{russian}{\textrm{\textbf{38}}}\foreignlanguage{russian}{\textrm{ Человек с высшим дэ не осуществляет
добрые дела, поэтому он является добродетельным; человек с низшим дэ не оставляет добрых дел, поэтому он не является
добродетельным; человек с высшим дэ бездеятелен и действует посредством недеяния; человек с низшим дэ деятелен и
действует с напряжением; человек высшей гуманности действует, и его деятельность осуществляется посредством недеяния;
человек высшей справедливости деятелен и действует с напряжением; человек высшей почтительности действует, и ему никто
не отвечает. Тогда он принуждает людей к почтению. Вот почему добродетель появляется только после утраты дао,
гуманность после утраты добродетельности, справедливость после утраты гуманности, почтительность после утраты
справедливости.}}}

{\selectlanguage{russian}
Почтительность – признак отсутствия доверия и преданности. Она начало смуты.}

{\selectlanguage{russian}
Внешний вид – это цветок дао, начало невежества. Поэтому великий человек берет существенное и оставляет ничтожное. Он
берет плод и отбрасывает его цветок. Он предпочитает первое и отказывается от второго.}

{\selectlanguage{english}\ttfamily
\foreignlanguage{russian}{\textrm{\textbf{39}}}\foreignlanguage{russian}{\textrm{ Вот те, которые с древних времен
находятся в единстве. Благодаря единству небо стало чистым, земля незыблемой, дух чутким, долина цветущей и начала
рождать все существа. Благодаря единству знать и государи становятся образцом в мире. Вот что создает единство.}}}

{\selectlanguage{english}\ttfamily
\foreignlanguage{russian}{\textrm{Если небо нечисто, оно разрушается; если земля зыбка, она раскалывается; если дух не
чуток, он исчезает; если долины не цветут, они превращаются в пустыню; если вещи не рождаются, они исчезают; если знать
и государи не являются примером благородства, они будут свергнуты.}}}

{\selectlanguage{russian}
Незнатные являются основой для знатных, а низкое – основанием для высокого. Поэтому знать и государи, которые сами себя
возвышают, прочного [положения] не имеют. Это происходит оттого, что они не рассматривают незнатных как свою основу.
Это ложный путь. Если разобрать колесницу, то от нее ничего не останется. Нельзя быть драгоценным, как яшма, а нужно
стать простым, как камень.}

{\selectlanguage{english}\ttfamily
\foreignlanguage{russian}{\textrm{\textbf{40}}}\foreignlanguage{russian}{\textrm{ Противоположность есть действие дао,
слабость есть средство дао. В мире все вещи рождаются в бытии, а бытие рождается в небытии.}}}

{\selectlanguage{english}\ttfamily
\foreignlanguage{russian}{\textrm{\textbf{41}}}\foreignlanguage{russian}{\textrm{ Мудрый человек, узнав о дао, стремится
к его осуществлению.}}}

{\selectlanguage{russian}
Образованный человек, узнав о дао, то сохраняет его, то теряет его. Неуч, узнав о дао, подвергает его насмешке. Если оно
не подвергалось бы насмешке, не являлось бы дао. Поэтому существует поговорка: кто узнает дао, похож на темного; кто
проникает в дао, похож на отступающего; кто на высоте дао, похож на заблуждающегося; человек высшей добродетели похож
на простого; великий просвещенный похож на презираемого; безграничная добродетельность похожа на ее недостаток;
распространение добродетели похоже на ее расхищение; истинная правда похожа на ее отсутствие.}

{\selectlanguage{russian}
Великий квадрат не имеет углов; большой сосуд долго изготовляется; сильный звук нельзя услышать; великий образ не имеет
формы.}

{\selectlanguage{russian}
Дао скрыто [от нас] и не имеет имени. Но оно оказывает помощь [всем существам] и ведет их к совершенству.}

{\selectlanguage{english}\ttfamily
\foreignlanguage{russian}{\textrm{\textbf{42}}}\foreignlanguage{russian}{\textrm{ Дао рождает одно, одно рождает два,
два рождают три, а три все существа. Все существа носят в себе
}}\foreignlanguage{russian}{\textrm{\textit{инь}}}\foreignlanguage{russian}{\textrm{ и
}}\foreignlanguage{russian}{\textrm{\textit{ян}}}\foreignlanguage{russian}{\textrm{, наполнены
}}\foreignlanguage{russian}{\textrm{\textit{ци}}}\foreignlanguage{russian}{\textrm{ и образуют гармонию.}}}

{\selectlanguage{russian}
Люди презирают тех, которые сами себя возвышают и называют себя государями и знатными. Все существа укрепляются после
ослабления и ослабляются после укрепления.}

{\selectlanguage{russian}
Люди распространяют свое учение, тем же занимаюсь и я. Жестокие и тираны не умирают своей смертью. Это я привожу как
пример в своем поучении.}


\bigskip

{\selectlanguage{english}\ttfamily
\foreignlanguage{russian}{\textrm{\textbf{43}}}\foreignlanguage{russian}{\textrm{ В мире самые слабые побеждают самых
сильных. Небытие проникает везде и всюду. Вот почему я знаю пользу от недеяния. В мире нет ничего, что можно было бы
сравнить с учением безмолвия и пользой недеяния.}}}

{\selectlanguage{english}\ttfamily
\foreignlanguage{russian}{\textrm{\textbf{44}}}\foreignlanguage{russian}{\textrm{ Что ближе к себе – слава или жизнь?
Что дороже – жизнь или богатство? Что тяжелее пережить – приобретение или потерю? Кто многое сберегает, тот понесет
большие потери. Кто много накапливает, тот потерпит большие убытки. Кто знает меру, у того не будет неудачи. Кто знает
предел, у того не }}\foreignlanguage{russian}{\textrm{будет опасности. Он становится долговечным.}}}

{\selectlanguage{english}\ttfamily
\foreignlanguage{russian}{\textrm{\textbf{45}}}\foreignlanguage{russian}{\textrm{ Великое совершенство похоже на
несовершенное, его действие бесконечно; великая полнота похожа на пустоту, ее действие неисчерпаемо. Великая прямота
похожа на кривое; великое остроумие похоже на глупость; великий оратор похож на заику.}}}

{\selectlanguage{russian}
Движение побеждает холод, покой побуждает жару. Спокойствие создает порядок в мире.}

{\selectlanguage{english}\ttfamily
\foreignlanguage{russian}{\textrm{\textbf{46}}}\foreignlanguage{russian}{\textrm{ Когда в стране существует дао, лошади
унаваживают землю; когда в стране отсутствует дао, боевые кони пасутся на полях. Нет большего несчастья, чем незнание
границы своей страсти, и нет большей опасности, чем стремление к приобретению [богатств]. Поэтому кто умеет
удовлетворяться, всегда доволен [своей жизнью].}}}

{\selectlanguage{english}\ttfamily
\foreignlanguage{russian}{\textrm{\textbf{47}}}\foreignlanguage{russian}{\textrm{ Не выходя со двора, мудрец познает
мир. Не выглядывая из окна, он видит естественное дао. Чем дальше он идет, тем меньше познает. Поэтому мудрый человек
не ходит, но познает. Не видя [вещей], [он] называет их. Он, не действуя, творит.}}}

{\selectlanguage{english}\ttfamily
\foreignlanguage{russian}{\textrm{\textbf{48}}}\foreignlanguage{russian}{\textrm{ Кто учится, с каждым днем увеличивает
[свои знания]. Кто служит дао, изо дня в день уменьшает [свои желания]. В непрерывном уменьшении [человек] доходит до
недеяния. Нет ничего такого, что бы не делало недеяние. Поэтому завоевание страны всегда осуществляется посредством
недеяния. Кто действует, не в состоянии овладеть страной.}}}

{\selectlanguage{english}\ttfamily
\foreignlanguage{russian}{\textrm{\textbf{49}}}\foreignlanguage{russian}{\textrm{ Мудрый человек не имеет собственного
сердца. Его сердце состоит из сердец народа. Добрым я делаю добро и недобрым также желаю добра. Это и есть добродетель,
порождаемая дэ. Искренним я верю и неискренним также верю. Это и есть искренность, вытекающая из дэ. Мудрый человек
живет в мире спокойно и в своем сердце собирает мнения народа. Он смотрит на народ как на своих детей.}}}

{\selectlanguage{english}\ttfamily
\foreignlanguage{russian}{\textrm{\textbf{50}}}\foreignlanguage{russian}{\textrm{ [Существа] рождаются и умирают. Из
десяти человек три [идут] к жизни, и из десяти — три человека [идут] к смерти. Из каждых десяти еще имеются три
человека, которые умирают от своих деяний. Почему это так? Это происходит оттого, что у них слишком сильно стремление к
жизни.}}}

{\selectlanguage{russian}
Я слышал, что кто умеет овладевать жизнью, идя по земле, не боится носорога или тигра, вступая в битву, не боится
вооруженных солдат. Носорогу некуда вонзить в него свой рог, тигру негде наложить на него свои когти, а солдату негде
поразить его своим мечом. Почему это так? Это происходит оттого, что для него не существует смерти.}

{\selectlanguage{english}\ttfamily
\foreignlanguage{russian}{\textrm{\textbf{51}}}\foreignlanguage{russian}{\textrm{ Дао рождает [вещи], дэ вскармливает
[их]. Вещи оформляются, формы завершаются. Поэтому нет вещи, которая не возвышала бы дао и не ценила бы дэ. Дао
возвышенно, дэ почтенно, потому что они не отдают приказаний, а следуют естественности.}}}

{\selectlanguage{russian}
Дао рождает [вещи], дэ вскармливает их, взращивает их, воспитывает их, \ совершенствует их, делает их зрелыми, ухаживает
за ними, поддерживает их.}

{\selectlanguage{russian}
Создавать и не присваивать, творить и не хвалиться, являясь старшим, не повелевать. Вот что называется глубочайшим дэ.}

{\selectlanguage{english}\ttfamily
\foreignlanguage{russian}{\textrm{\textbf{52}}}\foreignlanguage{russian}{\textrm{ В поднебесной имеется начало, и оно и
есть мать поднебесной. Когда будет достигнута мать, то можно узнать и ее детей. Когда уже известны ее дети, то снова
нужно помнить об их матери. В таком случае до конца жизни [у человека] не будет опасностей. Если [человек] оставляет
свои желания и освобождается от страстей, то до конца жизни не будет у него усталости. Если же он распускает свои
страсти и поглощен своими делами, то не будет спасения [от бед].}}}

{\selectlanguage{russian}
Видеть мельчайшее называется ясностью. Сохранение слабости называется могуществом. Употребляя его блеск, снова сделать
его [дао] ясным. Тогда до конца жизни [у человека] не будет несчастья. Это называется соблюдением постоянства.}


\bigskip

{\selectlanguage{english}\ttfamily
\foreignlanguage{russian}{\textrm{\textbf{53}}}\foreignlanguage{russian}{\textrm{ Если бы я владел знанием, то шел бы по
большой дороге. Единственная вещь, которой я боюсь, это действие. Большая дорога совершенно ровна, но народ любит
тропинки.}}}

{\selectlanguage{russian}
Если дворец роскошен, то поля покрыты сорняками и хлебохранилища совершенно пусты. [Знать] одевается в роскошные ткани,
носит острые мечи, не удовлетворяется [обычной] пищей и накапливает излишние богатства. Все это называется разбоем и
бахвальством. Оно является нарушением дао.}

{\selectlanguage{english}\ttfamily
\foreignlanguage{russian}{\textrm{\textbf{54}}}\foreignlanguage{russian}{\textrm{ Кто умеет крепко стоять, того нельзя
опрокинуть; кто умеет опереться, того нельзя свалить. Сыновья и внуки вечно сохранят память о нем.}}}

{\selectlanguage{russian}
Кто совершенствует [дао] внутри себя, у того добродетель становится искренней. Кто совершенствует дао в семье, у того
добродетель становится обильной. Кто совершенствует [дао] в деревне, у того добродетель становится обширной. Кто
совершенствует [дао] в царстве, у того добродетель становится богатой. Кто совершенствует [дао] в поднебесной, у того
добродетель становится великой.}

{\selectlanguage{russian}
По себе можно познать других; по одной семье можно познать другие; по одной деревне можно познать остальные; по одному
царству можно познать другие; по одной стране можно познать весь мир. Откуда я знаю мир как таковой? Благодаря этому.}

{\selectlanguage{english}\ttfamily
\foreignlanguage{russian}{\textrm{\textbf{55}}}\foreignlanguage{russian}{\textrm{ Кто содержит в себе совершенное дэ,
тот похож на новорожденного. Ядовитые насекомые и змеи }}\foreignlanguage{russian}{\textrm{его не кусают, свирепые
звери на него не нападают, хищные птицы его не схватывают. Кости у него мягкие, мышцы слабые, но он держит [дао]
крепко. Не зная союза двух полов, он обладает животворящей способностью. Он очень чуток. Он кричит весь день, и его
голос не изменяется. Он совершенно гармоничен.}}}

{\selectlanguage{russian}
Знание гармонии называется постоянством. Знание постоянства называется просвещением. Обогащение жизни называется
счастьем. Напряжение духа в сердце называется упорством. Существо, полное сил, стареет – это называется нарушением дао.
Кто без дао, погибает раньше времени.}

{\selectlanguage{english}\ttfamily
\foreignlanguage{russian}{\textrm{\textbf{56}}}\foreignlanguage{russian}{\textrm{ Знающие не говорят, говорящие не
знают. Кто оставляет свои желания, отказывается от страстей, притупляет свои стремления, освобождает свои [мысли] от
путаницы, умеряет свой блеск, сводит [свои впечатления] воедино, тот представляет собой торжество глубочайшего. Его
нельзя приблизить для того, чтобы с ним сродниться; его нельзя приблизить для того, чтобы им пренебрегать; его нельзя
приблизить для того, чтобы им воспользоваться; его нельзя приблизить для того, чтобы его возвысить; его нельзя
приблизить для того, чтобы его унизить. Вот почему он уважаем в стране.}}}

{\selectlanguage{english}\ttfamily
\foreignlanguage{russian}{\textrm{\textbf{57}}}\foreignlanguage{russian}{\textrm{ Страна управляется справедливостью,
война ведется хитростью. Завоевание страны осуществляется посредством недеяния. Откуда я знаю все это? Вот откуда:
когда в стране много ненужных вещей, народ становится бедным; когда у народа много острого оружия, в стране
увеличиваются смуты; когда много искусных мастеров, умножаются редкие предметы; когда растут законы и приказы,
увеличивается число воров и разбойников.}}}

{\selectlanguage{russian}
Поэтому мудрый человек говорит: если я не действую, народ будет находиться в самоизменении; если я спокоен, народ сам
будет исправляться. Если я пассивен, народ сам становится богатым; если я не имею страстей, народ становится
простодушным.}


\bigskip

{\selectlanguage{english}\ttfamily
\foreignlanguage{russian}{\textrm{\textbf{58}}}\foreignlanguage{russian}{\textrm{ Когда правительство спокойно, народ
становится простодушным. Когда правительство деятельно, народ становится несчастным. О, несчастье! Оно основа, на
которой держится счастье. О, счастье! В нем заключено несчастье.}}}

{\selectlanguage{russian}
Кто знает их границы? Они не имеют постоянства. Справедливость снова превращается в хитрость, добро во зло. Человек уже
давно находится в заблуждении. Поэтому мудрый человек справедлив и не отнимает ничего у другого. Он бескорыстен и не
вредит другим. Он правдив и не делает ничего плохого. Он светел, но не желает блестеть.}

{\selectlanguage{english}\ttfamily
\foreignlanguage{russian}{\textrm{\textbf{59}}}\foreignlanguage{russian}{\textrm{ Для того чтобы управлять страной и
служить людям, лучше всего соблюдать воздержание. Воздержание должно стать главной заботой. Оно называется
усовершенствованием дэ. Совершенное дэ – всепобеждающая сила.}}}

{\selectlanguage{russian}
Всепобеждающая сила неисчерпаема. Неисчерпаемая сила дает возможность овладеть страной.}

{\selectlanguage{russian}
Начало, при помощи которого управляется страна, долговечно и называется глубоким и прочным корнем. Оно вечно
существующее дао.}

{\selectlanguage{english}\ttfamily
\foreignlanguage{russian}{\textrm{\textbf{60}}}\foreignlanguage{russian}{\textrm{ Управление большим царством напоминает
приготовление блюда из мелких рыб. Если люди вступают в поднебесную путем дао, то духи умерших не творят чудеса. Не
только духи умерших не будут творить чудеса, они также перестанут вредить людям. Не только дух не будет вредить людям,
но и мудрый человек не будет вредить людям. Поскольку оба они не приносят вреда, то их дэ соединяются друг с другом.}}}

{\selectlanguage{english}\ttfamily
\foreignlanguage{russian}{\textrm{\textbf{61}}}\foreignlanguage{russian}{\textrm{ Великое царство – это низовье реки,
узел поднебесной, самка поднебесной.}}}

{\selectlanguage{russian}
Самка всегда невозмутимостью одолевает самца, а по своей невозмутимости [она] стоит ниже [самца]. Поэтому великое
царство располагает к себе маленькое тем, что ставит себя ниже последнего, а маленькое царство завоевывает симпатию
великого царства тем, что стоит ниже последнего.}

{\selectlanguage{russian}
Поэтому располагают к себе либо тем, что ставят себя ниже, либо потому, что сами по себе ниже. Пусть великое царство
будет желать не больше того, чтобы все одинаково были накормлены, а малое царство пусть будет желать не больше того,
чтобы служить людям. Тогда оба получат то, чего они желают. Великому полагается быть внизу.}

{\selectlanguage{english}\ttfamily
\foreignlanguage{russian}{\textrm{\textbf{62}}}\foreignlanguage{russian}{\textrm{ Дао – глубокая [основа] всех вещей.
Оно сокровище добрых и защита недобрых людей. Красивые слова можно произносить публично, доброе поведение можно
распространять на людей. Но зачем же покидают недобрых людей? В таком случае для чего же выдвигают государя и назначают
ему трех советников?}}}

{\selectlanguage{russian}
Государь и советники хотя и имеют драгоценные камни и могут ездить на колесницах, но лучше будет им спокойно следовать
дао.}

{\selectlanguage{russian}
Почему в древности ценили дао? В то время люди не стремились к приобретению богатств и преступления прощались. Поэтому
[дао] в поднебесной ценилось дорого.}

{\selectlanguage{english}\ttfamily
\foreignlanguage{russian}{\textrm{\textbf{63}}}\foreignlanguage{russian}{\textrm{ Нужно осуществлять недеяние, соблюдать
спокойствие и вкушать безвкусное. Великое состоит из мелких, а многое из малых. На ненависть нужно отвечать добром.}}}

{\selectlanguage{english}\ttfamily
\foreignlanguage{russian}{\textrm{Преодоление трудного начинается с легкого, осуществление великого дела начинается с
малого, ибо в мире трудное дело образуется из легких, а великое — из маленьких. Поэтому мудрец всегда начинает дело не
с великого, тем самым он завершает великое дело. Кто много обещает, тот не заслуживает
}}\foreignlanguage{russian}{\textrm{доверия.}}}

{\selectlanguage{russian}
Где много легких дел, там много и трудных. Поэтому мудрый человек погружается в трудности, в результате чего он их не
испытывает.}

{\selectlanguage{english}\ttfamily
\foreignlanguage{russian}{\textrm{\textbf{64 }}}\foreignlanguage{russian}{\textrm{То, что спокойно, легко сохранить. То,
что не показало признаков [своего существования], легко направить. То, что слабо, легко разделить. То, что мелко, легко
рассеять. Действие надо начать с того, чего еще нет.}}}

{\selectlanguage{russian}
Наведение порядка надо начать тогда, когда еще нет смуты. Ибо большое дерево вырастает из маленького, девятиэтажная
башня начинает строиться с горстки земли, путешествие в тысячу ли начинается с одного шага.}

{\selectlanguage{russian}
Кто действует, потерпит неудачу. Кто чем-либо владеет, потеряет. Вот почему мудрый человек бездеятелен и он не терпит
неудачи. Он ничего не имеет и поэтому ничего не теряет. Те, которые, совершая дела, спешат достигнуть успеха, потерпят
неудачу. Кто осторожно заканчивает свое дело, подобно тому, как он его начал, у того всегда будет благополучие. Поэтому
мудрый человек не имеет страсти, не ценит трудно добываемые предметы, учится у тех, которые не имеют знаний, и идет по
тому пути, по которому прошли другие. Он следует естественности вещей и не осмеливается [самовольно] действовать.}

{\selectlanguage{english}\ttfamily
\foreignlanguage{russian}{\textrm{\textbf{65}}}\foreignlanguage{russian}{\textrm{ В древности умевший служить дао не
просвещал народ, а делал его глупым.}}}

{\selectlanguage{russian}
Трудно управлять народом, когда у него много знаний. Поэтому управление страной при помощи знаний – враг страны, а без
их применения – счастье страны.}

{\selectlanguage{russian}
Кто знает эти две вещи, тот становится примером для других. Знание этого примера есть знание глубочайшего дэ.
Глубочайшее дэ, оно и глубоко и далеко. Оно противоположно всем существам. Следуя за ним, достигнешь великого
благополучия.}

{\selectlanguage{english}\ttfamily
\foreignlanguage{russian}{\textrm{\textbf{66}}}\foreignlanguage{russian}{\textrm{ Реки и моря потому могут властвовать
над равнинами, что они способны стекать вниз. Поэтому они властвуют над равнинами.}}}

{\selectlanguage{russian}
Когда [мудрый человек] желает возвыситься над народом, он должен ставить себя ниже других. Когда он желает быть впереди
людей, он должен ставить себя позади других. Поэтому, хотя он стоит над народом, но для народа он не в тягость; хотя он
находится впереди, народ ему не вредит. Поэтому люди с радостью его выдвигают и от него не отворачиваются. Он не
борется, благодаря чему он в мире непобедим.}

{\selectlanguage{english}\ttfamily
\foreignlanguage{russian}{\textrm{\textbf{67}}}\foreignlanguage{russian}{\textrm{ Все говорят о том, что мое дао велико
и не уменьшается. Если бы оно уменьшилось, то после долгого времени оно стало бы маленьким. Не уменьшается потому, что
оно является великим.}}}

{\selectlanguage{english}\ttfamily
\foreignlanguage{russian}{\textrm{Я имею три сокровища, которыми дорожу: первое – это человеколюбие; второе –
бережливость, а третье состоит в том, что я не смею быть впереди других. \newline
Я человеколюбив, поэтому могу стать храбрым. Я бережлив, поэтому могу стать щедрым. Я не смею быть впереди других,
поэтому могу стать умным вождем.}}}

{\selectlanguage{english}\ttfamily
\foreignlanguage{russian}{\textrm{Кто храбр без гуманности, щедр без бережливости, находясь впереди, отталкивает тех,
кто находится позади, – тот погибает. Кто ведет войну из-за человеколюбия, тот побеждает, и возведенная им оборона —
неприступна.}}}

{\selectlanguage{russian}
Естественность его спасает, человеколюбие его охраняет.}

{\selectlanguage{english}\ttfamily
\foreignlanguage{russian}{\textrm{\textbf{68}}}\foreignlanguage{russian}{\textrm{ Умный полководец не бывает
воинственен. Умелый воин не бывает гневен.}}}

{\selectlanguage{russian}
Умеющий побеждать врага не нападает. Умеющий управлять людьми ставит себя в низкое положение. Это я называю дэ,
избегающее борьбы. Это сила в управлении людьми. Это значит следовать природе и древнему началу [дао].}

{\selectlanguage{english}\ttfamily
\foreignlanguage{russian}{\textrm{\textbf{69}}}\foreignlanguage{russian}{\textrm{ Военное искусство гласит: я не смею
первым начинать, я должен ожидать.}}}

{\selectlanguage{russian}
Я не смею наступать хотя бы на вершок вперед, а отступаю на аршин назад. Это называется действием посредством недеяния,
ударом без усилия. В этом случае не будет врага, и я могу обходиться без солдат. Нет беды тяжелее, чем недооценивать
противника. Недооценка противника повредит моему сокровенному средству [дао]. В результате сражений те, кто скорбят,
одерживают победу.}

{\selectlanguage{english}\ttfamily
\foreignlanguage{russian}{\textrm{\textbf{70}}}\foreignlanguage{russian}{\textrm{ Мои слова легко понять и легко
осуществить. Но люди не могут понять и не могут осуществлять. В словах имеется начало, в делах имеется главное.}}}

{\selectlanguage{russian}
Поскольку люди их не знают, то они не знают и меня. Когда меня мало знают, тогда я дорог. Поэтому мудрый человек подобен
тому, кто одевается в грубые ткани, а при себе держит яшму.}

{\selectlanguage{english}\ttfamily
\foreignlanguage{russian}{\textrm{\textbf{71}}}\foreignlanguage{russian}{\textrm{ Кто имеет знания и делает вид
незнающего, тот на высоте. Кто без знаний и делает вид знающего, тот болен. Кто избавляет себя от болезни — не болеет.
Мудрый человек не болеет, потому что он избавляет себя от болезни. Поэтому он не болеет.}}}

{\selectlanguage{english}\ttfamily
\foreignlanguage{russian}{\textrm{\textbf{72}}}\foreignlanguage{russian}{\textrm{ Когда народ не боится могущественных,
тогда приходит могущество. Не тесните его жилища, не презирайте его жизни. Кто не презирает [народ], тот не будет
презрен [народом].}}}

{\selectlanguage{russian}
Поэтому мудрый человек, зная себя, себя не выставляет. Он любит себя и себя не возвышает. Он отказывается от самолюбия и
предпочитает невозвышение.}

{\selectlanguage{english}\ttfamily
\foreignlanguage{russian}{\textrm{\textbf{73}}}\foreignlanguage{russian}{\textrm{ Кто храбр и воинственен, погибает, кто
храбр и не воинственен, будет жить. Эти две вещи означают: одна пользу, а другая вред. Кто знает причины ненависти к
воинственным? Объяснить это затрудняется и мудрец.}}}

{\selectlanguage{russian}
Естественное дао не борется, но умеет побеждать. Оно не говорит, но умеет отвечать. Оно само приходит. Оно спокойно и
умеет управлять [вещами].}

{\selectlanguage{russian}
Сеть природы редка, но ничего не пропускает.}

{\selectlanguage{english}\ttfamily
\foreignlanguage{russian}{\textrm{\textbf{74}}}\foreignlanguage{russian}{\textrm{ Если народ не боится смерти, то зачем
же угрожать ему смертью? Кто заставляет людей бояться смерти и считает это занятие увлекательным, того я захвачу и
уничтожу. Кто осмеливается так действовать?}}}

{\selectlanguage{russian}
Всегда существует носитель смерти [дао], который убивает. А если кто его заменит, то, значит, заменит великого мастера
[дао]. Кто, заменяя великого мастера, рубит [топором], повредит свою руку.}

{\selectlanguage{english}\ttfamily
\foreignlanguage{russian}{\textrm{\textbf{75}}}\foreignlanguage{russian}{\textrm{ Народ голодает оттого, что слишком
велики поборы и налоги. Вот почему [народ] голодает. Трудно управлять народом оттого, что правительство слишком
деятельно. Вот почему трудно управлять. Народ легко умирает оттого, что у него слишком сильно стремление к жизни. Вот
почему легко умирает. Тот, кто пренебрегает своей жизнью, тем самым ценит свою жизнь.}}}

{\selectlanguage{english}\ttfamily
\foreignlanguage{russian}{\textrm{\textbf{76}}}\foreignlanguage{russian}{\textrm{ Человек при рождении нежен и слаб, а
после смерти тверд и крепок. Все существа и растения при своем рождении нежны и слабы, а при гибели тверды и крепки.
Твердое и крепкое это то, что погибает, а нежное и слабое есть то, что начинает жить. Поэтому могущественное войско не
побеждает, и оно, подобно крепкому дереву, [гибнет]. Сильное и могущественное не имеют того преимущества, какое имеют
нежное и слабое.}}}

{\selectlanguage{english}\ttfamily
\foreignlanguage{russian}{\textrm{\textbf{77}}}\foreignlanguage{russian}{\textrm{ Естественное дао напоминает
натягивание лука. Когда понижается его верхняя часть, поднимается нижняя. Оно отнимает лишнее и отдает отнятое тому,
кто в нем нуждается. Естественное дао отнимает у богатых и отдает бедным то, что у них отнято. Человеческое же дао —
наоборот. Оно отнимает у бедных и отдает богатым то, что отнято. Кто может отдать другим все лишнее?}}}

{\selectlanguage{russian}
Это могут сделать только те, которые следуют дао. Поэтому мудрый человек делает и не пользуется тем, что сделано,
совершает подвиги и себя не прославляет. Он благороден потому, что у него нет страстей.}

{\selectlanguage{english}\ttfamily
\foreignlanguage{russian}{\textrm{\textbf{78}}}\foreignlanguage{russian}{\textrm{ Вода – это самое мягкое и самое слабое
существо в мире, но в преодолении твердого и крепкого она непобедима, и на свете нет ей равного.}}}

{\selectlanguage{russian}
Слабые побеждают сильных, мягкое преодолевает твердое. Это знают все, но люди не могут это осуществлять. Поэтому мудрый
человек говорит: кто принял на себя унижение страны, становится государем, и кто принял на себя несчастье страны,
становится властителем.}

{\selectlanguage{russian}
Правдивые слова похожи на свою противоположность.}


\bigskip

{\selectlanguage{english}\ttfamily
\foreignlanguage{russian}{\textrm{\textbf{79}}}\foreignlanguage{russian}{\textrm{ После большого возмущения останутся
его последствия. Спокойствие можно назвать добром. Поэтому мудрый человек дает клятвенное обещание, что он не будет
никого порицать. Добрые люди соблюдают свою клятву, а недобрые ее нарушают. Естественное дао не имеет родственников,
оно всегда на стороне добрых.}}}

{\selectlanguage{english}\ttfamily
\foreignlanguage{russian}{\textrm{\textbf{80}}}\foreignlanguage{russian}{\textrm{ Нужно сделать государство маленьким, а
народ редким. Даже если имеется много орудий, не надо их употреблять. Нужно сделать так, чтобы народ не странствовал
далеко до конца своей жизни. Даже если имеются лодки и колесницы, не надо их употреблять. Даже если имеются вооруженные
войска, не надо их выставлять. Надо сделать так, чтобы народ снова начал плести узелки и употреблять их вместо письма.
Надо сделать вкусным его питание, прекрасным его одеяние, устроить ему спокойное жилище, сделать веселой его жизнь.}}}

{\selectlanguage{russian}
Соседние государства смотрели бы друг на друга издали, слушали бы друг у друга пение петухов и лай собак, а люди до
старости и смерти не должны были бы кочевать с места на место.}

{\selectlanguage{english}\ttfamily
\foreignlanguage{russian}{\textrm{\textbf{81}}}\foreignlanguage{russian}{\textrm{ Верные слова не изящны. Красивые слова
не заслуживают доверия. Добрый не красноречив. Красноречивый не может быть добрым. Знающий не доказывает, доказывающий
не знает.}}}

{\selectlanguage{russian}
Мудрый человек ничего не накапливает. Он все делает для людей и все отдает другим. Небесное дао приносит всем существам
пользу и им не вредит.}

{\selectlanguage{english}\ttfamily
\foreignlanguage{russian}{\textrm{Дао мудрого человека — это деяние без борьбы.}}}


\bigskip

{\centering\selectlanguage{russian}\bfseries
* \ \ \ \ \ \ \ \ \ * \ \ \ \ \ \ \ \ \ *
\par}
\end{document}
