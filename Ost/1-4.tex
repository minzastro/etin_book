% This file was converted to LaTeX by Writer2LaTeX ver. 1.4
% see http://writer2latex.sourceforge.net for more info
%\documentclass[a4paper]{article}
%\usepackage[utf8]{inputenc}
%\usepackage[T2A]{fontenc}
%\usepackage[russian,english]{babel}
%\usepackage{amsmath}
%\usepackage{amssymb,amsfonts,textcomp}
%usepackage{color}
%\usepackage{array}
%\usepackage{supertabular}
%\usepackage{hhline}
%\begin{document}
\section[Предисловие]{Предисловие}
В (не)доброе старое время язык, обычаи, экономическая культура, религия, идеология, государственные структуры и другие
средства самоидентификации более или менее надежно обособляли человеческие общности, не допускали их распада. Люди
(племени, рода, клана, народности, нации) всегда имели самоназвание, имевшее значение \textbf{\textit{люди}}: людьми
были только они.

Большие заборы, Великие стены, Железные и Бамбуковые занавесы не столько защищают от внешнего врага, сколько сплачивают
население Изумрудного Города, подчиняя его Гудвину — Великому и Ужасному. Очки самоидентификации жители носят с детства
и именно сквозь них видят свое светлое будущее.

Заборы становятся все ниже (их разбирают на дрова), стены ветшают (их растаскивают на сувениры), охрана занавесов уже не
так хорошо оплачивается, государственные границы превращаются в таможенные и просто стираются, доблестные воины
защищают Родину далеко от нее или внутри. Средства обособления (=самоидентификации) отступают перед лавиной
\flqq глобализации\frqq: знание иностранного языка многими ценится выше владения родной культурой.

Глобализация ставит нас перед проблемой самоидентификации: кто мы?

\textit{Выработанные предыдущими поколениями средства самоидентификации перестали работать, потребность в самоидентификации
обострилась: в старых тряпках холодно на ветру перемен.}

Вас готовы приютить: православие, ислам, протестантство, католичество, иудаизм, кришнаитство.

\flqq Продвинутый\frqq\ человек ищет (и находит!) средства самоидентификации на просторах Сети.

\textbf{На Западе ищут путей на Восток} (на Востоке в большом ходу Запад), полагая, что \flqq везде хорошо, где нас
нет\frqq\footnote{Поэтому и хорошо!}. Именно \flqq замкнутость\frqq\ Востока привлекает западных сирот:
на \flqq замкнутом\frqq\ месте Земля снова станет привычно
плоской\footnote{Психиатрам известна (образовательная) фобия: боязнь соскользнуть с круглой Земли.} и вокруг будут снова только \textit{свои}.

Мы побеседуем о Востоке вообще, мы побеседуем о философии на Востоке, не стараясь отделить ее от религии (безнадежное дело, скажу я вам).


\textit{Мы исключим из наших бесед иудаизм, ислам и христианство: в них женщина считается существом не только низшим, но и
нечистым. Полемика с такого сорта \flqq моралью\frqq\ — не мое дело: этому решительно препятствуют мое уважение к женщине — основе жизни, чувство собственного достоинства, образовательный ценз, жизненный опыт и ответственность перед вами.}

В привычных нам географических терминах мы соприкоснемся с Индией, \ Китаем и Японией.

\textbf{Индия: индуизм, джайнизм, буддизм. Философских систем \textbf{брахманизма} мы коснемся чуть-чуть.}

{\bfseries
Китай: даосизм, конфуцианство, чань (дзэн).}

{\bfseries
Япония: китайское влияние, синтоизм, дзэн; восточнее Востока.}


\bigskip

Вступительные Общие замечания неизбежны: они укрепят наш западный дух перед вступлением в чужой мир. Заключительные
Общие замечания \ размещены в конце, хотя с них следовало бы начать.

Мы кратко обсудим проблемы в мире современного западного образованного человека в сопоставлении с некоторыми восточными
системами мира и мышления.

Самоидентификация не будет предложена на поле западной культуры акт патриотической защиты \flqq самости\frqq\ — российской,
европейской, западной, евразийской и проч. (\flqq имя им – легион\frqq). Глобальность мира — такой же \textit{факт}, как
шарообразность Земли.

\section{Общие замечания - 1}
В религиозно-философских системах Запада в рамках концепции единобожия возникла масса проблем, питавших (и продолжающих
питать) все виды дискурса. Примером может послужить теодицея — учение о снятии с Бога ответственности за зло в мире
(Лейбниц). На Востоке западного Бога нет, нет западных дихотомий добро/зло, знание/незнание.

Концепции, категории, понятия, термины и воззрения Востока не могут быть даже описательно переведены на наши языки: мы —
\textit{другие}.

\textit{Русская бабушка, приехавшая в гости к своим американским внукам, наблюдает за игрой соседских детей: \flqq Неужели они
действительно не знают ни одного слова по-русски?\frqq.}

Убогая дихотомия многобожия и монотеизма, обязанная своим рождением межплеменной вражде в ее западном варианте, в
принципе неприложима к религиозно-философской практике Востока.

На Западе безраздельно господствует идея \textit{познаваемости мира}. Истина предполагается существующей, споры идут
только о конкретных формулировках или об авторских правах. На Востоке господствует парадигма \textit{невыразимости}
истины: и ты прав, и я прав, и они правы. Предмет спора отсутствует, профессиональные спорщики не имеют приемлемого
статуса. Западный умник не может вступить в борьбу с восточной ветряной мельницей, ему остается бороться с ветром.

Наше представление о религиозных текстах неприложимо на Востоке. Например, в последних гимнах Ригведы высказано не
только сомнение во всемогуществе богов, но даже в их существовании — широта охвата, недоступная пониманию тех, кто
приступает к тексту в убеждении, что автор либо ортодокс, либо еретик. В
V-VI веках до н.э. (Осевое время!) кшатрии выступают с
отрицанием Вед и брахманских ритуалов — оставаясь индуистами. Предуведомление: консолидированные \flqq протестанты\frqq\ породили
джайнизм и буддизм — \textit{внутренние движения} в индуизме.

В русле брахманизма ритуальные жертвенные животные образовали цепочку Человек - Конь - Бык - Баран - Козел, породившую в
итоге учение об \textit{ахимсе }— табу на причинение вреда живому.

\textit{Что соответствует этому процессу на Западе — инквизиция, гестапо, ВЧК, серийные убийцы, серийные диктаторы, серийные фильмы?}

\section{Индуизм}
\subsection[Немного истории, совсем немного]{Немного истории, совсем немного}
Бытует мнение, что арии\footnote{Арии = верные, люди одной расы (крови)} вторглись в
Индостан в III тысячелетии до нашей эры и разрушили существовавшую там цивилизацию.
Напраслина: протодравидийская общность пришла в упадок до вторжения ариев. Арии поначалу
XII-XI вв. до н.э.) освоили Пенджаб — место рождения Вед.
Варновые\footnote{Варна (= цвет).\textit{Двиджа}(=дважды рожденные): брахманы, кшатрии, вайшьи. Шудры – низшая варна.\par
Каста - португальский термин, соответствующий арийскому
\textit{джати.}Каст — тысячи!} преграды — средства
самоидентификации того времени — не помешали ариям заимствовать у аборигенов Шиву, Кришну и других богов. Местная
культура не только надолго сохранилась, но и существенно повлияла на формирование новых течений в арийской культуре.
Кочующие арии постепенно приобщались к развитой земледельческой культуре аборигенов, язык ариев пополнился названиями
хозяйственных орудий, животных, растений, предметов домашнего обихода.

Арии замечательно самоидентифицировались, отгородив носителей социальных ролей от внешних влияний системой варн. Каждая
варна жила в собственном изолированном мире. Для нарушителей ролевых табу предусмотрена социальная помойка, тоже
структурированная (\textit{шудры}).

Регламенту подчинены бытовая технология жизни и распорядок жизни в целом — \textit{ашрама}.

\textit{Ашрама }(путь мужчины): с 8 лет ученичество для подготовки ко второму рождению, к роли главы семьи; с 20 лет до
40 лет — жизнь в семье в рамках \textit{дхармы }(социального порядка) для достижения \textit{камы} (удовольствия) и
\textit{артхи} (пользы); после 40 лет мужчина уединяется, чтобы придать смысл наступающей старости, совершает Великий
Уход на северо-запад — возвращается к истоку и возвращается к палеолитическому собирательству. Старики не обременяют
семью. Община имеет свободный доступ к ресурсам семьи. С течением времени места для уединения тоже стали хозяйственными
угодьями, так что отшельников стали принимать города.

Города всегда были и будут резервуаром для лишних людей.

Арии позаботились о структуре мира, населив его виртуальными образами своего социума:

\begin{flushleft}
\tablefirsthead{}
\tablehead{}
\tabletail{}
\tablelasttail{}
\begin{supertabular}{m{3in}m{3in}}
\begin{itemize}
\item Варуна – начальник Вселенной;
\item Ваю – начальник пространства;
\item Дьяус (=Зевс=Деос=Тео) – прародитель;
\end{itemize}
 &
\begin{itemize}
\item Митра – начальник света и дня;
\item Сома – начальник сомы;
\item Рудра – начальник ветра.
\end{itemize}
\\
\end{supertabular}
\end{flushleft}
Структура сонма богов фиксировала структуру мира, подчиненного регламенту.


Тебя не устраивают наши боги — приводи своих (но не вместо наших), мы их примем, в нашем мире места много. Поклоняйся
любым богам, соблюдая регламент мира.

За соблюдением регламента во время совершения обрядов наблюдал брахман – старший жрец, ему за это причиталась ровно
половина всех пожертвований.

Индуизму, этой мощной духовной традиции, нет никакого соответствия на нашем Западе. Индуизм включает культы и
философские системы, ритуалы и церемонии, философию и практику жизни, поэзию и музыку,
танцы\footnote{Танцовщики считались полубогами.} и заклинания. Индуизм охватывает все то,
что мы недавно стали называть культурой в широком (всеобъемлющем) смысле слова.

Источник и хранилище идей и практик индуизма — \textbf{Веды}, собрание древних анонимных произведений
(\textit{veda} = знание). Тексты, передаваемые из поколения в поколение (начиная примерно с
2500 года до н.э.), делятся на \textbf{\textit{шрути}} — откровения живших в лесном уединении \textit{риши,}
предназначенные только посвященным (=двиджа), и \textbf{\textit{смрити}} — предания, доступные членам всех варн.

\subsection{Шрути }
Существуют четыре собрания \textbf{Вед}: \textit{самхиты}, \textit{брахманы,} \textit{араньяки,}
\textit{упанишады}.


Веды написаны на санскрите, священном языке Индии, известном и сегодня каждому \textup{образованному}
человеку\footnote{В Индии основных языков – 18.}. Максимальное число текстов создано в
Осевое время (!).

\textbf{Самхиты}\textbf{\textit{Ригведа}} содержит гимны, \textbf{\textit{Самаведа}} содержит напевы –
вариации на темы Ригведы, \textbf{\textit{Яджурведа}} посвящена жертвенным формулам (тоже опирается на Ригведу),
\textbf{\textit{Атхарваведа}} содержит (дравидийские) заклинания \textit{атхарвов} – жрецов огня.


Ригведа (веда гимнов) содержит 1028 гимнов (в среднем по 10 стихов в одном гимне). Самаведа (веда напевов) содержит
свыше 1500 стихов (мелодий), повторяя Ригведу. Яджурведа тоже дублирует Ригведу.

\textbf{Брахманы} относятся к каждой из \textbf{самхит} и содержат ритуалы и мифы.

\textbf{Араньяки} (тексты для уединившихся) и \textbf{упанишады} (тексты для отшельников) примыкают как непосредственно
к \textbf{самхитам}, так и к их \textbf{брахманам}. \textbf{Упанишады} (их более 100) посвящены размышлениям, которые
мы называем философскими. Доступ к ним имели только двиджа.


Античный Запад ознакомился с индуизмом через Упанишады.

\textbf{Упанишады} излагали: грамматику, правила почитания предков, науку чисел, искусство предсказаний, хронологию,
логику, правила поведения, этимологию, науку о священном знании, науку о демонах, военную науку, астрономию, науки о
змеях и низших божествах. \textbf{Упанишады} в целом называют \textit{веданта}. К Ведам примыкают (учебные)
\textit{веданги, }излагающие фонетику,\textit{ }ритуальные правила, грамматику, этимологию, метрику и астрономию.

\subsection[Смрити]{Смрити}
\textbf{Махабхарата} и \textbf{Рамаяна} – эпические (западный термин!) поэмы, скорее – целые литературы. \textbf{Пураны}
– богословские тексты. \textbf{Итихасы} – исторические предания. \textbf{Дхармасутры} и \textbf{Шастры – }повествования
об этике и обрядах.

\subsection{Философские учения }
\begin{flushright}
    
\textquotedbl Взяв, словно лук, великое оружие Упанишад,\newline
Следует возложить на него стрелу, отточенную медитацией \textquotedbl.
\end{flushright}

Их принято делить на \textbf{\textit{ортодоксальные}} (\textit{астика}):

\begin{itemize}
\item \textbf{ньяя} – восприятие, вывод (силлогизм), сравнение и доказательство суть \textit{источники знания};
\textit{объекты познания} суть: Я, познавательная способность, ум, деятельность, умственные дефекты, удовольствие и
боль, страдание, свобода от страдания, повторное рождение; все вещи мира должны иметь своего создателя; свобода воли
вносит в мир добро и зло;
\item \textbf{вайшешика} – \textit{объекты познания} суть: субстанция, качество, действие, всеобщность, особенность,
присущность и небытие; (субстанции суть: земля, вода, огонь, воздух, эфир, время, пространство, душа и ум; первые
четыре состоят из невидимых и неделимых атомов);
\item \textbf{санкхья} – \textit{пуруша} – сознательное начало; \textit{пракрити} – бессознательное начало, первопричина мира;
эволюция мира начинается с соединения этих двух начал (в ходе эволюции число начал достигает 25); вопрос о богах не
находит себе места; путь к освобождению – воспитание в технике йоги;
\item \textbf{йога} – принимает учение санкхьи о 25 началах, но признает бытие божества; полная сосредоточенность ума приводит
к пониманию (высший объект понимания – божество); прекращение всех функций ума = йога;
\item \textbf{миманса} – Веды = источник знания и веры, источник ритуалов; долг исполняется во имя долга, исполнение долга не
вознаграждается; душа вечна, но обладает сознанием только находясь в теле человека; к источникам знания добавляется
постулирование; мир реален и управляется законами кармы;
\item \textbf{веданта – }мир начинается из единого Брахмана, из единой души, душа = бог = реальность = неограниченное сознание
= блаженство; мир есть видимость, вызванная в нас силой \textit{майя}; как только ты поймешь, что мир – видимость, ты
перестанешь видеть за ним творца, перестанешь приписывать божеству свойства; путь к освобождению указывает Учитель;
\end{itemize}

в школе \textbf{адвайта} (недвойственной) Атман тождествен Брахману; в ограниченной адвайте Атман и Брахман
{\textquotedbl}пересекаются{\textquotedbl}, но не совпадают; в школе \textbf{двайта} они обособлены и не сливаются.

и \textbf{\textit{неортодоксальные}} (\textit{настика}):

\begin{itemize}
\item \textbf{джайнизм} – восприятие, доказательство и свидетельство суть \textit{источники знания}; сознание присуще всему
живому, хотя и в разной степени; существуют пространство, время, причины движения и покоя; поклоняться следует не
богам, а идеалам; кармы (страсти и желания) закрепощают душу материей; сострадание ко всему живому; уважение к другим
учениям;
\item \textbf{буддизм} – четыре благородные истины (наличие страданий, причина страданий, прекращение страданий, путь к
прекращению страданий – к \textit{нирване});
\item \textbf{локаята} – мир материален; ни души, ни богов нет; ритуалы кормят хитрых жрецов.
\end{itemize}
В каждом учении – множество школ.

В основе индуизма лежит мысль о том, что весь мир, окружающий нас, по-разному воплощает одну и ту же высшую реальность.
Эта реальность — \textbf{\textit{Брахман}}\textit{ —} представляет собой понятие, существованию которого индуизм обязан
своим принципиально монистическим характером, несмотря на почитание огромного количества богов и богинь.


Необозримый сонм богов и богинь индуизма \ приводит в смятение ум западного человека.

\textbf{Брахман} есть внутренняя сущность всех вещей. Он бесконечен и превосходит все представления.
\textit{\flqq Непостижима эта высшая Душа, безграничная, не рожденная, не подлежащая обсуждению, не допускающая мыслей\frqq.}

Воплощение Брахмана в душе человека называется Атман (=я сам). Упанишады учат единству Атмана и Брахмана, личной и
высшей реальностей: \flqq То, что является тончайшей сущностью, то, что является душой всего этого мира. Это
реальность. Это Атман. Это ты\frqq.

Брахман и Атман — две равноправных ипостаси Абсолюта (на \textit{нашем} жаргоне). У Брахмана три ипостаси: пространство
$\Rightarrow$ материя, движение $\Rightarrow$ энергия, закон $\Rightarrow$ закономерность.

Основной сюжет индуизма — сотворение мира путем самопожертвования Бога: Бог становится миром, мир в итоге становится
Богом. Брахман использует для этого подвига магическую созидательную силу — \textit{майя}. Созидательная деятельность
Божественного носит название \textit{Лила} (божественная игра), и весь мир — это результат и арена действия этой игры.


\textit{Нынче вместо мощи и силы \textbf{майя} стало означать психологическое состояние любого человека, находящегося под чарами
божественной игры. До тех пор, пока мы считаем реальностью мириады форм \textbf{Лилы}, не осознавая целостности
Брахмана, лежащего в основе всех этих форм, мы находимся \textbf{под властью чар майи}}.

Мир не есть иллюзия, иллюзорны лишь наши представления о том, что формы и структуры, вещи и события вокруг нас реальны,
в то время как все это — лишь сеть понятий, при помощи которых мы мыслим, измеряя и абстрагируя. Майя — иллюзорное
отождествление этих понятий с реальностью (изображения с изображаемым). Брахман лежит \textit{вне} всех понятий и
образов.

Движущая сила Лилы — \textit{карма} (действие). Карма приводит в движение Вселенную, все части которой динамически
связаны друг с другом.

\textit{\flqq Все действия занимают свое место во времени благодаря взаимопереплетению сил Природы, однако человек, погрязший в
заблуждениях эгоизма, думает, что он сам — деятель. Однако тот, кто знает о связи сил Природы с действиями, видит, как
одни силы Природы оказывают воздействие на другие силы Природы, и избегает участи их раба\frqq.}

Нынче карма трактуется \flqq ближе к жизни\frqq: пока (находясь под чарами майи) мы думаем, что мы существуем отдельно от
окружающей нас среды и можем действовать свободно и независимо, мы сковываем себя кармой. Чтобы освободиться от этих
уз, нужно осознать целостность и гармонию, царящие в природе, включая и нас самих, и действовать в соответствии с этим.
Это означает прочувствовать всем своим существом, что всё, включая нас самих, есть Брахман. Это ощущение называется
\textit{мокша} (освобождение), оно составляет основное содержание жизненной практики индуизма.


Не оглянувшись, уходит свободный


И так покидает печали,


Как с ветки, упавшей в воду,


Вспорхнув, улетает птица.

Важный и популярный путь к освобождению — \textit{йога} (сопрягать, соединять индивидуальную душу с Брахманом). Разные
школы йоги используют различные психологические практики, привлекающие людей различного характера и различного уровня
развития.


Неуверенность, неудовлетворенность, страдания – результат отчуждения, то есть результат утраты \ идентичности. Эта
трагедия не монополизирована Востоком.

Для большинства индуистов слияние с Божественным заключается в почитании какого-либо персонифицированного бога или
богини. Три наиболее популярных божества индуизма — Шива, Вишну и Божественная Мать.

\textbf{Шива} — один из древних индийских богов, предстает во многих обличиях. \textbf{Вишну} тоже имеет много обличий,
одно из которых — Кришна из Бхагавадгиты. Вишну сохраняет Вселенную. \textbf{Шакти} (Божественная Мать) — воплощает
женское начало и женскую энергию Вселенной. Шакти также выступает в роли жены Шивы, и изображения двух страстно
обнимающихся божеств часто можно видеть в храмовых скульптурных произведениях искусства, которые отличаются
удивительной чувственностью, совершенно изгнанной из церковного искусства Запада.

В \textit{тантризме} путь к освобождению предложен через глубокое погружение в переживания чувственной любви, в которой
\textit{каждый воплощает в себе обоих}:


\flqq Подобно тому, как мужчина в объятиях любимой жены не сознает ничего ни внутри, ни снаружи, так и освободившийся человек
в объятиях разумной Души не сознает ничего ни внутри, ни снаружи\frqq.

Богини изображаются не в облике святых дев, а в объятиях своих божественных супругов.

\subsection[Начало мира]{Начало мира}
\flqq Не было не-сущего и не было сущего тогда, не было ни воздушного пространства, ни неба над ним. Что двигалось туда и
сюда? Где? Под чьей защитой? Что за вода – глубокая бездна?

Не было ни смерти, ни бессмертия тогда, не было признака ни дня, ни ночи. Дышало, не колебля воздуха, по своему замыслу
Нечто Одно, и не было ничего другого, кроме него. Неподвижное, Единое, оно – быстрее мысли; чувства не достигают его,
оно двигалось впереди их. Стоя, оно обгоняет других – бегущих; оно движется – оно не движется; оно далеко – оно же и
близко; оно внутри всего – оно же вне всего. Поистине, кто видит всех существ в Атмане и Атмана во всех существах, тот
больше не страшится.

Когда для распознающего Атман стал всеми существами, то какое ослепление, какая печаль могут быть у зрящего единство?

Он простирается всюду — светлый, бестелесный, неранимый, лишенный жил, чистый, неуязвимый для зла, всеведущий, мыслящий,
вездесущий, самосущий — тот, кто должным образом распределил по своим местам все вещи на вечные времена. \ \ \ \ А была
мысль.\frqq

\textbf{Эволюция мира}. Мир эволюционирует, проходя циклы. Наименьшей единицей космического времени является
\textit{юга}. В основной цикл (\textit{махаюга} = великая юга) длительностью 4 320 000 лет входят четыре юги, каждая из
которых короче предыдущей, что определено убыванием (на ¼ ) дхармы (морального порядка). \textit{Крита юга} (эра
совершенства) длится 1 728 000 лет; \textit{трета юга} – 1~296~000 лет; \textit{двапара юга} – 864 000 лет; последняя
эра, \textit{кали юга}, в которой только ¼ начальной дхармы, продолжается 432 000 лет. Конец \textit{кали юги}
ознаменуется распадом социума, прекращением богопочитания, распространением неуважительного отношения к священному
писанию, к мудрецам и моральным принципам, произойдут наводнения, пожары и войны — и начнется новая \textit{махаюга}.

Тысяча махаюг составляют одну \textit{кальпу} – один день в жизни Брахмы. В конце каждого такого дня вся материя
вселенной поглощается универсальным духом и в течение ночи Брахмы, которая длится также одну кальпу, эта материя
существует только в потенции и ожидает восстановления. На заре каждого своего дня Брахма является из лотоса, растущего
из пупа бога Вишну, и материя формируется вновь.

Индуизм принимает три уровня понимания картины мира: мудрым дана веданта, средним отведены добродетели и боги, удел
низших – предписания, обряды, культ, поведение (на Западе это оценивают как профанацию). Античный Запад полагал, что
мир Востока – мир Упанишад.

\textbf{Цели человека.} Жизнь человека имеет четыре цели. Первые две цели: \textit{артха} – богатство и власть, и
\textit{кама} – наслаждение и удовлетворение желаний, прежде всего любовных. Артха и кама уступают по значимости двум
другим целям жизни: \textit{дхарме} – правильному поведению, и \textit{мокше} – освобождению от цикла нескончаемых
перерождений.

\subsection[У ног Учителя]{У ног Учителя}
\textit{Пространство}.В нем Солнце и Луна, молния, звезды, огонь. Зовут, слышат, отвечают через пространство. Рождаются
в пространстве, рождаются в пространство. \textit{Почитай же Пространство.}

\textit{Память больше, чем пространство}. Сидящие здесь, не имея памяти, ничего бы не слышали, ничего бы не мыслили,
ничего бы не познавали. \textit{Почитай же Память.}

\textit{Надежда больше, чем память}.

\textit{Движение больше, чем надежда}.

Дыхание – это отец, мать, брат, сестра, учитель, брахман. Если кто-либо скажет о них нечто недостойное, то ему говорят:
\flqq Стыдись, ты поистине убийца…\frqq

Тот, кто видит это, мыслит это, познает это, становится победителем в спорах. Пусть не оспаривает он этого. Тот
побеждает в спорах, кто побеждает с помощью истины. Непознающий не говорит истины, лишь познающий говорит истину.

— Я знаю Ригведу, Яджурведу, другие Веды.

— Поистине, лишь имена суть они. \textit{Почитай же Имя}.

Тот, кто почитает имя как Брахмана, может действовать по своему усмотрению в пределах того, что имя объемлет.

\textit{Речь больше, чем имя}. Посредством речи познаются Веды. Без речи нельзя было бы познать справедливость и ее
отрицание, истину и ее отрицание, доброе и его отрицание. \textit{Почитай же Речь}.

\textit{Мысль больше, чем речь}. Как сжатая рука охватывает два плода, так мысль охватывает и речь, и имя. Мысль – это
Атман, это – мир, это – Брахман. \textit{Почитай же Мысль}.

\textit{Замысел больше мысли}. Кто замышляет, тот мыслит, побуждает речь, побуждает ее в имени. В имени приходят в
единство мантры\footnote{Мантра = гимн.}. Замысел – их Атман, их опора. \textit{Почитай же Замысел}.

\textit{Ум больше замысла}. Если кто-либо обладает умом, он замышляет.

Если кто-либо много знает, но не умен, то он не умен, что бы он ни знал: если бы он был поистине знающим, он не был бы
столь неумен.

\textit{Размышление больше ума}. Те из людей, кто достигает величия в этом мире, размышляют. Мелкие люди вздорны,
злоречивы и злобны. Могущественные причастны размышлению. \textit{Почитай же Размышление}.

\textit{Познание больше размышления}: Веды, счет, искусство предсказаний, летосчисление познаются посредством познания.
\textit{Почитай же Познание}.

\textit{Сила больше познания}. Один сильный приводит в трепет сотню знающих. Земля, воздушное пространство, воды, горы,
боги и люди, скот, птицы, травы и деревья, звери, комары и муравьи — существуют благодаря \textit{силе}. Сильный
становится распрямляющимся, деятельным, странствующим, \textbf{садящимся у ног Учителя}, видящим, мыслящим, понимающим,
действующим, познающим.


Почитай же Силу.

\subsection[Страничка из хрестоматии]{Страничка из хрестоматии}
Исходный текст в поэтическом переводе на русский язык:

Дорога вся в лунном серебре.\newline
Родиться бы сосною на горе.

Исходный текст в поэтическом переводе на немецкий язык:

Ein Fichtenbaum steht einsam\newline
In Norden auf kahler Höh.

Русская вариация на тему немецкого текста:

На севере диком стоит одиноко\newline
На горной вершине сосна…

\section[Восток пришел к нам с Запада]{Восток пришел к нам с Запада}
\section[Темы для обсуждения]{Темы для обсуждения}
\begin{enumerate}
\item За какое время Восток преодолеет отставание от Запада?
\item К какому восточному учению стоит примкнуть?
\item Чем \ отличаются Запад и Восток?
\item Соединятся ли Запад и Восток?
\item От чего на самом деле освобождает йога?
\end{enumerate}
%\end{document}
